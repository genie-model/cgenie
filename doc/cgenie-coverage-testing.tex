\documentclass[a4paper,10pt,article]{memoir}

\usepackage[a4paper,margin=1in]{geometry}

\usepackage[utf8]{inputenc}
\usepackage{fourier}
\usepackage{amsmath,amssymb}

\usepackage{color}
\definecolor{orange}{rgb}{0.75,0.5,0}
\definecolor{magenta}{rgb}{1,0,1}
\definecolor{cyan}{rgb}{0,1,1}
\definecolor{grey}{rgb}{0.25,0.25,0.25}
\newcommand{\outline}[1]{{\color{grey}{\scriptsize #1}}}
\newcommand{\revnote}[1]{{\color{red}\textit{\textbf{#1}}}}
\newcommand{\note}[1]{{\color{blue}\textit{\textbf{#1}}}}
\newcommand{\citenote}[1]{{\color{orange}{[\textit{\textbf{#1}}]}}}

\usepackage{xspace}

\usepackage{listings}
\usepackage{courier}
\lstset{basicstyle=\tiny\ttfamily,breaklines=true,language=Python}

\usepackage{fancyvrb}
\usepackage{url}

\usepackage{float}
\newfloat{listing}{tbp}{lop}
\floatname{listing}{Listing}

\title{Coverage testing for GENIE \texttt{cupcake}}
\author{Ian~Ross}
\date{4 March 2015}

\begin{document}

\catcode`~=11    % Make tilde a normal character (danger of weirdness...)
%\catcode`~=13    % Make tilde an active character again

\maketitle

The idea here is to collect a set of tests to provide near-complete
coverage of the GENIE codebase, in order to help with preventing
inadvertent science changes when making wide-ranging code
transformations for optimisation.  There's a \texttt{coverage} script
in the main \texttt{cgenie} directory that compiles and runs a set of
tests with compiler flags suitable for coverage testing, then collects
coverage data using the \texttt{gcov} program and finally merges the
coverage data for the different test runs (you can't use
\texttt{gcov}'s cumulative coverage collection facility to do this
because the executables for different jobs may be different, depending
on the model resolution and other preprocessor settings).

The final results are stored in the \texttt{gcov-results} directory of
the top-level directory in \texttt{~/cgenie-jobs} used by the coverage
analysis script.  There is one coverage file for each Fortran 90
source file in GENIE, with each line annotated with the maximum number
of ``visits'' in any single simulation from the set of test jobs used.
Non-executable lines are annotated with a hyphen and executable lines
that were never executed are marked with ``\texttt{\#\#\#\#\#}''.
These latter lines are the most important to consider.

%======================================================================
\chapter{Step \#1: initial setup and experiments}

%----------------------------------------------------------------------
\section{Test jobs}

The first set of tests includes the following jobs:
\begin{description}
  \item[\texttt{ocean-atmos}]{Basic ocean-atmosphere test (the
    ``\texttt{testebgogs}'' job) [10 years].}
  \item[\texttt{biogem}]{Basic BIOGEM test (the
    ``\texttt{testbiogem}'' job) [10 years].}
  \item[\texttt{make-restart}]{Restart setup test (the
    ``\texttt{restartmake}'' job from the old \texttt{fruitcake} test
    jobs [10 years].}
  \item[\texttt{restart-read}]{Restart read test (the
    ``\texttt{restartread}'' job from the old \texttt{fruitcake} test
    jobs [10 years].}
  \item[\texttt{ocean-geochem-spin-up}]{Includes a large number of
    ocean tracers, and different use of sediments [100 years].}
  \item[\texttt{ocean-geochem-spin-up-2}]{Follows from above, testing
    restarts, and uses a fuller range of ROKGEM features [10000
      years].}
  \item[\texttt{cao-et-al-2009}]{'Gold standard' modern configuration,
    including climate components and atmospheric and oceanic
    biogeochemistry [10 years].}
  \item[\texttt{ridgwell-schmidt-2010}]{Well used paleo configuration,
    including sediments and basic ROKGEM [10 years].}
  \item[\texttt{orbital-variations-abiotic-ocean}]{Very basic
    climate-only test, but including some additional (orbital)
    features and a very different continental configuration [100
      years].}
\end{description}

%----------------------------------------------------------------------
\section{Results (subroutine level)}

Uncalled subroutines:
\begin{description}
  \item[\texttt{atchem/atchem\_box.f90}]{\texttt{sub\_calc\_terrCO2exchange},
    \texttt{sub\_calc\_oxidize\_CH4}, \\
    \texttt{sub\_calc\_wetlands\_CH4}}

  \item[\texttt{atchem/cpl\_flux\_atchem.f90}]{\texttt{cpl\_flux\_lndatm}}

  \item[\texttt{biogem/biogem\_box.f90}]{
    \texttt{sub\_calc\_bio\_uptake\_abio},
    \texttt{sub\_calc\_geochem\_Fe}, \\
    \texttt{sub\_box\_oxidize\_NH4toNO3},
    \texttt{sub\_box\_oxidize\_NH4toNO2},
    \texttt{sub\_box\_oxidize\_NO2}, \\ \texttt{sub\_box\_reduce\_NO2},
    \texttt{sub\_calc\_bio\_remin\_oxidize\_CH4},
    \texttt{sub\_box\_scav\_Fe}, \\ \texttt{sub\_calc\_scav\_Fe},
    \texttt{sub\_box\_scav\_H2S},
    \texttt{sub\_calc\_misc\_brinerejection}, \\
    \texttt{sub\_update\_force\_flux\_atm},
    \texttt{sub\_update\_force\_flux\_sed},
    \texttt{sub\_audit\_update}}

  \item[\texttt{biogem/biogem\_data\_ascii.f90}]{
    \texttt{sub\_data\_save\_global\_snap},
    \texttt{sub\_echo\_maxmin}}

  \item[\texttt{biogem/biogem\_data.f90}]{\texttt{sub\_init\_misc2D},
    \texttt{sub\_init\_audit}, \texttt{sub\_data\_audit\_diagnostics},
    \texttt{sub\_init\_force\_solconst}}

  \item[\texttt{biogem/biogem\_data\_netCDF.f90}]{
    \texttt{sub\_save\_netcdf\_flux\_seaair},
    \texttt{sub\_save\_netcdf\_3d\_sig}, \\
    \texttt{sub\_save\_netcdf\_ocn\_col\_extra},
    \texttt{sub\_save\_netcdf\_runtime}}

  \item[\texttt{common/gem\_netcdf.f90}]{\texttt{sub\_defvar\_scalar}}

  \item[\texttt{common/gem\_util.f90}]{\texttt{sub\_load\_data\_ijk}}

  \item[\texttt{embm/embm\_diag.f90}]{\texttt{diaga}, \texttt{aminmax},
    \texttt{diagend\_embm}, \texttt{diagfna}, \texttt{diagosc\_embm}, \\
    \texttt{read\_embm\_target\_field}}

  \item[\texttt{embm/embm.f90}]{\texttt{field\_interp},
    \texttt{ocean\_alb}, \texttt{rad\_out}}

  \item[\texttt{ents/ents\_data.f90}]{\texttt{in\_ents\_ascii},
    \texttt{in\_ents\_netcdf}, \texttt{out\_ents}}

  \item[\texttt{ents/ents\_diag.f90}]{\texttt{annav\_diags},
    \texttt{carbt\_diags}, \texttt{physt\_diags},
    \texttt{screen\_diags}, \texttt{aminmaxl},
    \texttt{slnd\_h2o\_invent}, \texttt{slnd\_c\_invent},
    \texttt{entsdiagosc}}

  \item[\texttt{ents/ents.f90}]{\texttt{initialise\_ents},
    \texttt{setup\_ents}, \texttt{step\_ents}, \texttt{sealevel},
    \texttt{carbon}, \texttt{photosynthesis}, \texttt{veg\_resp},
    \texttt{leaf\_litter}, \texttt{soil\_resp}}

  \item[\texttt{ents/ents\_netcdf.f90}]{\texttt{netcdf\_ents},
    \texttt{netcdf\_db\_ents}, \texttt{netcdf\_ts\_ents}}

  \item[\texttt{goldstein/goldstein\_data.f90}]{\texttt{outm\_surf\_ocn\_sic}}

  \item[\texttt{goldstein/goldstein\_diag.f90}]{\texttt{diag}, \texttt{diagend}}

  \item[\texttt{goldstein/goldstein.f90}]{\texttt{coshuffle}, \texttt{ediff},
    \texttt{krausturner}}

  \item[\texttt{goldsteinseaice/gold\_seaice\_data.f90}]{\texttt{diagsic},
    \texttt{diagosc\_sic}}

  \item[\texttt{goldsteinseaice/gold\_seaice.f90}]{\texttt{diagend\_seaice},
    \texttt{tstipsic}}

  \item[\texttt{rokgem/rokgem\_box.f90}]{\texttt{sub\_GKWM},
    \texttt{sub\_GEM\_CO2}, \texttt{sum\_calcium\_flux},
    \texttt{sum\_calcium\_flux\_CaSi}, \texttt{sub\_2D\_weath}}

  \item[\texttt{rokgem/rokgem\_data.f90}]{
    \texttt{sub\_load\_rokgem\_restart},
    \texttt{sub\_data\_input\_3D}, \texttt{sub\_load\_weath}}

  \item[\texttt{rokgem/rokgem\_data\_netCDF.f90}]{\texttt{rokgem\_netcdf},
    \texttt{sub\_save\_netcdf\_2d\_rg}, \texttt{sub\_save\_netcdf}}

  \item[\texttt{rokgem/rokgem\_lib.f90}]{\texttt{define\_2D\_arrays}}

  \item[\texttt{sedgem/sedgem\_box.f90}]{\texttt{sub\_update\_sed\_mud},
    \texttt{calc\_sed\_dis\_opal}}

  \item[\texttt{sedgem/sedgem\_box\_ridgwell2001\_sedflx.f90}]{
    \texttt{init\_sedflx\_Si}, \texttt{runge\_kutta\_4\_opal}}

  \item[\texttt{sedgem/sedgem\_box\_ridgwelletal2003\_sedflx.f90}]{
    \texttt{init\_sedflx\_Si}, \texttt{runge\_kutta\_4\_opal}}

  \item[\texttt{sedgem/sedgem\_data.f90}]{
    \texttt{sub\_load\_sed\_dis\_lookup\_CaCO3},
    \texttt{sub\_load\_sed\_dis\_lookup\_opal}, \\
    \texttt{sub\_load\_sed\_mix\_k},
    \texttt{sub\_sedgem\_save\_sedcore},
    \texttt{sub\_data\_save\_seddiag\_2D}, \\
    \texttt{sub\_data\_output\_years}, \texttt{sub\_output\_year},
    \texttt{sub\_output\_counters}}

  \item[\texttt{utils/writenc6.f90}]{\texttt{writevar\_1d}}

  \item[\texttt{wrappers/genie\_ini\_wrappers.f90}]{
    \texttt{initialise\_ents\_wrapper}}

  \item[\texttt{wrappers/genie\_loop\_wrappers.f90}]{
    \texttt{ents\_wrapper}, \texttt{cpl\_flux\_lndatm\_wrapper}, \\

    \texttt{diag\_biogem\_gem\_wrapper}}

  \item[\texttt{genie\_util.f90}]{\texttt{die}, \texttt{message}}

  \item[\texttt{local\_netcdf.f90}]{\texttt{lookupREAL1dVars},
    \texttt{lookupREAL2dVars}, \texttt{lookupREAL3dVars},
    \texttt{openNetCDFRead}, \texttt{closeNetCDF},
    \texttt{handle\_err}, \texttt{handle\_nc\_err}}
\end{description}

%----------------------------------------------------------------------
\section{Coverage conditions (subroutine level)}

\subsubsection*{\texttt{atchem/atchem\_box.f90}}

\paragraph{\texttt{sub\_calc\_terrCO2exchange}}

\texttt{atchem/atchem.f90:69}

\begin{verbatim}
  IF ((par_atm_FterrCO2exchange > const_real_nullsmall) .AND.
    atm_select(ia_pCO2) .AND. atm_select(ia_pCO2_13C)) THEN
\end{verbatim}

\paragraph{\texttt{sub\_calc\_oxidize\_CH4}}

\texttt{atchem/atchem.f90:54}

\begin{verbatim}
  IF (atm_select(ia_pCH4) .AND. atm_select(ia_pCO2) .AND.
    atm_select(ia_pO2)) THEN
\end{verbatim}

\paragraph{\texttt{sub\_calc\_wetlands\_CH4}}

\texttt{atchem/atchem.f90:59}

\begin{verbatim}
  IF (atm_select(ia_pCH4) .AND. atm_select(ia_pCO2) .AND.
    atm_select(ia_pO2)) THEN
\end{verbatim}

\subsection*{\texttt{atchem/cpl\_flux\_atchem.f90}}

\paragraph{\texttt{cpl\_flux\_lndatm}}

\texttt{genie.f90:210}

\begin{verbatim}
  IF (flag_ents) THEN
\end{verbatim}

\subsection*{\texttt{biogem/biogem\_box.f90}}

\paragraph{\texttt{sub\_calc\_bio\_uptake\_abio}}

\texttt{biogem/biogem.f90:1171}

\begin{verbatim}
  if (ctrl_bio_CaCO3precip .AND. sed_select(is_CaCO3)) then
\end{verbatim}

\paragraph{\texttt{sub\_calc\_geochem\_Fe}}

\texttt{biogem/biogem.f90:1185}

\begin{verbatim}
  if (sed_select(is_det) .AND. ocn_select(io_Fe)) then
\end{verbatim}

\paragraph{\texttt{sub\_box\_oxidize\_NH4toNO3} \& \texttt{sub\_box\_oxidize\_NH4toNO2}}

\texttt{biogem/biogem.f90:1140}

\begin{verbatim}
  if (ocn_select(io_O2) .AND. ocn_select(io_NO3) .AND.
    ocn_select(io_NH4)) then
     if (ocn_select(io_NO2)) then
        call sub_box_oxidize_NH4toNO2(i,j,loc_k1,loc_dtyr)
     else
        call sub_box_oxidize_NH4toNO3(i,j,loc_k1,loc_dtyr)
     end if
  end If
\end{verbatim}

\paragraph{\texttt{sub\_calc\_bio\_remin\_oxidize\_CH4}}

\texttt{biogem/biogem.f90:1158}

\begin{verbatim}
  if (ocn_select(io_O2) .AND. ocn_select(io_CH4)) then
\end{verbatim}

\paragraph{\texttt{sub\_box\_scav\_Fe}}

\texttt{biogem/biogem\_box.f90:2799}

\begin{verbatim}
  if (ocn_select(io_Fe)) then
     if (dum_vocn%mk(io2l(io_Fe),kk) > const_real_nullsmall) then
\end{verbatim}

\paragraph{\texttt{sub\_calc\_scav\_Fe}}

\texttt{biogem/biogem\_box.f90:1420}

\begin{verbatim}
  if (ocn_select(io_Fe)) then
     DO k=n_k,loc_k_mld,-1
        if (ocn(io_Fe,dum_i,dum_j,k) > const_real_nullsmall) then
\end{verbatim}

\paragraph{\texttt{sub\_box\_scav\_H2S}}

\texttt{biogem/biogem\_box.f90:2812}

\begin{verbatim}
  if (ocn_select(io_H2S) .AND. sed_select(is_POM_S)) then
     if (dum_vocn%mk(io2l(io_H2S),kk) > const_real_nullsmall) then
\end{verbatim}

\paragraph{\texttt{sub\_calc\_misc\_brinerejection}}

\texttt{biogem/biogem.f90:490}

\begin{verbatim}
  if (par_misc_brinerejection_frac > const_real_nullsmall) then
\end{verbatim}

\paragraph{\texttt{sub\_update\_force\_flux\_atm}}

\texttt{biogem/biogem.f90:1573}

\begin{verbatim}
  IF (force_flux_atm_select(ia)) THEN
\end{verbatim}

\paragraph{\texttt{sub\_update\_force\_flux\_sed}}

\texttt{biogem/biogem.f90:1590}

\begin{verbatim}
  IF (force_flux_sed_select(is)) THEN
\end{verbatim}

\paragraph{\texttt{sub\_audit\_update}}

\texttt{biogem/biogem.f90:1923}

\begin{verbatim}
  IF (ctrl_audit) THEN
\end{verbatim}


\subsection*{\texttt{biogem/biogem\_data\_ascii.f90}}

\paragraph{\texttt{sub\_data\_save\_global\_snap}}

\texttt{biogem/biogem/f90:2208}

\begin{verbatim}
  If (ctrl_data_save_GLOBAL .AND. ctrl_data_save_derived)
\end{verbatim}

\paragraph{\texttt{sub\_echo\_maxmin}}

\texttt{biogem/biogem.f90:1923}

\begin{verbatim}
  IF (ctrl_audit) THEN
\end{verbatim}


\subsection*{\texttt{biogem/biogem\_data.f90}}

\paragraph{\texttt{sub\_init\_misc2D}}

\texttt{biogem/initialise\_biogem.f90:248}

\begin{verbatim}
  IF (trim(opt_misc_geoeng) /= 'NONE') THEN
     ! initialize geoengineering
     IF (ctrl_debug_lvl2) print*, 'initialize geoengineering'
\end{verbatim}

\paragraph{\texttt{sub\_init\_audit}}

\texttt{biogem/initialise\_biogem.f90:227}

\begin{verbatim}
  IF (ctrl_audit) CALL sub_init_audit()
\end{verbatim}

\paragraph{\texttt{sub\_data\_audit\_diagnostics}}

\texttt{biogem/end\_biogem.f90:14}

\begin{verbatim}
  IF (ctrl_audit) THEN
\end{verbatim}

\paragraph{\texttt{sub\_init\_force\_solconst}}

\texttt{biogem/initialise\_biogem.f90:289}

\begin{verbatim}
  if (ctrl_force_solconst) call sub_init_force_solconst()
\end{verbatim}


\subsection*{\texttt{biogem/biogem\_data\_netCDF.f90}}

\paragraph{\texttt{sub\_save\_netcdf\_flux\_seaair}}

\texttt{biogem/biogem\_data\_netCDF.f90:1187}

\begin{verbatim}
  IF (ctrl_data_save_slice_fairsea) then
     CALL sub_save_netcdf_flux_seaair()
  end if
\end{verbatim}

\paragraph{\texttt{sub\_save\_netcdf\_3d\_sig}}

\texttt{biogem/biogem.f90:2663}

\begin{verbatim}
  if (ctrl_data_save_3d_sig) then
     call sub_save_netcdf(loc_yr_save,4)
     CALL sub_save_netcdf_3d_sig()
\end{verbatim}

\paragraph{\texttt{sub\_save\_netcdf\_ocn\_col\_extra}}

\texttt{biogem/biogem\_data\_netCDF.f90:2028}

\begin{verbatim}
  IF (ctrl_data_save_derived) THEN
     ! color tracer ratios
     IF (ocn_select(io_colr) .AND. ocn_select(io_colb)) then
        CALL sub_save_netcdf_ocn_col_extra()
     END IF
  END IF
\end{verbatim}

\paragraph{\texttt{sub\_save\_netcdf\_runtime}}

\texttt{biogem/biogem.f90:2652}

\begin{verbatim}
  IF (int_t_sig > const_real_nullsmall) then
     IF (ctrl_data_save_sig_ascii) then
        CALL sub_data_save_runtime(loc_yr_save)
     else
        CALL sub_save_netcdf_runtime(loc_yr_save)
     end IF
\end{verbatim}


\subsection*{\texttt{common/gem\_netcdf.f90}}

\paragraph{\texttt{sub\_defvar\_scalar}}

Called from \texttt{sub\_save\_netcdf\_runtime} in
\texttt{biogem/biogem\_data\_netCDF.f90}.


\subsection*{\texttt{common/gem\_util.f90}}

\paragraph{\texttt{sub\_load\_data\_ijk}}

\texttt{biogem/biogem\_data.f90:672} + others

\begin{verbatim}
  if (ctrl_force_scav_fpart_POC) then
\end{verbatim}


\subsection*{\texttt{embm/embm\_diag.f90}}

All of these just need EMBM diagnostics to be enabled by setting the
\texttt{debug\_loop} and/or \texttt{debug\_end} flags:
\textbf{\texttt{diaga}}, \textbf{\texttt{aminmax}},
\textbf{\texttt{diagend\_embm}}, \textbf{\texttt{diagfna}},
\textbf{\texttt{diagosc\_embm}},
\textbf{\texttt{read\_embm\_target\_field}}.


\subsection*{\texttt{embm/embm.f90}}

\paragraph{\texttt{field\_interp}}

\texttt{embm/embm.f90:1620}

\begin{verbatim}
  IF (flag_ents) THEN
\end{verbatim}

\paragraph{\texttt{ocean\_alb}, \texttt{rad\_out}}

\texttt{embm/embm.f90:2258}

\begin{verbatim}
  IF (flag_ents) CALL ocean_alb(oscss, osccc, oscday, j, istep)
\end{verbatim}


\subsection*{\texttt{ents/ents\_data.f90}}

Need ENTS to be enabled: \textbf{\texttt{in\_ents\_ascii}},
\textbf{\texttt{in\_ents\_netcdf}}, \textbf{\texttt{out\_ents}}.


\subsection*{\texttt{ents/ents\_diag.f90}}

Need ENTS to be enabled: \textbf{\texttt{annav\_diags}},
\textbf{\texttt{carbt\_diags}}, \textbf{\texttt{physt\_diags}},
\textbf{\texttt{screen\_diags}}, \textbf{\texttt{aminmaxl}},
\textbf{\texttt{slnd\_h2o\_invent}},
\textbf{\texttt{slnd\_c\_invent}}, \textbf{\texttt{entsdiagosc}}.


\subsection*{\texttt{ents/ents.f90}}

Need ENTS to be enabled: \textbf{\texttt{initialise\_ents}},
\textbf{\texttt{setup\_ents}}, \textbf{\texttt{step\_ents}},
\textbf{\texttt{sealevel}}, \textbf{\texttt{carbon}},
\textbf{\texttt{photosynthesis}}, \textbf{\texttt{veg\_resp}},
\textbf{\texttt{leaf\_litter}}, \textbf{\texttt{soil\_resp}}.


\subsection*{\texttt{ents/ents\_netcdf.f90}}

Need ENTS to be enabled: \textbf{\texttt{netcdf\_ents}},
\textbf{\texttt{netcdf\_db\_ents}},
\textbf{\texttt{netcdf\_ts\_ents}}.


\subsection*{\texttt{goldstein/goldstein\_diag.f90}}

These just need GOLDSTEIN diagnostics to be enabled by setting the
\texttt{debug\_loop} and/or \texttt{debug\_end} flags:
\textbf{\texttt{diag}}, \textbf{\texttt{diagend}}.


\subsection*{\texttt{goldstein/goldstein.f90}}

\paragraph{\texttt{coshuffle}}

\texttt{goldstein/goldstein.f90:2477}

\begin{verbatim}
  IF (iconv == 1) THEN
\end{verbatim}

\paragraph{\texttt{ediff}}

\texttt{goldstein/goldstein.f90:1870}

\begin{verbatim}
  IF (iediff > 0) CALL ediff
\end{verbatim}

\paragraph{\texttt{krausturner}}

\texttt{goldstein/goldstein.f90:2130}

\begin{verbatim}
  IF (imld == 1) THEN
     ...
     DO i = 1, imax
        DO j = 1, jmax
           IF (k1(i,j) <= kmax) THEN
              IF (mldemix(i,j) > 0.0) THEN
\end{verbatim}


\subsection*{\texttt{goldsteinseaice/gold\_seaice\_data.f90}}

These just need diagnostics to be enabled by setting the
\texttt{debug\_loop} flag: \textbf{\texttt{diagsic}},
\textbf{\texttt{diagosc\_sic}}.


\subsection*{\texttt{goldsteinseaice/gold\_seaice.f90}}

This just needs diagnostics to be enabled by setting the
\texttt{debug\_end} flag: \textbf{\texttt{diagend\_seaice}}.

\paragraph{\texttt{tstipsic}}

\texttt{goldsteinseaice/gold\_seaice.f90:538}

\begin{verbatim}
  IF (impsic) THEN
\end{verbatim}


\subsection*{\texttt{rokgem/rokgem\_box.f90}}

\paragraph{\texttt{sub\_GKWM}}

\texttt{rokgem/rokgem.f90:65}

\begin{verbatim}
  SELECT case (par_weathopt)
  ...
  case ('GKWM')
\end{verbatim}

Also \textbf{\texttt{sub\_GEM\_CO2}},
\textbf{\texttt{sum\_calcium\_flux\_CaSi}} and
\textbf{\texttt{sub\_2D\_weath}}.


\subsection*{\texttt{rokgem/rokgem\_data.f90}}

\paragraph{\texttt{sub\_load\_rokgem\_restart}}

\texttt{rokgem/initialise\_rokgem.f90:72}

\begin{verbatim}
  IF (ctrl_continuing.AND.opt_append_data) THEN
\end{verbatim}

\paragraph{\texttt{sub\_load\_weath}}

\texttt{rokgem/initialise\_rokgem.f90:234}

\begin{verbatim}
  IF (par_weathopt.ne.'Global_avg') THEN
\end{verbatim}

Also \textbf{\texttt{sub\_data\_input\_3D}}.


\subsection*{\texttt{rokgem/rokgem\_data\_netCDF.f90}}

\paragraph{\texttt{rokgem\_netcdf}}

\texttt{rokgem/rokgem\_box.f90:972}

\begin{verbatim}
  IF (tstep_count.eq.output_tsteps_2d(output_counter_2d)) THEN
     IF (opt_2d_netcdf_output) THEN
\end{verbatim}

Also \textbf{\texttt{sub\_save\_netcdf\_2d\_rg}} and
\textbf{\texttt{sub\_save\_netcdf}}.


\subsection*{\texttt{rokgem/rokgem\_lib.f90}}

\paragraph{\texttt{define\_2D\_arrays}}

\texttt{rokgem/initialise\_rokgem.f90:234}

\begin{verbatim}
  IF (par_weathopt.ne.'Global_avg') THEN
\end{verbatim}


\subsection*{\texttt{sedgem/sedgem\_box.f90}}

\paragraph{\texttt{sub\_update\_sed\_mud}}

\texttt{sedgem/sedgem.f90:267}

\begin{verbatim}
  elseif (sed_mask_muds(i,j)) then
\end{verbatim}

\paragraph{\texttt{calc\_sed\_dis\_opal}}

\texttt{sedgem/sedgem\_box.f90:346}

\begin{verbatim}
  select case (par_sed_diagen_opalopt)
  case ('ridgwelletal2003lookup')
\end{verbatim}


\subsection*{\texttt{sedgem/sedgem\_box\_ridgwelletal2003\_sedflx.f90}}

\paragraph{\texttt{init\_sedflx\_Si}}

\texttt{sedgem/sedgem\_box.f90:1789}

\begin{verbatim}
  select case (par_sed_diagen_opalopt)
  case ('ridgwelletal2003explicit')
\end{verbatim}

Also \textbf{\texttt{runge\_kutta\_4\_opal}}.


\subsection*{\texttt{sedgem/sedgem\_data.f90}}

\paragraph{\texttt{sub\_load\_sed\_dis\_lookup\_CaCO3}}

\texttt{sedgem/sedgem\_data.f90:672}

\begin{verbatim}
  if (par_sed_diagen_CaCO3opt == 'ridgwell2001lookup' .OR.
      par_sed_diagen_CaCO3opt == 'ridgwell2001lookupvec') then
\end{verbatim}

\paragraph{\texttt{sub\_load\_sed\_dis\_lookup\_opal}}

\texttt{sedgem/sedgem\_data.f90:711}

\begin{verbatim}
  if (par_sed_diagen_opalopt == 'ridgwelletal2003lookup') then
\end{verbatim}

\paragraph{\texttt{sub\_load\_sed\_mix\_k}}

\texttt{sedgem/initialise\_sedgem.f90:104}

\begin{verbatim}
  IF (ctrl_sed_bioturb) THEN
     if (ctrl_sed_bioturb_Archer) then
        ALLOCATE(par_sed_mix_k(0:par_n_sed_mix),STAT=error)
     else
\end{verbatim}

\paragraph{\texttt{sub\_sedgem\_save\_sedcore}}

\texttt{sedgem/end\_sedgem.f90:44}

\begin{verbatim}
  if (ctrl_data_save_ascii) call sub_sedgem_save_sedcore()
\end{verbatim}

\paragraph{\texttt{sub\_data\_save\_seddiag\_2D}}

\texttt{sedgem/end\_sedgem.f90:47}

\begin{verbatim}
  if (ctrl_data_save_ascii) call sub_data_save_seddiag_2D(loc_dtyr,dum_sfcsumocn)
\end{verbatim}

\paragraph{\texttt{sub\_data\_output\_years}}

\texttt{sedgem/initialise\_sedgem.f90:137}

\begin{verbatim}
  IF (ctrl_timeseries_output) THEN
\end{verbatim}

\paragraph{\texttt{sub\_output\_year}}

\texttt{sedgem/sedgem.f90:461}

\begin{verbatim}
  IF (ctrl_timeseries_output) THEN
\end{verbatim}

Also \textbf{\texttt{sub\_output\_counters}}.


\subsection*{\texttt{wrappers/genie\_ini\_wrappers.f90}}

\paragraph{\texttt{initialise\_ents\_wrapper}}

Need to use ENTS!


\subsection*{\texttt{wrappers/genie\_loop\_wrappers.f90}}

Need to use ENTS: \textbf{\texttt{ents\_wrapper}},
\textbf{\texttt{cpl\_flux\_lndatm\_wrapper}}.

\paragraph{\texttt{diag\_biogem\_gem\_wrapper}}

Needs \texttt{debug\_loop} flag to be set.


\subsection*{\texttt{genie\_util.f90}}

Ignore: \textbf{\texttt{die}}, \textbf{\texttt{message}}.


\subsection*{\texttt{local\_netcdf.f90}}

\paragraph{\texttt{lookupREAL1dVars}}

\texttt{embm/embm\_diag.f90:447}

\begin{verbatim}
  IF (.NOT. interpolate) THEN
  ...
  ELSE
\end{verbatim}

Also \textbf{\texttt{lookupREAL2dVars}},
\textbf{\texttt{openNetCDFRead}} and \textbf{\texttt{closeNetCDF}}.

\paragraph{\texttt{lookupREAL3dVars}}

\texttt{goldstein/goldsteain\_diag.f90:703}

\begin{verbatim}
  IF (.NOT. interpolate) THEN
  ...
  ELSE
\end{verbatim}

Ignore: \textbf{\texttt{handle\_err}},
\textbf{\texttt{handle\_nc\_err}}.


%----------------------------------------------------------------------
\section{Additional tests (subroutine level)}

\begin{itemize}
  \item{ENTS jobs -- none of the existing test jobs use ENTS at all.
    I've added a test job derived from an ENTS job in the original
    \texttt{muffin} distribution which should cover at least some of
    the ENTS code.}
  \item{Debug and output -- there is quite a bit of diagnostic and
    data output code that can be exercised by enabling the following
    options: \texttt{debug\_loop}, \texttt{debug\_end},
    \texttt{ctrl\_audit}, \texttt{ctrl\_data\_save\_GLOBAL},
    \texttt{ctrl\_data\_save\_derived},
    \texttt{ctrl\_data\_save\_slice\_fairsea},
    \texttt{ctrl\_data\_save\_3d\_sig},
    \texttt{opt\_2d\_netcdf\_output},
    \texttt{ctrl\_data\_save\_ascii},
    \texttt{ctrl\_timeseries\_output}.  The most efficient way to do
    this is probably to enable all of these flags for a single test
    job -- I've done this for the \texttt{ocean-geochem-spin-up-2} job
    (the \texttt{ctrl\_timeseries\_output} and
    \texttt{ctrl\_data\_save\_GLOBAL} options are excluded from this
    because they slightly change the output results).}
  \item{Atmospheric methane -- a suitable job is from Section 5.3 of
    Andy's \texttt{muffin} examples document: ``Modern 36x36x16
    configuration + Fe \& CH4 cycles''.  The examples document has
    this running for 10,000 years, but that's way too long -- 100
    years should be enough (job name \texttt{cover/fe-atmos-ch4}).
    This job also includes ocean iron dynamics.}
  \item{An additional feature is the terrestrial carbon reservoir
    exchange introduced for doing inversions -- I've added a
    \texttt{cover/inversion} job to do this.}
  \item{I can't find any example jobs for ocean nitrate and ammonia
    cycling, although there are some configuration files that look
    reasonable.}
  \item{Ocean methane -- Section 6.1 in the examples document is
    ``Eocene $36 \times 36 \times 16$ configuration + $\mathrm{CH_4}$
    cycle'', which I've set up as \texttt{cover/eocene-ch4} and which
    should exercise some of the ocean methane cycle.}
  \item{Sedimenation -- I've added a job based on Section 2.6.1 in the
    examples document (part of ``Ridgwell and Hargreaves [2007]'')
    which has some sedimenation forcing
    (\texttt{cover/ridgwell-hargreaves-2007}).}
\end{itemize}


%======================================================================
\chapter{Step \#2: subroutine gap-filling}

%----------------------------------------------------------------------
\section{Test jobs}

At this point, the coverage test suite includes the following jobs:
\begin{itemize}
  \setlength\itemsep{0pt}
  \item{\texttt{biogem}}
  \item{\texttt{cao-et-al-2009}}
  \item{\texttt{ents}}
  \item{\texttt{eocene-ch4}}
  \item{\texttt{fe-atmos-ch4}}
  \item{\texttt{inversion}}
  \item{\texttt{make-restart}}
  \item{\texttt{ocean-atmos}}
  \item{\texttt{ocean-geochem-spin-up}}
  \item{\texttt{ocean-geochem-spin-up-2}}
  \item{\texttt{orbital-variations-abiotic-ocean}}
  \item{\texttt{restart-read}}
  \item{\texttt{ridgwell-hargreaves-2007}}
  \item{\texttt{ridgwell-schmidt-2010}}
\end{itemize}

The next step is to fill in any remaining subroutine-level gaps before
looking at smaller-scale results.

%----------------------------------------------------------------------
\section{Results (remaining uncalled subroutines)}

Uncalled subroutines:

\begin{tabular}{ll}
  \multicolumn{2}{l}{\textbf{\texttt{biogem/biogem\_box.f90}}} \\
  \texttt{sub\_calc\_bio\_uptake\_abio}    & NO CONFIGS \\
  \texttt{sub\_box\_oxidize\_NH4toNO3}     & \textbf{SEE \#3 BELOW} \\
  \texttt{sub\_box\_oxidize\_NH4toNO2}     & \textbf{SEE \#3 BELOW} \\
  \texttt{sub\_box\_scav\_H2S}             & NO CONFIGS \\
  \texttt{sub\_calc\_misc\_brinerejection} & NO CONFIGS \\

  \multicolumn{2}{l}{\textbf{\texttt{biogem/biogem\_data\_ascii.f90}}} \\
  \texttt{sub\_data\_save\_global\_snap} & NO CONFIGS \\

  \multicolumn{2}{l}{\textbf{\texttt{biogem/biogem\_data.f90}}} \\
  \texttt{sub\_init\_misc2D} & NO CONFIGS \\

  \multicolumn{2}{l}{\textbf{\texttt{biogem/biogem\_data\_netCDF.f90}}} \\
  \texttt{sub\_save\_netcdf\_flux\_seaair}    & NO CONFIGS \\
  \texttt{sub\_save\_netcdf\_ocn\_col\_extra} & \textbf{SEE \#4 BELOW} \\
  \texttt{sub\_save\_netcdf\_runtime}         & \textbf{SEE \#5 BELOW} \\

  \multicolumn{2}{l}{\textbf{\texttt{common/gem\_netcdf.f90}}} \\
  \texttt{sub\_defvar\_scalar} & IGNORE \\

  \multicolumn{2}{l}{\textbf{\texttt{common/gem\_util.f90}}} \\
  \texttt{sub\_load\_data\_ijk} & IGNORE \\

  \multicolumn{2}{l}{\textbf{\texttt{embm/embm\_diag.f90}}} \\
  \texttt{diaga}                     & \textbf{SEE \#6 BELOW} \\
  \texttt{aminmax}                   & \textbf{SEE \#6 BELOW} \\
  \texttt{diagend\_embm}             & \textbf{SEE \#6 BELOW} \\
  \texttt{diagfna}                   & \textbf{SEE \#6 BELOW} \\
  \texttt{diagosc\_embm}             & \textbf{SEE \#6 BELOW} \\
  \texttt{read\_embm\_target\_field} & \textbf{SEE \#6 BELOW} \\

  \multicolumn{2}{l}{\textbf{\texttt{ents/ents\_data.f90}}} \\
  \texttt{in\_ents\_ascii}  & IGNORE \\
  \texttt{in\_ents\_netcdf} & IGNORE \\

  \multicolumn{2}{l}{\textbf{\texttt{goldstein/goldstein.f90}}} \\
  \texttt{coshuffle}   & NO CONFIGS \\
  \texttt{ediff}       & NO CONFIGS \\
  \texttt{krausturner} & NO CONFIGS \\

  \multicolumn{2}{l}{\textbf{\texttt{goldsteinseaice/gold\_seaice\_data.f90}}} \\
  \texttt{diagsic}      & \textbf{SEE \#6 BELOW} \\
  \texttt{diagosc\_sic} & \textbf{SEE \#6 BELOW} \\

  \multicolumn{2}{l}{\textbf{\texttt{goldsteinseaice/gold\_seaice.f90}}} \\
  \texttt{diagend\_seaice} & \textbf{SEE \#6 BELOW} \\
  \texttt{tstipsic}        & NO CONFIGS \\

  \multicolumn{2}{l}{\textbf{\texttt{rokgem/rokgem\_box.f90}}} \\
  \texttt{sub\_GKWM}                & NO CONFIGS \\
  \texttt{sub\_GEM\_CO2}            & NO CONFIGS \\
  \texttt{sum\_calcium\_flux\_CaSi} & NO CONFIGS \\
  \texttt{sub\_2D\_weath}           & NO CONFIGS \\

  \multicolumn{2}{l}{\textbf{\texttt{rokgem/rokgem\_data.f90}}} \\
  \texttt{sub\_load\_rokgem\_restart} & IGNORE \\
  \texttt{sub\_data\_input\_3D}       & IGNORE \\
  \texttt{sub\_load\_weath}           & NO CONFIGS \\

  \multicolumn{2}{l}{\textbf{\texttt{rokgem/rokgem\_lib.f90}}} \\
  \texttt{define\_2D\_arrays} & NO CONFIGS \\

  \multicolumn{2}{l}{\textbf{\texttt{sedgem/sedgem\_box.f90}}} \\
  \texttt{sub\_update\_sed\_mud} & \textbf{SEE \#7 BELOW} \\
  \texttt{calc\_sed\_dis\_opal}  & \textbf{SEE \#1 BELOW} \\

  \multicolumn{2}{l}{\textbf{\texttt{sedgem/sedgem\_box\_ridgwelletal2003\_sedflx.f90}}} \\
  \texttt{init\_sedflx\_Si}      & \textbf{SEE \#1 BELOW} \\
  \texttt{runge\_kutta\_4\_opal} & \textbf{SEE \#1 BELOW} \\
\end{tabular}

\begin{tabular}{ll}
  \multicolumn{2}{l}{\textbf{\texttt{sedgem/sedgem\_data.f90}}} \\
  \texttt{sub\_load\_sed\_dis\_lookup\_opal} & \textbf{SEE \#1 BELOW} \\
  \texttt{sub\_load\_sed\_mix\_k}            & \textbf{SEE \#2 BELOW} \\
  \texttt{sub\_data\_output\_years}          & IGNORE \\
  \texttt{sub\_output\_year}                 & IGNORE \\
  \texttt{sub\_output\_counters}             & IGNORE
\end{tabular}

\subsubsection*{IDEA \#1 [DONE]}

Set up a job with base configuration
\texttt{cgenie.eb\_go\_gs\_ac\_bg\_sg\_rg\_gl.p0000e.BASESFeLiCa} and
user configuration \texttt{EXAMPLE.p0000e.PO4FeSi\_S36x36.SPIN}.

\textbf{BROKEN: the forcings for this aren't available and I've not
  figured out which forcings I have are compatible.  I'm going to
  ignore this one.}

\subsubsection*{IDEA \#2 [DONE]}

Need \texttt{sg\_ctrl\_sed\_bioturb=.TRUE.} and
\texttt{sg\_ctrl\_sed\_bioturb\_Archer=.FALSE.} (see \#7).

\subsubsection*{IDEA \#3 [DONE]}

Need selected ocean tracers: $\mathrm{O_2}$ (10), $\mathrm{NO_3}$ (6)
and $\mathrm{NH_4}$ (28), then $\mathrm{NO_2}$ (34) selects between
\texttt{sub\_box\_oxidize\_NH4toNO2} and
\texttt{sub\_box\_oxidize\_NH4toNO3}.

Added a new job, \texttt{nitrogen-no3}, using base configuration
\texttt{cgenie.eb\_go\_gs\_ac\_bg.worjh2.BASESFeN} and user
configuration \texttt{EXAMPLE.worjh2.PO4FeN.SPIN}.  Oxidation to
$\mathrm{NO_2}$ seems to require setting some additional parameters,
so I'm going to ignore it for now.

\subsubsection*{IDEA \#4 [DONE]}

Need a job with colour tracers enabled.  I've added a
\texttt{geoengineering} job with these switched on.

\subsubsection*{IDEA \#5 [DONE]}

Need to have \texttt{ctrl\_data\_save\_sig\_ascii} false for NetCDF
output.  \emph{I think I'll ignore this one.}

\subsubsection*{IDEA \#6 [DONE]}

EMBM and goldSIC have their own \texttt{debug\_loop} and
\texttt{debug\_end} flags.  \emph{These can be switched on for the
  \texttt{ocean-geochem-spin-up-2} job that has all the other debug
  stuff in it.}

\subsubsection*{IDEA \#7 [DONE]}

Need to set \texttt{par\_sed\_Dmax\_neritic} to something sensible to
trigger mud formation.  Quite a few configuration options for this.
I've added a \texttt{ridgwell-schmidt-2010-mud} that has a depth
setting that causes mud formation to be executed.  It also has
bioturbation set up to cover \#2.


%----------------------------------------------------------------------
\section{Additional jobs}

I've added the following new jobs:
\begin{itemize}
  \setlength\itemsep{0pt}
  \item{\texttt{geoengineering}}
  \item{\texttt{nitrogen-no3}}
  \item{\texttt{ridgwell-schmidt-2010-mud}}
\end{itemize}
and I've modified the \texttt{ocean-geochem-spin-up-2} job to include
more debugging code.


%======================================================================
\chapter{Step \#3: small-scale coverage analysis}

%----------------------------------------------------------------------
\section{Test jobs}

At this point, the coverage test suite includes the following jobs:
\begin{itemize}
  \setlength\itemsep{0pt}
  \item{\texttt{biogem}}
  \item{\texttt{cao-et-al-2009}}
  \item{\texttt{ents}}
  \item{\texttt{eocene-ch4}}
  \item{\texttt{fe-atmos-ch4}}
  \item{\texttt{geoengineering}}
  \item{\texttt{inversion}}
  \item{\texttt{make-restart}}
  \item{\texttt{nitrogen-no3}}
  \item{\texttt{ocean-atmos}}
  \item{\texttt{ocean-geochem-spin-up}}
  \item{\texttt{ocean-geochem-spin-up-2}}
  \item{\texttt{orbital-variations-abiotic-ocean}}
  \item{\texttt{restart-read}}
  \item{\texttt{ridgwell-hargreaves-2007}}
  \item{\texttt{ridgwell-schmidt-2010}}
  \item{\texttt{ridgwell-schmidt-2010-mud}}
\end{itemize}

%----------------------------------------------------------------------
\section{Results (remaining uncalled subroutines)}

Uncalled subroutines:

\begin{tabular}{ll}
  \multicolumn{2}{l}{\textbf{\texttt{biogem/biogem\_box.f90}}} \\
  \texttt{sub\_calc\_bio\_uptake\_abio}    & NO CONFIGS \\
  \texttt{sub\_box\_oxidize\_NH4toNO2}     & NO CONFIGS \\
  \texttt{sub\_box\_scav\_H2S}             & NO CONFIGS \\
  \texttt{sub\_calc\_misc\_brinerejection} & NO CONFIGS \\

  \multicolumn{2}{l}{\textbf{\texttt{biogem/biogem\_data\_ascii.f90}}} \\
  \texttt{sub\_data\_save\_global\_snap} & NO CONFIGS \\

  \multicolumn{2}{l}{\textbf{\texttt{biogem/biogem\_data.f90}}} \\
  \texttt{sub\_init\_misc2D}          & NO CONFIGS \\
  \texttt{sub\_init\_force\_solconst} & NO CONFIGS \\

  \multicolumn{2}{l}{\textbf{\texttt{biogem/biogem\_data\_netCDF.f90}}} \\
  \texttt{sub\_save\_netcdf\_flux\_seaair}    & NO CONFIGS \\
  \texttt{sub\_save\_netcdf\_ocn\_col\_extra} & ??? \\
  \texttt{sub\_save\_netcdf\_runtime}         & ??? \\

  \multicolumn{2}{l}{\textbf{\texttt{common/gem\_netcdf.f90}}} \\
  \texttt{sub\_defvar\_scalar} & IGNORE \\

  \multicolumn{2}{l}{\textbf{\texttt{common/gem\_util.f90}}} \\
  \texttt{sub\_load\_data\_ijk} & IGNORE \\

  \multicolumn{2}{l}{\textbf{\texttt{ents/ents\_data.f90}}} \\
  \texttt{in\_ents\_ascii}  & IGNORE \\
  \texttt{in\_ents\_netcdf} & IGNORE \\

  \multicolumn{2}{l}{\textbf{\texttt{goldstein/goldstein.f90}}} \\
  \texttt{coshuffle}   & NO CONFIGS \\
  \texttt{ediff}       & NO CONFIGS \\
  \texttt{krausturner} & NO CONFIGS \\

  \multicolumn{2}{l}{\textbf{\texttt{goldsteinseaice/gold\_seaice.f90}}} \\
  \texttt{tstipsic}        & NO CONFIGS \\

  \multicolumn{2}{l}{\textbf{\texttt{rokgem/rokgem\_box.f90}}} \\
  \texttt{sub\_GKWM}                & NO CONFIGS \\
  \texttt{sub\_GEM\_CO2}            & NO CONFIGS \\
  \texttt{sum\_calcium\_flux\_CaSi} & NO CONFIGS \\
  \texttt{sub\_2D\_weath}           & NO CONFIGS \\
\end{tabular}

\begin{tabular}{ll}
  \multicolumn{2}{l}{\textbf{\texttt{rokgem/rokgem\_data.f90}}} \\
  \texttt{sub\_load\_rokgem\_restart} & IGNORE \\
  \texttt{sub\_data\_input\_3D}       & IGNORE \\
  \texttt{sub\_load\_weath}           & NO CONFIGS \\

  \multicolumn{2}{l}{\textbf{\texttt{rokgem/rokgem\_lib.f90}}} \\
  \texttt{define\_2D\_arrays} & NO CONFIGS \\

  \multicolumn{2}{l}{\textbf{\texttt{sedgem/sedgem\_box.f90}}} \\
  \texttt{calc\_sed\_dis\_opal}  & IGNORE \\
\end{tabular}

\begin{tabular}{ll}
  \multicolumn{2}{l}{\textbf{\texttt{sedgem/sedgem\_box\_ridgwelletal2003\_sedflx.f90}}} \\
  \texttt{init\_sedflx\_Si}      & IGNORE \\
  \texttt{runge\_kutta\_4\_opal} & IGNORE \\

  \multicolumn{2}{l}{\textbf{\texttt{sedgem/sedgem\_data.f90}}} \\
  \texttt{sub\_load\_sed\_dis\_lookup\_opal} & IGNORE \\
  \texttt{sub\_data\_output\_years}          & IGNORE \\
  \texttt{sub\_output\_year}                 & IGNORE \\
  \texttt{sub\_output\_counters}             & IGNORE
\end{tabular}

%----------------------------------------------------------------------
\section{Results (unexecuted lines)}

``OK'' means trivial stuff (where ``trivial'' means sections of code
that, although they might be quite large, are unlikely to be affected
by the upcoming optimisation changes); ``NC'' means there are no
configurations available that will trigger this code; numbers refer to
the points below.

\begin{center}
\begin{tabular}{cll}
  1/597  & \texttt{sedgem/sedgem\_lib.f90}                           & OK(1) \\
  1/60   & \texttt{utils/util1.f90}                                  & OK \\
  2/103  & \texttt{wrappers/initialise\_genie.f90}                   & OK \\
  2/154  & \texttt{atchem/atchem\_box.f90}                           & OK(2) \\
  2/19   & \texttt{utils/open\_file\_nc.f90}                         & OK \\
  2/20   & \texttt{utils/close\_file\_nc.f90}                        & OK \\
  2/360  & \texttt{goldsteinseaice/gold\_seaice\_netcdf.f90}         & OK \\
  2/417  & \texttt{embm/embm\_netcdf.f90}                            & OK \\
  2/723  & \texttt{goldstein/goldstein\_netcdf.f90}                  & OK(3) \\
  3/438  & \texttt{wrappers/genie\_global.f90}                       & OK \\
  5/172  & \texttt{rokgem/rokgem.f90}                                & NC \\
  6/159  & \texttt{atchem/atchem.f90}                                & OK \\
  6/162  & \texttt{utils/writenc6.f90}                               & OK \\
  6/358  & \texttt{biogem/initialise\_biogem.f90}                    & OK/NC \\
  6/94   & \texttt{gemlite/gemlite\_data.f90}                        & OK \\
  7/678  & \texttt{ents/ents\_diag.f90}                              & OK \\
  8/200  & \texttt{ents/ents\_lib.f90}                               & OK \\
  9/45   & \texttt{utils/extras.f90}                                 & OK \\
 10/132  & \texttt{sedgem/end\_sedgem.f90}                           & OK/NC \\
 10/410  & \texttt{rokgem/rokgem\_lib.f90}                           & NC \\
 10/76   & \texttt{common/initialise\_gem.f90}                       & OK \\
 13/1047 & \texttt{sedgem/sedgem\_box\_archer1991\_sedflx.f90}       & OK? \\
 16/144  & \texttt{common/gem\_data.f90}                             & OK \\
 16/299  & \texttt{embm/embm\_data.f90}                              & OK \\
 16/65   & \texttt{utils/get1d\_data\_nc.f90}                        & OK \\
 20/546  & \texttt{gemlite/gemlite.f90}                              & OK(4) \\
 22/77   & \texttt{utils/get2d\_data\_nc.f90}                        & OK \\
 23/451  & \texttt{goldstein/goldstein\_data.f90}                    & OK(5) \\
 28/103  & \texttt{wrappers/genie\_util.f90}                         & OK \\
 28/89   & \texttt{utils/get3d\_data\_nc.f90}                        & OK \\
 29/1578 & \texttt{sedgem/sedgem\_data\_netCDF.f90}                  & OK(6) \\
 31/213  & \texttt{sedgem/initialise\_sedgem.f90}                    & OK(7) \\
 32/119  & \texttt{sedgem/sedgem\_box\_benthic.f90}                  & NC(8) \\
 36/290  & \texttt{atchem/atchem\_data.f90}                          & OK \\
 37/1652 & \texttt{biogem/biogem\_lib.f90}                           & OK \\
 37/604  & \texttt{genie.f90}                                        & NEW JOB (9) \\
\end{tabular}
\end{center}

\begin{center}
\begin{tabular}{cll}
 37/807  & \texttt{ents/ents.f90}                                    & OK(10) \\
 38/134  & \texttt{ents/ents\_data.f90}                              & OK \\
 43/279  & \texttt{rokgem/initialise\_rokgem.f90}                    & OK/NC \\
 50/659  & \texttt{ents/ents\_netcdf.f90}                            & OK \\
 51/450  & \texttt{goldsteinseaice/gold\_seaice\_data.f90}           & OK \\
 58/575  & \texttt{sedgem/sedgem.f90}                                & OK(11) \\
 59/260  & \texttt{wrappers/local\_netcdf.f90}                       & OK \\
 87/1156 & \texttt{common/gem\_netcdf.f90}                           & OK \\
119/479  & \texttt{sedgem/sedgem\_box\_ridgwelletal2003\_sedflx.f90} & NC(12) \\
124/1032 & \texttt{goldsteinseaice/gold\_seaice.f90}                 & OK/NC \\
125/1453 & \texttt{common/gem\_carbchem.f90}                         & NC(13) \\
189/832  & \texttt{rokgem/rokgem\_data.f90}                          & OK/NC \\
219/2700 & \texttt{biogem/biogem.f90}                                & NEW JOB (14) \\
226/799  & \texttt{embm/embm\_diag.f90}                              & OK \\
237/1152 & \texttt{goldstein/goldstein\_diag.f90}                    & OK(15) \\
237/2475 & \texttt{biogem/biogem\_data\_ascii.f90}                   & OK \\
255/1931 & \texttt{common/gem\_util.f90}                             & OK \\
289/2417 & \texttt{sedgem/sedgem\_data.f90}                          & OK(15) \\
305/2072 & \texttt{sedgem/sedgem\_box.f90}                           & NEW JOB (16) \\
343/3529 & \texttt{goldstein/goldstein.f90}                          & NEW JOB (17) \\
378/1742 & \texttt{rokgem/rokgem\_box.f90}                           & OK/NC \\
443/3598 & \texttt{embm/embm.f90}                                    & OK \\
456/3872 & \texttt{biogem/biogem\_box.f90}                           & OK(18) \\
584/2999 & \texttt{biogem/biogem\_data\_netCDF.f90}                  & OK \\
643/3151 & \texttt{biogem/biogem\_data.f90}                          & OK \\
\end{tabular}
\end{center}

\begin{enumerate}
  \item{Need \texttt{ctrl\_data\_save\_wtfrac} to be false.}
  \item{Need \texttt{atm\_select(ia\_pCH4\_14C)} and
    \texttt{atm\_select(ia\_pCO2\_14C)} to be set.}
  \item{GOLDSTEIN-specific \texttt{debug\_loop}.}
  \item{Various tracers: need to have GEMLITE enabled and
    \texttt{ocn\_select(io\_Ca)} false, \texttt{ocn\_select(io\_SO4)}
    false, \texttt{ocn\_select(io\_B)} true,
    \texttt{ocn\_select(io\_F)} true, \texttt{ocn\_select(io\_PO4)}
    false, \texttt{ocn\_select(io\_SiO2)} true,
    \texttt{ocn\_select(io\_H2S)} false, \texttt{ocn\_select(io\_NH4)}
    true.  \emph{Probably not worth worrying about...}}
  \item{GOLDSTEIN-specific \texttt{debug\_init}.}
  \item{Some stuff to do with saving cores, but looks OK.}
  \item{Needs \texttt{ctrl\_time\_series\_output},
    \texttt{ctrl\_continuing} and \texttt{ctrl\_append\_data} to be
    set.  \emph{All to do with SEDGEM restarts -- OK for now.}}
  \item{Needs \texttt{par\_sed\_diagen\_Corgopt} to be set to
    ``\texttt{arndtetal2013}'' -- no configs.}
  \item{\emph{Added a new job with \texttt{gem\_adapt\_auto} flag
      set.}}
  \item{Not sure about \texttt{snowswitch} here; also ENTS restarts;
    plus a couple of other things -- should all be OK though.}
  \item{Radioactive tracer; detrital age tracer; foram tracers; debug
    and time series output -- should all be OK.}
  \item{Need \texttt{par\_sed\_diagen\_opalopt} set to
    ``\texttt{ridgwelletal2003explicit}'' -- no configs.}
  \item{Possible values for \texttt{par\_carbconstset\_name}:
    ``\texttt{DicksonMillero}'', ``\texttt{Hansson}''
    ``\texttt{Roy}'' -- no configs.}
  \item{Some stuff to do with forcing of ocean tracers (e.g. Ca and
    $\mathrm{Ca^{44}}$); some stuff to do with sedimentation of opal
    and POC -- \emph{added a \texttt{calcium-isotopes} job}.}
  \item{Some dynamics changes, but all things that would cause
    breakage elsewhere in executed code if they were wrong.}
  \item{Possible values for \texttt{par\_sed\_diagen\_Corgopt}:
    ``\texttt{simple}'', ``\texttt{dunne2007}'',
    ``\texttt{arndtetal2013}'' -- \emph{added a
      \texttt{simple-corg-diagen} job with ``\texttt{simple} setting
      for this option (only config available)}.}
  \item{Hosing experiments: \emph{add hosing job from LABS}.}
  \item{Radioactive tracers; quite a few things to do with tracer
    indexing; some stuff to do with opal remineralisation -- probably
    all OK though since other things are likely to break along with
    these.}
\end{enumerate}

%======================================================================
\chapter{Final job list}

Based on the preceding investigations, the full coverage test suite
includes the following jobs:
\begin{itemize}
  \setlength\itemsep{0pt}
  \item{\texttt{biogem}}
  \item{\texttt{calcium-isotopes}}
  \item{\texttt{cao-et-al-2009}}
  \item{\texttt{ents}}
  \item{\texttt{eocene-ch4}}
  \item{\texttt{fe-atmos-ch4}}
  \item{\texttt{gem-adapt-auto}}
  \item{\texttt{geoengineering}}
  \item{\texttt{hosing}}
  \item{\texttt{inversion}}
  \item{\texttt{make-restart}}
  \item{\texttt{nitrogen-no3}}
  \item{\texttt{ocean-atmos}}
  \item{\texttt{ocean-geochem-spin-up}}
  \item{\texttt{ocean-geochem-spin-up-2}}
  \item{\texttt{orbital-variations-abiotic-ocean}}
  \item{\texttt{restart-read}}
  \item{\texttt{ridgwell-hargreaves-2007}}
  \item{\texttt{ridgwell-schmidt-2010}}
  \item{\texttt{ridgwell-schmidt-2010-mud}}
  \item{\texttt{simple-corg-diagen}}
\end{itemize}

Some of these are ``standard'' jobs, while others are ``artificial''
jobs constructed purely to exercise certain paths through the GENIE
code.

Running the full set of coverage tests takes:
\begin{center}
\begin{tabular}{ll}
  Test mode (optimised)       & 1h 28min \\
  Coverage mode (unoptimised) & 3h 00min
\end{tabular}
\end{center}

The following modules have changes in the number of unexecuted lines
compared to the previous test:
\begin{center}
\begin{tabular}{cl}
 33/604  & \texttt{genie.f90}                        \\
111/1032 & \texttt{goldsteinseaice/gold\_seaice.f90} \\
122/1435 & \texttt{common/gem\_carbchem.f90}         \\
217/2700 & \texttt{biogem/biogem.f90}                \\
233/1152 & \texttt{goldstein/goldstein\_diag.f90}    \\
236/2475 & \texttt{biogem/biogem\_data\_ascii.f90}   \\
274/2072 & \texttt{sedgem/sedgem\_box.f90}           \\
331/3529 & \texttt{goldstein/goldstein.f90}          \\
455/3872 & \texttt{biogem/biogem\_box.f90}           \\
582/2999 & \texttt{biogem/biogem\_data\_netCDF.f90}  \\
629/3151 & \texttt{biogem/biogem\_data.f90}          \\
\end{tabular}
\end{center}

\end{document}
