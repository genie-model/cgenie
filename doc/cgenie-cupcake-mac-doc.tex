%
% cgenie-cupcake-mac-doc.tex
%
% Developed by Gregory J. L. Tourte <g.j.l.tourte@bristol.ac.uk>
% Copyright (c) 2015 School of Geographical Sciences - The University of Bristol
% Licensed under the terms of GNU General Public License.
% All rights reserved.
%
% Changelog:
% 2015-04-15 - created
%
% arara: lualatex: {shell: true, draft: true}
%% arara: biber
%% arara: lualatex: {shell: true, draft: true}
% arara: lualatex: {shell: true}
%% arara: clean: {extensions: [aux, toc, bbl, run.xml, bcf, blg, out]}
%
\RequirePackage[l2tabu,orthodox]{nag}
\pdfpageattr {/Group << /S /Transparency /I true /CS /DeviceRGB>>}
\RequirePackage[2015/01/01]{latexrelease}

\documentclass{scrartcl}

\usepackage{scrhack}
\usepackage{scrpage2}
\usepackage{mathtools}
\AfterPackage!{hyperref}{\usepackage[nameinlink]{cleveref}}
\usepackage {fontspec,microtype}
\usepackage {typearea,geometry}
\usepackage {needspace}
\usepackage {graphicx}
\DeclareGraphicsExtensions{.pdf,.png,.jpg}
\usepackage {xcolor}
\usepackage {luatextra}
\usepackage {booktabs}
\usepackage {quoting}
\usepackage {tabu}
\PassOptionsToPackage{hyphens,obeyspaces,spaces}{url}
\usepackage {polyglossia}
\setdefaultlanguage[variant=british]{english}
\usepackage {minted}
\usepackage {siunitx,numprint}
\ExplSyntaxOn
\cs_new_eq:NN \siunitx_table_collect_begin:Nn \__siunitx_table_collect_begin:Nn
\ExplSyntaxOff
\usepackage {enumitem}
\setlist{nosep}
\usepackage [english]{selnolig}
\usepackage [british,calc]{datetime2}
\DTMlangsetup[en-GB]{ord=raise}
\usepackage [autostyle,english=british]{csquotes}
\usepackage {hyperxmp}
\usepackage {hyperref}
\usepackage [os=mac]{menukeys}



% Font definitions
\defaultfontfeatures{Ligatures = {TeX}, Scale = {MatchLowercase}}
%\setmainfont [Numbers = {Proportional,OldStyle}]{Linux Libertine O}
\setmainfont [Numbers = {Proportional,OldStyle}]{Minion Pro}
\setsansfont [Numbers = {Proportional,OldStyle}]{Linux Biolinum O}
\setmonofont[ItalicFont = *, ItalicFeatures = FakeSlant, RawFeature = -tlig;-trep, StylisticSet={1,3}]{Inconsolata zi4}
%\newfontfamily\liningroman[Numbers={Lining}]{Linux Libertine O}

\usepackage[math-style=TeX]{unicode-math}
\setmathfont{Tex Gyre Pagella Math}


\KOMAoptions{
    fontsize	= 11pt,
    numbers		= noendperiod,
    parskip		= half-,
%	twocolumn,
%	twoside,
	%draft,
	captions	= tableheading,
%	toc		= listofnumbered,
	listof		= leveldown,
%	chapterprefix	= true,
	DIV		= 9,
%	BCOR		= 7mm,
}
\recalctypearea
\pagestyle{scrheadings}

% Add accessibility functions with pdf (>v1.5)
\usepackage {accsupp}
% declare function to hide text from selection in pdf, used for the draft
% watermark and linenos in listing
\DeclareRobustCommand\squelch[1]{%
  \BeginAccSupp{method=plain,ActualText={}}#1\EndAccSupp{}%
}

%\BeforeBeginEnvironment{minted}{\begin{singlespace}}
%\AfterEndEnvironment{minted}{\end{singlespace}}
\renewcommand{\theFancyVerbLine}{
  \fontspec[Numbers = {Lining,Monospaced}]{Minion Pro}\color[rgb]{0.5,0.5,0.5}\scriptsize\squelch{\arabic{FancyVerbLine}}}

\usemintedstyle{autumn}

\setminted{%frame = lines,
%	framesep=2mm,
%	linenos=true,
	numbersep = 5pt,
	%gobble = 2,
%	fontsize = \footnotesize,
%	xleftmargin=20pt,
%	xrightmargin=20pt,
	tabsize=3,
	breaklines = true,
	breaksymbolleft = \squelch{\tiny\ensuremath{\hookrightarrow}},
}

\title{cGENIE (Cupcake)\\
Mac OS X Instructions}
\author{Gregory J. L. Tourte}

\pdfminorversion=7

%\DTMsetstyle{pdf}
\makeatletter
\hypersetup {
	hidelinks,
	breaklinks,
	linktocpage             = true,
	unicode                 = true,
	%hyperfootnotes          = true,
	bookmarksnumbered       = true,
	bookmarksopen           = true,
	pdfdisplaydoctitle      = true,
	%pdfpagelabels           = false,
	plainpages              = false,
	pdfauthor               = {\@author},
	pdftitle                = {\@title},
	pdfcontactemail		= {g.j.l.tourte@bristol.ac.uk},
	pdfcontactcity		= {Bristol},
	pdfcontactpostcode	= {BS8 1SS},
	pdfcontactcountry	= {United Kingdom},
	pdfcontacturl		= {http://www.bridge.bris.ac.uk/},
	pdfsubject              = {},
	pdfkeywords             = {},
	pdfcopyright		= {Copyright © 2015, Gregory J. L. Tourte},
	pdflicenseurl		= {http://creativecommons.org/licenses/by-nc-sa/4.0/},
	pdflang                 = en-GB,
	pdfencoding             = auto,
	pdfduplex               = DuplexFlipLongEdge,
	pdfprintscaling         = None,
	pdfinfo					= {
		CreationDate={D:20150323103000Z},
%		ModDate={\DTMnow},
	},
	%draft,
}
\makeatother

\begin{document}

\maketitle

These instructions have been tested on Mac OS X Yosemite (10.10) but should
work on Mavericks (10.9) as well. However we recommend that you use the latest
version of the OS available as we may not be able to reproduce errors and bugs
you may report.

The following instructions will require you to use the Mac OS X Terminal
application. This can be opened by using the \keys{\cmd+\SPACE} key combination
and searching for `\textit{Terminal}', alternatively it can be found in the
\directory{Macintosh HD/Applications/Utilities} directory. We suggest you pin
the application to your Dock for easy access if you haven't done so already.

\section{System Requirements}

To install and run cGENIE on Mac OS X you will need the following packages
installed:

\begin{itemize} \item Apple Xcode \item Xcode Command line Tools \item the
		MacPorts environment \item the HomeBrew packaging system \item gfortran
		compiler \item NetCDF libraries \end{itemize}

\subsection{Apple Xcode}

Apple Xcode can be downloaded for free from the Apple App Store. More details
can be found at \url{https://developer.apple.com/xcode/downloads/}. XCode
contains the GNU C Compiler (\texttt{gcc}) and most of the other libraries and
tools to allow the compilation of the model.

In addition to Xcode, you will need to install the Xcode Command Line Tools.
These used to be installed by default on older versions of Xcode but are now
distributed separately. Once Xcode is installed, you will need to run the
following command in the terminal:
%You can find instructions on installing this at
%\url{http://railsapps.github.io/xcode-command-line-tools.html}\,\footnote{Future
%versions of this document will reproduce these instructions in case the link
%disapear}.

\begin{minted}{console}
$ xcode-select --install
\end{minted} 

You can check if Xcode is properly installed using the command:

\mint{console}|$ xcode-select -p|
which should contain the following line in the response:
\begin{minted}{console}
/Applications/Xcode.app/Contents/Developer
\end{minted}

While we are checking the environment, you can check that \texttt{gcc} is installed properly:

\begin{minted}{console}
$ gcc --version
Configured with: --prefix=/Applications/Xcode.app/Contents/Developer/usr --with-gxx-include-dir=/usr/include/c++/4.2.1
Apple LLVM version 6.1.0 (clang-602.0.49) (based on LLVM 3.6.0svn)
Target: x86_64-apple-darwin14.3.0
Thread model: posix
\end{minted}

\subsection{Homebrew \& MacPorts}

Homebrew is a package manager for OS X which allows users to download and
install packages found in other UNIX style environment such as Linux and keep
them up to date in a managed way in the sense that one can update packages and
manage dependencies. MacPort has a similar aim but with a slightly different
philosophy. We will not compare these two here as it goes beyond the scope of
this document but we have used both to install software required to run cGENIE
on the Mac.

Homebrew and information about it can be found at \url{http://brew.sh/}. To
install the environment, simply type the following line in the Terminal:

\mint{console}|$ ruby -e "$(curl -fsSL https://raw.githubusercontent.com/Homebrew/install/master/install)"|

MacPort and information about it can be found at
\url{https://www.macports.org/} and instructions on how to install it are found
at \url{https://www.macports.org/install.php}. Following the installation, you
should make sure your installation is fully up to date by running:

\mint{console}|$ sudo port -v selfupdate|

\subsection{Fortran Compiler (\texttt{gfortran})}

We will get gfortran from homebrew. This is simply done by using the command:

\mint{console}|$ brew install gfortran|

it will download and install all the dependencies as well as the request
software itself. At the time of writing, the version of \texttt{gfortran}
available on the brew servers is 4.9.2.  

You can check that \texttt{gfortran} is properly installed by issuing the command:

\begin{minted}{console}
$ gfortran --version
GNU Fortran (Homebrew gcc 4.9.2_1) 4.9.2
Copyright (C) 2014 Free Software Foundation, Inc.

GNU Fortran comes with NO WARRANTY, to the extent permitted by law.
You may redistribute copies of GNU Fortran
under the terms of the GNU General Public License.
For more information about these matters, see the file named COPYING
\end{minted}

\subsection{NetCDF Libraries}

For NetCDF we will use the version available from MacPorts (we have tried using
homebrew for this in order to minimise the requirements and not mix the
packaging environment but unfortunately, the NetCDF brew package could not be
installed on our test machine. If this changes, we will reevaluate our
instructions.

Although previous versions of cGENIE required the C, C++ and fortran versions
of the library, this is no longer the case and the C++ layer is not a
requirement anymore. The following commands  will install the necessary NetCDF
libraries:

\begin{minted}{console}
$ sudo port install netcdf
$ sudo port install netcdf-fortran
\end{minted}

\section{Installing cGENIE (cupcake)}


\end{document}

% vim: ts=4
% vim600: fdl=0 fdm=marker fdc=3 spl=en_gb spell
