%---------------------------------------------------------------------------------------------------------------------------------
%
% cGENIE Examples document
%
% Andy Ridgwell, March 2011
%
%---------------------------------------------------------------------------------------------------------------------------------

\documentclass[10pt,twoside]{article}
\usepackage[paper=a4paper,portrait=true,hmargin=1.5cm,tmargin=2.0cm,bmargin=2.0cm,voffset=0pt,ignorehead,footnotesep=1cm]{geometry}
\usepackage{graphicx}
\usepackage{hyperref}
\usepackage{paralist}
\usepackage{caption}
\usepackage{float}
\usepackage{wasysym}

\linespread{1.1}
\setlength{\pltopsep}{2.5pt}
\setlength{\plparsep}{2.5pt}
\setlength{\partopsep}{2.5pt}
\setlength{\parskip}{5.0pt plus 1pt minus 1pt}

\title{Examples for cGENIE: 'muffin' pre-release version}
\author{Andy Ridgwell}
\date{\today}

\begin{document}

%=================================================================================================================================
%=== BEGIN DOCUMENT ==============================================================================================================
%=================================================================================================================================

\maketitle

%=================================================================================================================================
%=== CONTENTS ====================================================================================================================
%=================================================================================================================================

\tableofcontents

%=================================================================================================================================
%=== CHAPTERS ====================================================================================================================
%=================================================================================================================================

%---------------------------------------------------------------------------------------------------------------------------------
%--- Introduction -----------------------------------------------------------------------------------------------------------
%---------------------------------------------------------------------------------------------------------------------------------

\newpage
\section{Introduction}\label{Introduction}

This document is intended to provide practical help in configuring and running experiments using \texttt{cGENIE}.

The first section describes specific published configurations of \texttt{cGENIE} alongside a limited number of (also published) experiments. 
\\The second section describes spin-ups for a variety of general 'illustrative' configurations (not including published configurations).
\\The third section describes a variety of 'illustrative' exercises using these spin-ups.
\\This document should be read in conjunction with the \texttt{cGENIE} '\textit{HOW-TO}', which contains related and supporting information.
\\Disclaimers
\footnote{Unless otherwise stated, there is \textbf{no} feedback between CO2 and climate enabled in the experimental designs as provided, i.e., any simulated change in atmospheric CO2 will not affect climate. This can be changed by setting: \texttt{ea\_36=y}. Radiative forcing of the EMBM atmospheric module will then follow the relative (log) deviation (from 278 ppm) of CO2.}
\footnote{Unless otherwise stated, a 'CO2-calcification' feedback (see: \textit{Ridgwell et al.} [2007a,b] is enabled by default, with marine (pelagic) CaCO3 production is calculated based on ambient environmental (saturation) conditions. In order to fix the CaCO3:POC rain ratio (either spatially uniform, or with a (pre-calculated) spacial pattern -- see the \textit{HOW-TO}.}
\footnote{Only a selection of variables will be saved in the form of the spatial data fields (2- and 3-D) and data as \textit{time-series}. Not all the predicted results that you want might will therefore necessarily be saved. (Alternatively, rather more data than you care to look at might be saved (thus bloating the netCDF fields) ... ). Either way, you can adjust the data that is saved by changing existing or adding new parameter specifications in the \textit{user-config} file. Refer to the \textit{User manual} for more information.}
\footnote{Similarly -- data may not be saved with a sufficient frequency (or alternatively too frequently) or simply not at the required points in time.}
\footnote{Each experiment specifies a file containing a list of time-slice years (at which 2-D and 3-D fields will be saved) pointed to by the parameter \texttt{bg\_par\_infile\_slice\_name}. The file containing the time-series years (at which time-series data is saved) is pointed to by the parameter \texttt{bg\_par\_infile\_sig\_name}. To change the frequency and/or timing of data saving for time-slice or time-series data saving, either edit one of the files provided on SVN (some of which are used in the \textit{user-configs} provided), or create a new file and set the relevant parameter equal its name.}
apply ... \textbf{Refer to the \textit{User manual} for full details.:}

%---------------------------------------------------------------------------------------------------------------------------------

\subsection{Model configuration format}

Briefly: the \texttt{cGENIE} experimental design as described for each of the example experiments consists of 3 main components:

\begin{compactenum}

        \item A \textit{base-config} file, which contains the parameter values defining the basic grid dimensions (and continental configuration, the number of \textit{tracers}, and the basic physics configuration.
        This does not need to be edited.
        
        \item A \textit{user-config} file containing the parameter value changes for a specific experiment.
        This is edited.
        
        \item A set of files defining any selected forcings of biogeochemical tracers. The tracer \textit{forcing} definitions are provided in the form of subdirectories located in \texttt{\~{}/cgenie.muffin/genie-forcings}, with each containing:
        
        \begin{compactitem}
                \item 3 files defining which \textit{forcings} are to be selected, what sort of \textit{forcing} is required (i.e., \textit{restoring} vs. \textit{flux}), and any additional information, with a filename of the format:
                \\ (\texttt{configure\_forcings\_xxx.dat}).
                \item A series of forcing definition files (\texttt{biogem\_force\_*.dat}) for each biogeochemical forcing selected (if any).
        \end{compactitem}
        
        The names of the subdirectories correspond to the value of the \texttt{bg\_par\_fordir\_name} parameter set in the \textit{user-config} file. If the \texttt{bg\_par\_fordir\_name} parameter is not set in the \textit{user-config} file then the default setting is used (no tracer forcings are selected).
        
\end{compactenum}

        No \textit{forcing} need be specified, but the \textit{base-config} and \textit{user-config} are essential.
        
\noindent You can use the example experimental configurations provided as a template for your own experiments. To do this, just copy the \textit{user-config} file and rename it. Alter the parameter values contained in it and/or add additional parameter changes from the defaults to the end of the file. \textit{Forcing} directories can be similarly copied and edited, or they could be used unaltered for a variety of experiments.\footnote{The \textit{user-config} files provided may have to be edited consistent with your local software environment (particularly with how the home directory is defined/represented).}

\noindent The syntax for the command-line launching of a model experiment is as per detailed in the \textit{User manual}, i.e.:
\vspace{-5pt}\begin{verbatim}./runmuffin.sh cgenie.eb_go_gs_ac_bg.worbe2.BASE /
EXAMPLE.worbe2.Ridgwelletal1997.SPIN 10000
\end{verbatim}\vspace{-5pt}
this is all on ONE LINE (although in practice it may wrap on a normal screen width), and the components must be SPACE SEPERATED.
\\Again: \textbf{ONE LINE; SPACE SEPERATED}

Note that the default run-script is \texttt{runmuffin.sh}, but to be completely consistent with various older published experiments, a slightly different time-stepping is needed which can be enacted by using the modified run-script: \texttt{runmuffin.t100.sh}. For the specific cases where this alternative is recommended, is noted in the experiment descriptions.

%---------------------------------------------------------------------------------------------------------------------------------
%--- Published cGENIE configurations ---------------------------------------------------------------------------
%---------------------------------------------------------------------------------------------------------------------------------

\newpage
\section{Published configurations (\textit{spin-ups}) and experiments}\label{published}

%---------------------------------------------------------------------------------------------------------------------------------
%---------------------------------------------------------------------------------------------------------------------------------

\subsection{\textit{Cao et al. }[2010]}

These are the configurations, of (preindustrial) spin-up and historical transient as described by \textit{Cao et al. }[2010] (16-level ocean).

%---------------------------------------------------------------------------------------------------------------------------------

\subsubsection{preindustrial \textit{spin-up}}\label{EXAMPLE.worjh2.Caoetal2009.SPIN}

This is a pre-industrial \textit{spin-up} as described by \textit{Cao et al. }[2010].

\noindent \textbf{Physics configuration}: 16-level GOLDSTEIN ocean + sea-ice + EMBM atmosphere modules. Climatology is seasonal and identical to that described in \textit{Cao et al.} [2009] (and references therein).

\noindent \textbf{Biogeochemistry configuration}: Basic ocean (and atmosphere) carbon cycle as described \textit{Cao et al.} [2009], plus radiocarbon and CFC tracers. Atmospheric restoring of CO2 (plus d13C and d14C) and CFCs to preindustrial values.

\noindent \textbf{Base-config} The \textit{base-config} file is named:
\vspace{-10pt}\begin{verbatim}cgenie.eb_go_gs_ac_bg.worjh2.ANTH\end{verbatim}\vspace{-10pt}
and defines the use (and initial values) of the following diagnostic tracers (in addition to the basic carbon cycle (e.g. of \textit{Ridgwell et al.} [2007]):

\begin{compactenum}

        \item Atmospheric (gaseous) tracers (\texttt{gm\_atm\_select\_xx}):
        \\\texttt{ia\_pCO2\_14C} (xx=5), \texttt{ia\_pCFC11} (xx=18), \texttt{ia\_pCFC12} (xx=19)
        \item Ocean (dissolved) tracers (\texttt{gm\_ocn\_select\_xx}):
        \\\texttt{io\_DIC\_14C} (xx=5), \texttt{io\_DOM\_C\_14C} (xx=17), \texttt{io\_CFC11} (xx=45), \texttt{io\_CFC12} (xx=46)
        
\end{compactenum}

\noindent By default, a zero concentration for CFCs (in ocean and atmosphere) is set, while the d14C isotopic composition of all carbon species is set to 0 per mil.

\noindent \textbf{User-config} The \textit{user-config} file is named:
\vspace{-10pt}\begin{verbatim}EXAMPLE.worjh2.Caoetal2009.SPIN\end{verbatim}\vspace{-10pt}
and differs the equivalent basic modern configuration of \textit{Ridgwell et al.} [2007]:

\begin{compactitem}
        
        \item \texttt{--- BIOLOGICAL NEW PRODUCTION ---}
        \\ Various [parameter values appropriate to 16-level ocean model calibration].
        \item \texttt{--- INORGANIC MATTER EXPORT RATIOS ---}
        \\ Various [parameter values appropriate to 16-level ocean model calibration].
        \item \texttt{--- REMINERALIZATION ---}
        \\ Various [parameter values appropriate to 16-level ocean model calibration].
        \item \texttt{--- FORCINGS ---}
        \\ The \textit{forcing} prescribes fixed boundary conditions of atmospheric pCO2 and plus its isotopes and CFCs.
        The parameter values that follow simply scale atmospheric composition:
        \vspace{-5pt}\begin{verbatim}
        bg_par_forcing_name="pyyyyz.preindustrial"
        \end{verbatim}\vspace{-5pt}

\end{compactitem}

\noindent \textbf{Pre-requisites:} NONE.

\noindent \textbf{Execution}: A command-line launching\footnote{Note here that the alternative (older papers) time-stepping is used and hence the run script is \texttt{runmuffin.t100.sh} rather than \texttt{runmuffin.sh}.} of the model experiment (10000 years integration) would be:
\vspace{-10pt}\begin{verbatim}./runmuffin.t100.sh cgenie.eb_go_gs_ac_bg.worjh2.ANTH /
EXAMPLE.worjh2.Caoetal2009.SPIN 10000\end{verbatim}\vspace{-5pt}

\noindent \textbf{Ideas for further development:} 

\noindent \textbf{Relevant HOW-TO}:

%---------------------------------------------------------------------------------------------------------------------------------

\subsubsection{Historical transient}\label{EXAMPLE.worjh2.Caoetal2009.historical}

BLAH

%---------------------------------------------------------------------------------------------------------------------------------
%---------------------------------------------------------------------------------------------------------------------------------

\subsection{\textit{Colbourn et al.} [2013]}

\subsubsection{CO2 pulse emissions transient}\label{EXAMPLE.worbe2.Colbournetal2013.EMISSIONS}

This is a configuration very similar to, but not necessarily *identical* to one of those described by \textit{Colbourn et al.} [2013]. It should be used following the 2-step \textit{spin-up} process described in \textit{Ridgwell and Hargreaves} [2007] (for which EXAMPLES are given above).

\noindent \textbf{Physics configuration:} GOLDSTEIN ocean + sea-ice + EMBM atmosphere modules with no seasonal insolation forcing. Described in \textit{Ridgwell et al.} [2007].

\noindent \textbf{Biogeochemistry configuration:} Basic ocean (and atmosphere) carbon cycle as described \textit{Ridgwell et al.} [2007]. No atmospheric restoring of \textit{p}CO2 (+ $\delta^{13}$C) and instead the global carbon cycle is left to determine its own fate in response to imposed CO$_{2}$ emissions and according to weathering feedback vs. volcanic CO$_{2}$ out-gassing.

\noindent \textbf{Base-config:} The \textit{base-config} file is:
\vspace{-10pt}\begin{verbatim}cgenie.eb_go_gs_ac_bg_sg_rg.worbe2.BASE.config\end{verbatim}\vspace{-10pt}
and specifies sediment (SEDGEM) and weathering (ROKGEM) modules.

\noindent \textbf{User-config:} 
\\ The associated \textit{user-config}:
\vspace{-10pt}\begin{verbatim}EXAMPLE.worbe2.Colbournetal2013.EMISSIONS\end{verbatim}\vspace{-10pt}
has the following noteworthy features compared to the 2nd stage open system \textit{spin-up}
\\ \texttt{EXAMPLE.worbe2.RidgwellHargreaves1997\_S36x36.SPIN2}:

\begin{compactitem}
                \item \texttt{--- WEATHERING ---}
                \\ As before, an \textit{open  system} is specified:
\vspace{-5pt}\begin{verbatim}
bg_ctrl_force_sed_closedsystem=.false.
                \end{verbatim}\vspace{-5pt}
                Now however, temperature-only dependence plus baseline fluxes of silicate and carbonate weathering is specified in the bulk of the lines in this section (plus volcanic CO$_{2}$ out-gassing and isotopic compositions):
\vspace{-5pt}\begin{verbatim}
rg_opt_weather_T_Ca=.true.
rg_opt_weather_T_Si=.true.
rg_par_ref_T0=8.48
rg_par_weather_CaCO3=5.59E+12
rg_par_weather_CaSiO3=5.59E+12 
rg_par_outgas_CO2=5.59E+12
rg_par_outgas_CO2_d13C=-6.0
rg_par_weather_CaCO3_d13C=12.8
                \end{verbatim}\vspace{-5pt}
The first two options specify a temperature-dependency for the weathering of carbonate and silicate rocks respectively. The reference temperature parameter, \texttt{rg\_par\_ref\_T0} is the mean global land surface temperature that any deviation from will induce a change in weathering fluxes compared to the reference value specified (the parameters following). The mean global land surface temperature is given in the \textbf{BIOGEM} \textit{time-series} file:
\\\texttt{biogem\_series\_misc\_SLT.res}
\\which in the \textit{spin-up} is 8.48$^{\circ}$C in this example.
\\The total weathering flux (before temperature modification) is taken from the total CaCO$_{3}$ burial flux calculated by \texttt{EXAMPLE.worbe2.RidgwellHargreaves1997\_S36x36.SPIN2} (the \textit{spin-up} for this experiment) -- 11.19 Tmol Ca yr$^{-1}$. However, this is now split between silicate and carbonate weathering, which in this example is assumed to be in a simple proportion (in terms of Ca$^{2+}$ flux to the ocean, i.e. giving values of 5.59 Tmol Ca yr$^{-1}$ from each of silicate\footnote{This choice is a little more complicated if a seasonal cycle is selected -- e.g. refer to the Permian configuration or \textit{HOW-TO} document.} and carbonate weathering.To balance the silicate weathering, volcanic CO$_{2}$ out-gassing is then also assigned a value of 5.59 Tmol CO$_{2}$ yr$^{-1}$.
\\The carbon isotopic signature of volcanic CO$_{2}$ out-gassing is assigned a value of -6.0\permil. The $\delta$$^{13}$C of weathered CaCO$_{3}$ is then simply set in order that inputs equal the mean $\delta$$^{13}$C of carbonate burial, which as given in the \textbf{BIOGEM} \textit{time-series} file:
\\\texttt{biogem\_series\_sed\_CaCO3\_13C.res}
\\is 3.40\permil. The resulting $\delta$$^{13}$C of weathered CaCO$_{3}$ comes out to be 12.8\permil.
                \item \texttt{--- FORCINGS ---}
Finally, a pulse of CO$_{2}$ to the atmosphere is specified:
\vspace{-5pt}\begin{verbatim}
bg_par_forcing_name="worbe2.FpCO2_Fp13CO2.detplusopalSED"
bg_par_atm_force_scale_val_3=5000.0
bg_par_atm_force_scale_val_4=-27.0
bg_par_atm_force_scale_time_3=1.0
bg_par_atm_force_scale_time_4=1.0
                \end{verbatim}\vspace{-5pt}
The forcing is generic and corresponds to a 1 PgC (8.3333e+013 mol C) release over 1 year, which is then scaled to 5000 PgC by parameter \texttt{bg\_par\_atm\_force\_scale\_val\_3}. An isotopic signature appropriate for mean fossil fuel carbon is also set (\texttt{bg\_par\_atm\_force\_scale\_val\_4}) (scaled from a default value of 1.0). The default 1 yr duration is kept unchanged.
\end{compactitem}

\noindent Imbalance between volcanic CO$_{2}$ out-gassing and silicate weathering -- i.e. the tendency for the atmosphere to lose or gain CO$_{2}$, is provided in the \textbf{BIOGEM} \textit{time-series} file:
\\\texttt{biogem\_series\_misc\_exweather\_Ca.res}
\\which reports the absolute imbalance between silicate weathering and CO$_{2}$ out-gassing (in mol yr$^{-1}$) and also this as a percentage.

\noindent \textbf{Pre-requisites:} \texttt{EXAMPLE.worbe2.RidgwellHargreaves1997\_S36x36.SPIN2}

\noindent \textbf{Execution}: Command-line launching of the model experiment for a 200000\footnote{Note that a longer run-time for the \textit{spin-up} may be necessary to have *completely* equilibrated silicate weathering vs. CO$_{2}$ out-gassing and carbon isotopes, if the system is not carefully balanced to start with. An example employing an acceleration technique for the long-term mass balance is given: \\ \texttt{EXAMPLE.worbe2.RidgwellHargreaves1997\_S36x36.SPIN2gl}.} year experiment:
\vspace{-10pt}\small\begin{verbatim}
./runmuffin.t100.sh cgenie.eb_go_gs_ac_bg_sg_rg.worbe2.BASE / 
EXAMPLE.worbe2.Colbournetal2013.EMISSIONS 200000 
EXAMPLE.worbe2.RidgwellHargreaves1997_S36x36.SPIN2
\end{verbatim}\normalsize\vspace{-10pt}

\noindent \textbf{Ideas for further development:} An additional example: \texttt{EXAMPLE.worbe2.Colbournetal2013.CTRL} is provided. This differs only in the scaling flux of CO$_{2}$ emitted (\texttt{bg\_par\_atm\_force\_scale\_val\_3}) being set to zero. It provides an essential check on the existence of any drift remaining following the \textit{spin-up}, particularly for extended long-term experiments such as this.
\\One could also change the magnitude of the forcing (5000 PgC as set here), and/or the duration, perhaps to answer the question(s): how does the magnitude and/or time-scale or carbon release affect the strength of silicate weathering feedback (if at all). 

\noindent \textbf{Relevant HOW-TOs:} 'Accelerate the weathering-sedimentation mass balance'.

%---------------------------------------------------------------------------------------------------------------------------------
%---------------------------------------------------------------------------------------------------------------------------------

\subsection{\textit{Monteiro et al.} [2012]}

SUMMARY BLAH

%---------------------------------------------------------------------------------------------------------------------------------

\subsubsection{pre OAE2}\label{EXAMPLE.p0093k.Monteiroetal2012 PREOAE2.SPIN}

BLAH

%---------------------------------------------------------------------------------------------------------------------------------

\subsubsection{syn OAE2}\label{EXAMPLE.p0093k.Monteiroetal2012 OAE2.SPIN}

BLAH

%---------------------------------------------------------------------------------------------------------------------------------
%---------------------------------------------------------------------------------------------------------------------------------

\subsection{\textit{Ridgwell} [2007]}

BLAH

%---------------------------------------------------------------------------------------------------------------------------------
%---------------------------------------------------------------------------------------------------------------------------------

\subsection{\textit{Ridgwell et al.} [2007]}

BLAH

%---------------------------------------------------------------------------------------------------------------------------------
%---------------------------------------------------------------------------------------------------------------------------------

\subsection{\textit{Ridgwell and Hargreaves} [2007]}

The model configuration used by \textit{Ridgwell and Hargreaves} [2007] is based on a low vertical resolution modern continental configuration, with a basic (PO$_{4}$-only nutrient) based ocean carbon cycle together with carbonate (CaCO$_{3}$) sedimentation and burial and weathering input.
\\ There are two main stages required in the \textit{spin-up}, as described in \textit{Ridgwell and Hargreaves} [2007], with a (third) alternative 2nd stage outlined here that enables an accelerated \textit{spin-up} of long-lived isotope tracers such as (+ $\delta^{13}$C):

        \begin{compactenum}
        
                \item The first stage of \textit{spin-up} creates an equilibrium climate and ocean circulation consistent with the prescribed value of atmospheric \textit{p}CO$_{2}$ and a given (initial, observed) value of ocean alkalinity. It is also used to diagnose the equilibrium distribution of surface sediment composition and hence the global rate of (pelagic) CaCO$_{3}$ burial. Bioturbation (mixing of the sediments) is turned 'off' to facilitate a shorter time to equilibrium of surface sediment composition and CaCO$_{3}$ burial rates.
                
                \item The second stage of \textit{spin-up} is used to equilibrate the sediments with bioturbational mixing enabled and to a depth in the sediment column sufficient to support the dissolution ('chemical erosion') of previously preserved CaCO$_{3}$ in any subsequent perturbation e.g. CO2 release experiment. This 2nd stage is also where the system is 'opened' and weathering inputs and sediment loses brought into balance. 
                
                \item The third \textit{spin-up} is given as an alternative to the second and uses an 'accelerated' configuration of the model to balance carbon and alkalinity and more importantly, long-term isotopic tracers such as (+ $\delta^{13}$C). It also introduces interactive weathering (weathering feedback), also then bringing climate, silicate weathering, and volcanic out-gassing all into steady state with respect to sediment burial. A simpler configuration including weathering feedback but not acceleration of the long-term mass balance is provided as \texttt{EXAMPLE.worbe2.Colbournetal2013}.
                
        \end{compactenum}

%---------------------------------------------------------------------------------------------------------------------------------

\subsubsection{1st stage (CLOSED CaCO3 cycle) \textit{spin-up}}\label{EXAMPLE.worbe2.RidgwellHargreaves1997_S36x36.SPIN1}

This is the first stage of a 2-part \textsl{spin-up} of modern marine CaCO$_{3}$ cycling as described in \textit{Ridgwell and Hargreaves} [2007].

\noindent \textbf{Physics configuration:} GOLDSTEIN ocean + sea-ice + EMBM atmosphere modules with no seasonal insolation forcing as described in \textit{Ridgwell et al.} [2007].

\noindent \textbf{Biogeochemistry configuration:} Basic ocean (and atmosphere) carbon cycle as described \textit{Ridgwell et al.} [2007] but with re-calibrated biogeochemical parameter values as described in \textit{Ridgwell and Hargreaves} [2007]. Atmospheric restoring of \textit{p}CO2 (+ $\delta^{13}$C). Carbonate (only) weathering forced equal to projected CaCO$_{3}$ (pelagic) burial in a forced 'closed system' balance.

\noindent \textbf{Base-config:} The \textit{base-config} file\footnote{Remembering to omit the '\texttt{.config}' when specifying its name.} is:
\vspace{-10pt}\begin{verbatim}cgenie.eb_go_gs_ac_bg_sg_rg.worbe2.BASE.config\end{verbatim}\vspace{-10pt}
and specifies sediment (SEDGEM) and weathering (ROKGEM) modules in addition to the basic ocean-only carbon cycle configuration (i.e. \texttt{cgenie.eb\_go\_gs\_ac\_bg.worbe2.BASE.config} as per \textit{Ridgwell et al.} [2007]).

\noindent \textbf{User-config:} 
\\ The associated \textit{user-config}:
\vspace{-10pt}\begin{verbatim}EXAMPLE.worbe2.RidgwellHargreaves1997_S36x36.SPIN1\end{verbatim}\vspace{-10pt}
has the following noteworthy features (and differences compared to e.g. the ocean-only configuration):
\begin{compactitem}
                \item \texttt{--- SEDIMENTS ---}
                \\ A 36x36 resolution sediment topography is assumed and a mask defining (no) reefal grid points specified.
                \\ A series of sediment cores are requested to be created (file: \texttt{sedcore.nc}), defined by the sediment 'save' mask:
\vspace{-5pt}\begin{verbatim}
sg_par_sedcore_save_mask_name='worbe2_save_mask.36x36'
                \end{verbatim}\vspace{-5pt}
Note that 2D, ocean-sediment data will also still be saved, as will a restart (file: \texttt{\_restart.nc}), which by default is in netCDF format and includes the first 50 sediment layers at all sediment grid points (so partly duplicating both the 2D and \textit{sedcore} output).
                \\ The sedimentary diagenesis (dissolution) option for CaCO$_{3}$ is as per summarized in \textit{Ridgwell and Hargreaves} [2007] and \textit{Ridgwell et al.} [2003], and is described in full in Ridgwell [2001]:
\vspace{-5pt}\begin{verbatim}
sg_par_sed_diagen_CaCO3opt='ridgwell2001lookup'
                \end{verbatim}\vspace{-5pt}
                In addition, no bioturbational mixing is applied to the sediment layers in order to achieve a faster equilibrium distribution of surface sediment composition (and global CaCO$_{3}$ burial rate):
                \vspace{-5pt}\begin{verbatim}
sg_ctrl_sed_bioturb=.false.
                \end{verbatim}\vspace{-5pt}
No detrital (non carbonate) flux is specified as an explicit field is provided as part of the \textit{forcing} (see below).
                \item \texttt{--- WEATHERING ---}
                \\ A \textit{closed system} is specified for this initial \textsl{spin-up}:
\vspace{-5pt}\begin{verbatim}
bg_ctrl_force_sed_closedsystem=.true.
                \end{verbatim}\vspace{-5pt}
Because the system is being run as a \textit{closed system}, the weathering flux settings are pretty irrelevant because weathering fluxes are internally automatically and continuously re-scaled in order to exactly balance the burial loss in marine sediments. For completeness, a basic default\footnote{Note that a non-zero value must be specified because the given value is continuously scaled during the run to always exactly balance the sedimentary sink.} is given:
\vspace{-5pt}\begin{verbatim}
rg_par_weather_CaCO3=10.00E+12
rg_par_weather_CaCO3_13C=3.0
                \end{verbatim}\vspace{-5pt}
                \item \texttt{--- DATA SAVING ---}
                \\ A fairly standard level of output is set:
\vspace{-5pt}\begin{verbatim}
bg_par_data_save_level=4
                \end{verbatim}\vspace{-5pt}
                \item \texttt{--- FORCINGS ---}
                \\ Atmospheric composition is continuously restored to a prescribed (pre-Industrial) composition:
\vspace{-5pt}\begin{verbatim}
bg_par_forcing_name="worbe2.RpCO2_Rp13CO2_detplusopalSED"
bg_par_atm_force_scale_val_3=278.0E-06
bg_par_atm_force_scale_val_4=-6.5
                \end{verbatim}\vspace{-5pt}
and which also includes a spatially-explicit non-carbonate dilution flux to the sediments (see \textit{Ridgwell and Hargreaves} [2007]).
        \end{compactitem}
        
\noindent \textbf{Pre-requisites:} None.

\noindent \textbf{Execution}: Command-line launching\footnote{Note here that the alternative (older papers) time-stepping is used and hence the run script is \texttt{runmuffin.t100.sh} rather than \texttt{runmuffin.sh}.} of the model experiment for a 20000 year \textit{spin-up} from 'cold':
\vspace{-10pt}\small\begin{verbatim}./runmuffin.t100.sh cgenie.eb_go_gs_ac_bg_sg_rg.worbe2.BASE / 
EXAMPLE.worbe2.RidgwellHargreaves1997_S36x36.SPIN1 20000 \end{verbatim}\normalsize\vspace{-10pt}

\noindent \textbf{Ideas for further development:} 

\noindent \textbf{Relevant HOW-TOs}: 'Spin-up the full marine carbon cycle including sediments'.

%---------------------------------------------------------------------------------------------------------------------------------

\subsubsection{2nd stage (OPEN CaCO3 cycle) \textit{spin-up}}\label{EXAMPLE.worbe2.RidgwellHargreaves1997_S36x36.SPIN2}

This is the second stage \textsl{spin-up} of the \textit{Ridgwell and Hargreaves} [2007] configuration of modern marine CaCO$_{3}$ cycling.

\noindent \textbf{Physics configuration:} GOLDSTEIN ocean + sea-ice + EMBM atmosphere modules with no seasonal insolation forcing. Described in \textit{Ridgwell et al.} [2007].

\noindent \textbf{Biogeochemistry configuration:} Basic ocean (and atmosphere) carbon cycle as described \textit{Ridgwell et al.} [2007]. Atmospheric restoring of \textit{p}CO$_{2}$ (+ $\delta^{13}$C). Carbonate (only) weathering adjusted and set equal to the global CaCO$_{3}$ (pelagic) burial flux as diagnosed form the 1st stage \textit{spin-up} 
\\ (\texttt{EXAMPLE.worbe2.RidgwellHargreaves1997\_S36x36.SPIN}),
\\ in an 'open system' balance as per \textit{Ridgwell and Hargreaves} [2007].

\noindent \textbf{Base-config:} The \textit{base-config} file is as per the 1st stage \textit{spin-up}:
\vspace{-10pt}\begin{verbatim}cgenie.eb_go_gs_ac_bg_sg_rg.worbe2.BASE.config\end{verbatim}\vspace{-10pt}
and specifies sediment (SEDGEM) and weathering (ROKGEM) modules in addition to the basic ocean-only carbon cycle configuration (e.g. \texttt{cgenie.eb\_go\_gs\_ac\_bg.worbe2.BASE.config} as per \textit{Ridgwell et al.} [2007].

\noindent \textbf{User-config:} 
\\ The associated \textit{user-config}:
\vspace{-10pt}\begin{verbatim}EXAMPLE.worbe2.RidgwellHargreaves1997_S36x36.SPIN2\end{verbatim}\vspace{-10pt}
has the following noteworthy differences compared to the 1st stage \textit{spin-up}:

\begin{compactitem}
                \item \texttt{--- SEDIMENTS ---}
Bioturbational mixing is now applied to the sediment layers:
                \vspace{-5pt}\begin{verbatim}
sg_ctrl_sed_bioturb=.true.
                \end{verbatim}\vspace{-5pt}
No detrital (non carbonate) flux is specified as an explicit field is provided as part of the \textit{forcing} (see below).
                \item \texttt{--- WEATHERING ---}
                \\ An \textit{open  system} is now specified:
\vspace{-5pt}\begin{verbatim}
bg_ctrl_force_sed_closedsystem=.false.
                \end{verbatim}\vspace{-5pt}
The prescribed weathering flux (\texttt{rg\_par\_weather\_CaCO3}) is now revised and set equal to the diagnosed global CaCO3 burial rate ('\texttt{Total CaCO3 pres (sediment grid)}') as reported in the SEDGEM module results file (\texttt{seddiag\_misc\_DATA\_GLOBAL.res}), giving:
\vspace{-5pt}\begin{verbatim}
rg_par_weather_CaCO3=0.1119E+14
                \end{verbatim}\vspace{-5pt}
                In addition -- balance in the d13C system can be approximately achieved by setting the $\delta^{13}$C of weathered carbonates to the mean $\delta^{13}$C of buried pelagic carbonates (which is given in the BIOGEM time-series file: \texttt{biogem\_series\_sed\_CaCO3\_13C.res}), e.g.:
                \vspace{-5pt}\begin{verbatim}
rg_par_weather_CaCO3_d13C=3.409
                \end{verbatim}\vspace{-5pt}
        \end{compactitem}

\noindent \textbf{Pre-requisites:} \texttt{EXAMPLE.worbe2.RidgwellHargreaves1997\_S36x36.SPIN1}

\noindent \textbf{Execution}: Command-line launching of the model experiment for a 50000 year spin up:
\vspace{-10pt}\small\begin{verbatim}
./runmuffin.t100.sh cgenie.eb_go_gs_ac_bg_sg_rg.worbe2.BASE / 
EXAMPLE.worbe2.RidgwellHargreaves1997_S36x36.SPIN2 50000 
EXAMPLE.worbe2.RidgwellHargreaves1997_S36x36.SPIN1
\end{verbatim}\normalsize\vspace{-10pt}

\noindent \textbf{Ideas for further development:} 

\noindent \textbf{Relevant HOW-TOs}: 'Spin-up the full marine carbon cycle including sediments'.

%---------------------------------------------------------------------------------------------------------------------------------

\subsubsection{Alternative 2nd stage (OPEN CaCO3 cycle) \textit{spin-up}}\label{EXAMPLE.worbe2.RidgwellHargreaves1997_S36x36.SPIN2gl}

This is the alternative 2nd stage of modern marine CaCO$_{3}$ cycling and it not part of the publication of \textit{Ridgwell and Hargreaves} [2007].

\noindent \textbf{Physics configuration:} GOLDSTEIN ocean + sea-ice + EMBM atmosphere modules with no seasonal insolation forcing. Described in \textit{Ridgwell et al.} [2007].

\noindent \textbf{Biogeochemistry configuration:} Basic ocean (and atmosphere) carbon cycle as described \textit{Ridgwell et al.} [2007]. No atmospheric restoring of \textit{p}CO2 (+ $\delta^{13}$C) and instead the global carbon cycle is left to determine its own state according to weathering feedback vs. volcanic CO$_{2}$ out-gassing.

\noindent \textbf{Base-config:} The \textit{base-config} file is:
\vspace{-10pt}\begin{verbatim}cgenie.eb_go_gs_ac_bg_sg_rg_gl.worbe2.BASE.config\end{verbatim}\vspace{-10pt}
and specifies sediment (SEDGEM) and weathering (ROKGEM) modules in addition to the basic ocean-only carbon cycle configuration (e.g. \texttt{cgenie.eb\_go\_gs\_ac\_bg.worbe2.BASE.config} as per \textit{Ridgwell et al.} [2007] and now plus a module for accelerating long-term mass weathering vs. burial balance ('\texttt{GEMlite}').

\noindent \textbf{User-config:} 
\\ The associated \textit{user-config}:
\vspace{-10pt}\begin{verbatim}EXAMPLE.worbe2.RidgwellHargreaves1997_S36x36.SPIN2gl\end{verbatim}\vspace{-10pt}
has the following noteworthy features compared to the 1st stage \textit{spin-up}:

\begin{compactitem}
                \item \texttt{--- SEDIMENTS ---}
Bioturbational mixing is now applied to the sediment layers:
                \vspace{-5pt}\begin{verbatim}
sg_ctrl_sed_bioturb=.true.
                \end{verbatim}\vspace{-5pt}
No detrital (non carbonate) flux is specified as an explicit field is provided as part of the \textit{forcing} (see below).
                \item \texttt{--- WEATHERING ---}
                \\ An \textit{open  system} is now specified:
\vspace{-5pt}\begin{verbatim}
bg_ctrl_force_sed_closedsystem=.false.
                \end{verbatim}\vspace{-5pt}
                Temperature-only dependence plus baseline fluxes of silicate and carbonate weathering is specified in the bulk of the lines in this section (plus volcanic CO$_{2}$ out-gassing and isotopic compositions):
\vspace{-5pt}\begin{verbatim}
rg_opt_weather_T_Ca=.true.
rg_opt_weather_T_Si=.true.
rg_par_ref_T0=8.48
rg_par_weather_CaCO3=5.58E+12
rg_par_weather_CaCO3_d13C=12.4
rg_par_weather_CaSiO3=5.58E+12 
rg_par_outgas_CO2=5.58E+12
rg_par_outgas_CO2_d13C=-6.4
                \end{verbatim}\vspace{-5pt}
giving a total base (before temperature modification) weathering rate of 11.19 Tmol Ca yr$^{-1}$.
\\ IN this simple example -- silicate and carbonate weathering are split 50:50.  
                \item \texttt{--- GEOCHEM ACCELERATION ---}
The only really significant change is in the prescription of geochemical acceleration:
\vspace{-5pt}\begin{verbatim}
ma_gem_notyr=10
ma_gem_yr=990
ma_gem_adapt_auto=.false.
\end{verbatim}\vspace{-5pt}
which species a fixed (\texttt{ma\_gem\_adapt\_auto} is \texttt{.false.}) interleaving of non-accelerated and accelerated blocks of time-stepping.
        \end{compactitem}

\noindent \textbf{Pre-requisites:} \texttt{EXAMPLE.worbe2.RidgwellHargreaves1997\_S36x36.SPIN1}

\noindent \textbf{Execution}: Command-line launching of the model experiment for a 500000 year spin up:
\vspace{-10pt}\small\begin{verbatim}
./runmuffin.t100.sh cgenie.eb_go_gs_ac_bg_sg_rg.worbe2.BASE / 
EXAMPLE.worbe2.RidgwellHargreaves1997_S36x36.SPIN2gl 500000 
EXAMPLE.worbe2.RidgwellHargreaves1997_S36x36.SPIN1
\end{verbatim}\normalsize\vspace{-10pt}
Note that an even further extended run-time of the order 1 Myr (here: 500 kyr) may be necessary to *completely* equilibrate carbon isotopes in a fully open system. (It rather helps if the system is reasonable close to balance between the flux-weighted $\delta^{13}$C of the inputs and outputs of to start with ...)

\noindent \textbf{Ideas for further development:} 

\noindent \textbf{Relevant HOW-TOs}: 'Accelerate the weathering-sedimentation mass balance'.

%---------------------------------------------------------------------------------------------------------------------------------

\subsubsection{Historical transient}\label{EXAMPLE.worbe2.RidgwellHargreaves1997_S36x36.HISTORICAL}

BLAH

%---------------------------------------------------------------------------------------------------------------------------------
%---------------------------------------------------------------------------------------------------------------------------------

\subsection{\textit{Ridgwell and Schmidt} [2010]}

The model configuration used by \textit{Ridgwell and Schmidt} [2010] is based on a 16 vertical level ocean circulation with an early Eocene continental configuration, basic (PO$_{4}$-only nutrient) based ocean carbon cycle, plus pelagic carbonate (CaCO$_{3}$) sedimentation and burial in the  deep-sea and weathering input.
\\ There are two main stages required in the \textit{spin-up}, as described in \textit{Ridgwell and Hargreaves} [2007] and for which example experiments and associated description is provided above. A third example experiment is given for one of the CO$_{2}$ release experiments described in \textit{Ridgwell and Schmidt} [2010].

        \begin{compactenum}
        
                \item The first stage of \textit{spin-up} creates an equilibrium climate and ocean circulation consistent with the prescribed value of atmospheric \textit{p}CO$_{2}$ and a given (initial, observed) value of ocean alkalinity. It is also used to diagnose the equilibrium distribution of surface sediment composition and hence the global rate of (pelagic) CaCO$_{3}$ burial. Bioturbation (mixing of the sediments) is turned 'off' to facilitate a shorter time to equilibrium of surface sediment composition and CaCO$_{3}$ burial rates.
                
                \item The second stage of \textit{spin-up} is used to equilibrate the sediments with bioturbational mixing enabled and to a depth in the sediment column sufficient to support the dissolution ('chemical erosion') of previously preserved CaCO$_{3}$ in any subsequent perturbation e.g. CO2 release experiment. This 2nd stage is also where the system is 'opened' and weathering inputs and sediment loses brought into balance. 
                
                \item An experiment involving a CO$_{2}$ release to the system created at the end of the 2nd stage \textit{spin-up}. Consistent with \textit{Ridgwell and Schmidt} [2010], no weathering feedbacks are specified (although climate feedbacks are enabled). The particular example given is for a 2180 PgC release consistent with a PETM CH$_{4}$ carbon source> The release is over 1 kyr (and the experiment run for a total of 11 kyr).
                
        \end{compactenum}

%---------------------------------------------------------------------------------------------------------------------------------

\subsubsection{1st stage (CLOSED CaCO3 cycle) \textit{spin-up}}\label{EXAMPLE.p0055c.RidgwellSchmidt2010.SPIN1}

This example uses an early Eocene continental configuration, with a basic (P-only) based ocean carbon cycle, and deep-sea (CaCO3) sedimentation and burial and weathering input in a '\textit{closed system}'.

\noindent \textbf{Physics configuration}: GOLDSTEIN ocean + sea-ice + EMBM atmosphere modules with with seasonal insolation forcing. Adjusted: continental configuration, planetary albedo, solar constant, ocean salinity, annual averaged wind stress and winds.

\noindent \textbf{Biogeochemistry configuration}: Basic ocean (and atmosphere) carbon cycle as described \textit{Cao et al.} [2009]. Atmospheric restoring of pCO2 (+ d13C).

\noindent \textbf{Base-config:} The \textit{base-config} file is:
\vspace{-10pt}\begin{verbatim}cgenie.eb_go_gs_ac_bg_sg_rg.p0055c.BASES\end{verbatim}\vspace{-10pt}
and specifies sediment (SEDGEM) and weathering (ROKGEM) modules in addition to the basic climate and ocean-only carbon cycle components.
This differs from the equivalent standard modern configuration in having:
        \begin{compactitem}
        \item An early Eocene continental configuration is prescribed, and the grid started at -180E.
                \item CH4 (and d13C of CH4) tracers are selected as additional tracers. 
                \item Ocean temperatures are initialized at 10C:
                \\ \texttt{go\_10=10.}0, \texttt{go\_10=10.0}.
                \item Solar constant reduced by 0.46\% for end Paleocene:
                \\ \texttt{ma\_genie\_solar\_constant=1361.7}.
                \item Planetary albedo adjusted:
                \vspace{-5pt}\begin{verbatim}
ea_albedop_offs=0.200
ea_albedop_amp=0.260
ea_albedop_skew=0.0
ea_albedop_skewp=0
ea_albedop_mod2=-0.000
ea_albedop_mod4=0.000
ea_albedop_mod6=0.250
\end{verbatim}\vspace{-5pt}
\item Ocean salinity reduced by 1 per mil to take into account absence of large land-based ice sheets:
\\ \texttt{go\_saln0=33.9.}
        \end{compactitem}

\noindent \textbf{User-config:} The \textit{user-config}:
\vspace{-10pt}\begin{verbatim}EXAMPLE.p0055c.RidgwellSchmidt2010.SPIN1\end{verbatim}\vspace{-10pt}
differs from the 16-level ocean biogeochemistry of \textit{Cao et al.} [2009] and from the sediment configuration of \textit{Ridgwell and Hargreaves} [2007] and the examples:
\\ \texttt{EXAMPLE.worjh2.Caoetal2009.SPIN} and \texttt{EXAMPLE.worbe2.RidgwellHargreaves1997\_S36x36.SPIN1}, respectively, by:
\begin{compactitem}
                \item \texttt{--- INORGANIC MATTER EXPORT RATIOS ---}
                \\ A uniform CaCO3:POC biological export ratio is set:
\vspace{-5pt}\begin{verbatim}bg_par_bio_red_POC_CaCO3=0.200\end{verbatim}\vspace{-5pt}
and made independent of ambient saturation state by:
\vspace{-5pt}\begin{verbatim}bg_par_bio_red_POC_CaCO3_pP=0.0\end{verbatim}\vspace{-5pt}
                \item \texttt{--- SEDIMENTS ---}
                \\ Sediment topographic configuration:
\vspace{-5pt}\begin{verbatim}
# sediment water depth grid name
sg_par_sed_topo_D_name='p0055x_topo.36x36x16'
# reef mask
sg_par_sed_reef_mask_name='p0055x_reef_mask.36x36x16'
# neritic depth cutoff
sg_par_sed_Dmax_neritic=175.0
# sediment core save mask name
sg_par_sedcore_save_mask_name='p0055x_save_mask.36x36x16'               
\end{verbatim}\vspace{-5pt}
which specifies Eocene sediment depths set used by \textit{Ridgwell and Schmidt} [2010] along with a set of example sediment core locations and modified topographies (not actually used by \textit{Ridgwell and Schmidt} [2010]).
Note that these topography, reef mask, and sediment core sites have been supplanted by an identical basic deep-sea sediment topography but with the core locations and paleo depths of \textit{Panchuk et al.} [2008] -- see example \texttt{EXAMPLE.p0055c.Panchuketal2013.SPIN1}.
\\Carbonate diagenesis model:
\vspace{-5pt}\begin{verbatim}
# sediment diagenesis options
sg_par_sed_diagen_CaCO3opt="archer1991explicit"
sg_ctrl_sed_bioturb=.false.
sg_ctrl_sed_bioturb_Archer=.false.
sg_par_n_sed_mix=20
\end{verbatim}\vspace{-5pt}
which specifies the explicit \textit{Archer} [1991] sediment diagenesis model as used in \textit{Ridgwell} [2007] rather than the look-up table approximation of \textit{Ridgwell} [2001].
\\Finally, a uniform detrital flux to the sediments is specified:
\vspace{-5pt}\begin{verbatim}
# additional detrital flux (g cm-2 kyr-1)
sg_par_sed_fdet=0.180
\end{verbatim}\vspace{-5pt}
following \textit{Panchuk et al.} [2008].
                \item \texttt{--- FORCINGS ---}
        \\ The selected \textit{forcing} prescribes fixed boundary conditions only of atmospheric pCO2 (+ d13C):
\vspace{-5pt}\begin{verbatim}
bg_par_forcing_name="pyyyyz.RpCO2_Rp13CO2"
bg_par_atm_force_scale_val_3=834.0E-06
bg_par_atm_force_scale_val_4=-4.9
\end{verbatim}\vspace{-5pt}
at x3 CO2 and a heavier atmospheric isotopic signature.
A different wind-speed field is applied for calculating air-sea gas exchange:
                \item \texttt{--- MISC ---}
Finally, a different-from-modern initial ocean alkalinity is specified:
\vspace{-5pt}\begin{verbatim}
bg_par_windspeed_file="p0055c_windspeed.dat"
\end{verbatim}\vspace{-5pt}
and the gas exchange coefficient itself is re-scaled to give ~0.058 mol m-2 yr-1 uatm-1 global mean air-sea coefficient:
\vspace{-5pt}\begin{verbatim}
bg_par_gastransfer_a=0.5196
\end{verbatim}\vspace{-5pt}
\vspace{-5pt}\begin{verbatim}
bg_ocn_init_12=1.975E-03
                \end{verbatim}\vspace{-5pt}
again following \textit{Panchuk et al.} [2008] and as described in \textit{Ridgwell and Schmidt} [2010].
        \end{compactitem}

\noindent \textbf{Pre-requisites:} NONE.

\noindent \textbf{Execution}: Command-line launching\footnote{Note here that the alternative (older papers) time-stepping is used and hence the run script is \texttt{runmuffin.t100.sh} rather than \texttt{runmuffin.sh}.} of the model experiment for a 20000 year integration:
\vspace{-5pt}\begin{verbatim}./runmuffin.t100.sh cgenie.eb_go_gs_ac_bg_sg_rg.p0055c.BASES /
EXAMPLE.p0055c.RidgwellSchmidt2010.SPIN1 20000\end{verbatim}\vspace{-5pt}

\noindent \textbf{Ideas for further development:} 

\noindent \textbf{Relevant HOW-TO:} 'Spin-up the full marine carbon cycle including sediments'

%---------------------------------------------------------------------------------------------------------------------------------

\subsubsection{2nd stage (OPEN CaCO3 cycle) \textit{spin-up}}\label{EXAMPLE.p0055c.RidgwellSchmidt2010.SPIN2}

This is the second stage \textsl{spin-up} of the \textit{Ridgwell and Schmidt} [2010] configuration of early Eocene marine CaCO$_{3}$ cycling.

\noindent \textbf{Physics configuration}: GOLDSTEIN ocean + sea-ice + EMBM atmosphere modules with with seasonal insolation forcing. Adjusted: continental configuration, planetary albedo, solar constant, ocean salinity, annual averaged wind stress and winds.

\noindent \textbf{Biogeochemistry configuration}: Basic ocean (and atmosphere) carbon cycle as described \textit{Cao et al.} [2009]. Atmospheric restoring of \textit{p}CO$_{2}$ (+ $\delta^{13}$C). Carbonate (only) weathering adjusted and set equal to the global CaCO$_{3}$ (pelagic) burial flux as diagnosed form the 1st stage \textit{spin-up} 
\\ (\texttt{EXAMPLE.p0055c.RidgwellSchmidt2010.SPIN1}),
\\ in an \textit{open system} balance as per the methodology described by \textit{Ridgwell and Hargreaves} [2007].

\noindent \textbf{Base-config:} The \textit{base-config} file is as per the 1st stage \textit{spin-up}:
\vspace{-10pt}\begin{verbatim}cgenie.eb_go_gs_ac_bg_sg_rg.p0055c.BASES\end{verbatim}\vspace{-10pt}
and specifies sediment (SEDGEM) and weathering (ROKGEM) modules in addition to the basic climate and ocean-only carbon cycle components.

\noindent \textbf{User-config:} The associated \textit{user-config}:
\vspace{-10pt}\begin{verbatim}EXAMPLE.p0055c.RidgwellSchmidt2010.SPIN2\end{verbatim}\vspace{-10pt}
has the following noteworthy differences compared to the 1st stage \textit{spin-up}\footnote{These changes are similar to the changes between the 1st and 2nd stage \textit{spin-up} of the modern ocean, \textit{Ridgwell and Hargreaves} [2007] examples.}:

\begin{compactitem}
                \item \texttt{--- SEDIMENTS ---}
Bioturbational mixing is now applied to the sediment layers:
                \vspace{-5pt}\begin{verbatim}
sg_ctrl_sed_bioturb=.true.
                \end{verbatim}\vspace{-5pt}
                \item \texttt{--- WEATHERING ---}
                \\ An \textit{open  system} is now specified:
\vspace{-5pt}\begin{verbatim}
bg_ctrl_force_sed_closedsystem=.false.
                \end{verbatim}\vspace{-5pt}
The prescribed weathering flux (\texttt{rg\_par\_weather\_CaCO3}) is now revised and set equal to the diagnosed global CaCO3 burial rate ('\texttt{Total CaCO3 pres (sediment grid)}') as reported in the SEDGEM module results file (\texttt{seddiag\_misc\_DATA\_GLOBAL.res}), giving:
\vspace{-5pt}\begin{verbatim}
rg_par_weather_CaCO3=0.150212E+14
                \end{verbatim}\vspace{-5pt}
                In addition -- balance in the d13C system can be approximately achieved by setting the $\delta^{13}$C of weathered carbonates to the mean $\delta^{13}$C of buried pelagic carbonates (which is given in the BIOGEM time-series file: \texttt{biogem\_series\_sed\_CaCO3\_13C.res}), e.g.:
                \vspace{-5pt}\begin{verbatim}
rg_par_weather_CaCO3_d13C=3.794
                \end{verbatim}\vspace{-5pt}
        \end{compactitem}

\noindent \textbf{Pre-requisites:} \texttt{EXAMPLE.p0055c.RidgwellSchmidt2010.SPIN1}

\noindent \textbf{Execution}: Command-line launching of the model experiment for a 50000 year spin up:
\vspace{-10pt}\small\begin{verbatim}
./runmuffin.t100.sh cgenie.eb_go_gs_ac_bg_sg_rg.p0055c.BASES / 
EXAMPLE.p0055c.RidgwellSchmidt2010.SPIN2 50000 
EXAMPLE.p0055c.RidgwellSchmidt2010.SPIN1
\end{verbatim}\normalsize\vspace{-10pt}

\noindent \textbf{Ideas for further development:} 

\noindent \textbf{Relevant HOW-TOs:} 'Spin-up the full marine carbon cycle including sediments'.

%---------------------------------------------------------------------------------------------------------------------------------

\subsubsection{CO2 release experiment}\label{EXAMPLE.p0055c.RidgwellSchmidt2010.EMISSIONS}

This is an example CO$_{2}$ emissions experiment from \textit{Ridgwell and Schmidt} [2010].

\noindent \textbf{Physics configuration}: As per \texttt{EXAMPLE.p0055c.RidgwellSchmidt2010.SPIN2}.

\noindent \textbf{Biogeochemistry configuration}: As per \texttt{EXAMPLE.p0055c.RidgwellSchmidt2010.SPIN2}, except for a prescribed emissions of CO$_{2}$ to the atmosphere rather than atmospheric restoring of \textit{p}CO$_{2}$ (+ $\delta^{13}$C).

\noindent \textbf{Base-config:} The \textit{base-config} file is as per the 1st stage \textit{spin-up}:
\vspace{-10pt}\begin{verbatim}cgenie.eb_go_gs_ac_bg_sg_rg.p0055c.BASES\end{verbatim}\vspace{-10pt}

\noindent \textbf{User-config:} The associated \textit{user-config}:
\vspace{-10pt}\begin{verbatim}EXAMPLE.p0055c.RidgwellSchmidt2010.EMISSIONS\end{verbatim}\vspace{-10pt}
has the following differences compared to the 2st stage \textit{spin-up}:

\begin{compactitem}
                \item \texttt{--- FORCINGS ---}
Finally, a pulse of CO$_{2}$ to the atmosphere is specified:
\vspace{-5pt}\begin{verbatim}
bg_par_forcing_name="pyyyyz.FpCO2_Fp13CO2"
bg_par_atm_force_scale_val_3=2.180
bg_par_atm_force_scale_val_4=-60.0
bg_par_atm_force_scale_time_3=1000.0
bg_par_atm_force_scale_time_4=1000.0
                \end{verbatim}\vspace{-5pt}
The \textit{forcing} \texttt{pyyyyz.FpCO2\_Fp13CO2} itself is generic and corresponds to a 1 PgC (8.3333e+013 mol C) release over 1 year. This is scaled to 2180 PgC by parameter \texttt{bg\_par\_atm\_force\_scale\_val\_3} and a duration of 1000 years by parameter \texttt{bg\_par\_atm\_force\_scale\_time\_3}. An isotopic signature appropriate for methane (hydrates) is also set, by (\texttt{bg\_par\_atm\_force\_scale\_val\_4}) (and whose time-scale is also scaled to 1000 years for completeness --  parameter \texttt{bg\_par\_atm\_force\_scale\_time\_4}).
        \end{compactitem}

\noindent \textbf{Pre-requisites:} \texttt{EXAMPLE.p0055c.RidgwellSchmidt2010.SPIN2}

\noindent \textbf{Execution}: Command-line launching of a 11000 year transient experiment:
\vspace{-10pt}\small\begin{verbatim}
./runmuffin.t100.sh cgenie.eb_go_gs_ac_bg_sg_rg.p0055c.BASES / 
EXAMPLE.p0055c.RidgwellSchmidt2010.EMISSIONS 11000 
EXAMPLE.p0055c.RidgwellSchmidt2010.SPIN2
\end{verbatim}\normalsize\vspace{-10pt}

\noindent \textbf{Ideas for further development:} Adding (silicate) weathering feedback -- see cGENIE \textit{HOTWO}. Creating a control experiment (important) by scaling the emissions to zero, i.e.:
\\\texttt{bg\_par\_atm\_force\_scale\_val\_3=0.0}

\noindent \textbf{Relevant HOW-TOs:} \textit{Set up a (silicate) weathering feedback}.

%---------------------------------------------------------------------------------------------------------------------------------
%---------------------------------------------------------------------------------------------------------------------------------

\subsection{\textit{Zeebe et al.} [2014]}

BLAH

%---------------------------------------------------------------------------------------------------------------------------------
%--- Submitted cGENIE configurations ---------------------------------------------------------------------------
%---------------------------------------------------------------------------------------------------------------------------------

\newpage
\section{Submitted (or in prep) configurations (\textit{spin-ups}) and experiments}\label{submitted}

This section details model configurations and experiments that are not yet published, but are at least not only in use, but written up to the extent that they are submitted (or *almost*). Hence, they cannot be relied upon to be set-in-stone, and may in due course be propoted to the published Examples section.

%---------------------------------------------------------------------------------------------------------------------------------
%---------------------------------------------------------------------------------------------------------------------------------

\subsection{\textit{Gibbs et al.} [2015]}

This is not (yet) a published configuration, but illustrates setting a simple silicate weathering feedback. 

The model configuration is based on \textit{Ridgwell and Schmidt} [2010] -- 16 vertical level ocean circulation with an early Eocene continental configuration, basic (PO$_{4}$-only nutrient) based ocean carbon cycle, plus pelagic carbonate (CaCO$_{3}$) sedimentation and burial in the  deep-sea and weathering input (itself closely related to \textit{Panchuk et al.} [2008]), and differs only in the definition of the sediment core locations (specific to the data set used in \textit{Gibbs et al.} [2015]) and with unadjusted paleo depths.

As per described in \textit{Ridgwell and Hargreaves} [2007] and \textit{Panchuk et al.} [2008] etc, there are two main stages required in the \textit{spin-up}, but with the 2nd stage \textit{spin-up} replaced with an accelerated (with GEMlite) configuration plus silicate weathering feedback.

        \begin{compactenum}
        
                \item As previously: the first stage of \textit{spin-up} creates an equilibrium climate and ocean circulation consistent with the prescribed value of atmospheric \textit{p}CO$_{2}$ and a given (initial, observed) value of ocean alkalinity and is used to diagnose the equilibrium distribution of surface sediment composition and hence the global rate of (pelagic) CaCO$_{3}$ burial.
                
                \item An accelerated 2nd stage \textit{spin-up} including silicate weathering feedback.
                
        \end{compactenum}

%---------------------------------------------------------------------------------------------------------------------------------

\subsubsection{1st stage (CLOSED CaCO3 cycle) \textit{spin-up}}\label{EXAMPLE.p0055c.Gibbsetal2015.SPIN1}

First stage \textsl{spin-up} of the \textit{Gibbs et al.} [2015] configuration of early Eocene marine CaCO$_{3}$ cycling.

\noindent \textbf{Physics configuration}: GOLDSTEIN ocean + sea-ice + EMBM atmosphere modules with with seasonal insolation forcing. Adjusted: continental configuration, planetary albedo, solar constant, ocean salinity, annual averaged wind stress and winds.

\noindent \textbf{Biogeochemistry configuration}: Basic ocean (and atmosphere) carbon cycle with atmospheric restoring of \textit{p}CO$_{2}$ (+ $\delta^{13}$C).

\noindent \textbf{Base-config:} The \textit{base-config} file is:
\vspace{-10pt}\begin{verbatim}cgenie.eb_go_gs_ac_bg_sg_rg.p0055c.BASES\end{verbatim}\vspace{-10pt}

\noindent \textbf{User-config:} The associated \textit{user-config}:
\vspace{-10pt}\begin{verbatim}EXAMPLE.p0055c.Gibbsetal2015.SPIN1\end{verbatim}\vspace{-10pt}
has the following noteworthy differences compared to the 1st stage \textit{spin-up} of 
\\\texttt{EXAMPLE.p0055c.RidgwellSchmidt2010.SPIN1}:

\begin{compactitem}
                \item \texttt{--- INORGANIC MATTER EXPORT RATIOS ---}
                \vspace{-5pt}\begin{verbatim}
# fixed CaCO3:POC
bg_opt_bio_CaCO3toPOCrainratio='FIXED'
# CaCO3 export as a proportion of particulate organic matter (CaCO3/POC)
bg_par_bio_red_POC_CaCO3=0.200
                \end{verbatim}\vspace{-5pt}
which is a slightly different way of accomplishing the same thing (fixed 0.2 rain ratio) as per the \texttt{Ridgwell and Schmidt} [2010] EXAMPLE.\footnote{The only difference is that this ratio is imposed regardless of surface saturation state, whereas the alternative in \texttt{Ridgwell and Schmidt} [2010] has no CaCO3 export and hence a rain ratio of zero for under-saturated surface conditions}
                \item \texttt{--- SEDIMENTS ---}
                \vspace{-5pt}\begin{verbatim}
# sediment water depth grid name
sg_par_sed_topo_D_name='p0055x.depth.36x36x16.Gibbs'
# reef mask
sg_par_sed_reef_mask_name='p0055x.reefmask.36x36x16.Gibbs'
# neritic depth cutoff
sg_par_sed_Dmax_neritic=175.0
# sediment core save mask name
sg_par_sedcore_save_mask_name='p0055x.sedcoremask.36x36x16.Gibbs'
                \end{verbatim}\vspace{-5pt}
                which differ from \texttt{EXAMPLE.p0055c.RidgwellSchmidt2010.SPIN1} primarily just in the sedcore locations\footnote{\texttt{p0055x.sedcoremask.36x36x16.Gibbs}}.
        \end{compactitem}

\noindent \textbf{Pre-requisites:} [NONE]

\noindent \textbf{Execution}: Command-line launching of the model experiment for a 20000 year run:
\vspace{-10pt}\small\begin{verbatim}
./runmuffin.sh cgenie.eb_go_gs_ac_bg_sg_rg.p0055c.BASES / 
EXAMPLE.p0055c.Gibbsetal2015.SPIN1 20000 
\end{verbatim}\normalsize\vspace{-10pt}
Note here that \texttt{runmuffin.sh} rather than \texttt{runmuffin.t100.sh} (as used in \texttt{Ridgwell and Schmidt} [2010]) is employed. This is primarily to enable a 'cleaner' extraction (averaging) of seasonal or monthly data.

\noindent \textbf{Ideas for further development:} 

\noindent \textbf{Relevant HOW-TOs:} 

%---------------------------------------------------------------------------------------------------------------------------------

\subsubsection{2nd stage (accelerated, OPEN CaCO3 cycle) \textit{spin-up}}\label{EXAMPLE.p0055c.Gibbsetal2015.SPIN2gl}

Second stage \textsl{spin-up} of the \textit{Gibbs et al.} [2015] configuration of early Eocene marine CaCO$_{3}$ cycling -- accelerated and including the silicate weathering feedback.

\noindent \textbf{Physics configuration}: GOLDSTEIN ocean + sea-ice + EMBM atmosphere modules with with seasonal insolation forcing. Adjusted: continental configuration, planetary albedo, solar constant, ocean salinity, annual averaged wind stress and winds.

\noindent \textbf{Biogeochemistry configuration}: Basic ocean (and atmosphere) carbon cycle with atmospheric restoring of \textit{p}CO$_{2}$ (+ $\delta^{13}$C).

\noindent \textbf{Base-config:} The \textit{base-config} file now differs from the 1st stage \textit{spin-up} in that is specifies the GEMlite module (\texttt{gl}):
\vspace{-10pt}\begin{verbatim}cgenie.eb_go_gs_ac_bg_sg_rg_gl.p0055c.BASES\end{verbatim}\vspace{-10pt}

\noindent \textbf{User-config:} The associated \textit{user-config} is:
\vspace{-10pt}\begin{verbatim}EXAMPLE.p0055c.Gibbsetal2015.SPIN2gl\end{verbatim}\vspace{-10pt}
and differs from the standard 2nd stage \textit{spin-up} as follows:

\begin{compactitem}
                \item \texttt{--- WEATHERING ---}
                \\ As before, an \textit{open  system} is specified:
\vspace{-5pt}\begin{verbatim}
bg_ctrl_force_sed_closedsystem=.false.
                \end{verbatim}\vspace{-5pt}
                Now however, a temperature dependence on silicate and carbonate weathering is specified:
\vspace{-5pt}\begin{verbatim}
rg_opt_weather_T_Ca=.true.
rg_opt_weather_T_Si=.true.
                \end{verbatim}\vspace{-5pt}
A reference temperature parameter, \texttt{rg\_par\_ref\_T0} representing controlling relative changes in weathering fluxes needs to be set. The value for the mean global land surface temperature is given in the \textbf{BIOGEM} \textit{time-series} file\footnote{Note that in the BIOGEM time-series output, this is reported as an annual average. In the ROGEM output, the instantaneous value is reported. Use the annual average.}:
\\\texttt{biogem\_series\_misc\_SLT.res}
\\which in the \texttt{EXAMPLE.p0055c.Gibbsetal2015.SPIN1} \textit{spin-up} is 19.00$^{\circ}$C, resulting in the addition to the \textit{user-config}:
\vspace{-5pt}\begin{verbatim}
rg_par_ref_T0=19.00
                \end{verbatim}\vspace{-5pt}
                It is important to note here that the reference value set is the mean global annual average temperature for the Eocene, not pre-industrial. A strict interpretation of the GEOCARB formulations would lead to a modern reference temperature being used, and base fluxes scaled accordingly (see below). However, given that the continental configuration, lithologies, amount of land ice, etc etc are very different, it is arguably acceptable to adjust the reference values and assume a similar sensitivity of weathering to a change in climate.
The values for the baseline fluxes (for an annual average temperature equal to the reference temperature in the case of the 2 weathering fluxes) are:
\vspace{-5pt}\begin{verbatim}
rg_par_weather_CaCO3=7.241E+12
rg_par_weather_CaSiO3=7.150E+12
rg_par_outgas_CO2=7.241E+12
                \end{verbatim}\vspace{-5pt}
Finally, the carbon isotopic signature of volcanic CO$_{2}$ out-gassing is assigned a value of -6.0\permil. The $\delta$$^{13}$C of weathered CaCO$_{3}$ is then simply set in order that inputs equal the mean $\delta$$^{13}$C of carbonate burial, which as given in the \textbf{BIOGEM} \textit{time-series} file:
\\\texttt{biogem\_series\_sed\_CaCO3\_13C.res}
giving:
\vspace{-5pt}\begin{verbatim}
rg_par_outgas_CO2_d13C=-6.0
rg_par_weather_CaCO3_d13C=13.58
\end{verbatim}\vspace{-5pt}
                \item \texttt{--- GEOCHEM ACCELERATION ---}
Geochemical acceleration is specified in a 10 (normal) to 90 (accelerated) years ratio\footnote{This ratio could equally probably be 10 to 990. This lower gearing is chosen to give a total 20 kyr of full model simulation (out of the 200 kyr total), otherwise the full model runs for only 2 kyr out of the 200 kyr total.}:
\vspace{-5pt}\begin{verbatim}
gl_ctrl_update_pCO2=.true.
ma_gem_notyr_min=10
ma_gem_notyr_max=10
ma_gem_yr_min=90
ma_gem_yr_max=90
ma_gem_dyr=0
ma_gem_adapt_auto=.false.
\end{verbatim}\vspace{-5pt}
\item \texttt{--- FORCINGS ---}
No modification of atmospheric composition is needed (as now volcanic CO$_{2}$ out-gassing is ultimately being balance by sediment burial).
\end{compactitem}

How the weathering fluxes are set up as listed above, is a little involved. There are several steps ...

\begin{compactenum}

                \item Firstly, you must decide about the (initial) balance between silicate and carbonate weathering. For simplicity here, we'll assume an equal, 50:50 split, but noting that for modern weathering, it is more like 2:3, silicate vs. carbonate weathering (in germs of Ca2+ flux).
                
                \item The total weathering flux (before temperature modification) is taken from the total CaCO$_{3}$ burial flux\footnote{Outputted to \textbf{seddiag\_misc\_DATA\_GLOBAL.res} by SEDGEM} calculated by \texttt{EXAMPLE.p0055c.Gibbsetal2015.SPIN1} (the \textit{spin-up} for this experiment) -- 0.144822E+14 mol Ca$^{2+}$ yr$^{-1}$.
                Based on assumption \#1 (above), this can now be split tentatively into carbonate and silicate weathering:
\vspace{-5pt}\begin{verbatim}
rg_par_weather_CaCO3=7.241E+12
rg_par_weather_CaSiO3=7.241E+12
                \end{verbatim}\vspace{-5pt}
                in a 50:50 split, and with volcanic out-gassing providing the course of CO2 for the 50\% of CaCO3 burial derived from silicate weathering.
\vspace{-5pt}\begin{verbatim}
rg_par_outgas_CO2=7.241E+12
                \end{verbatim}\vspace{-5pt}
                
                \item However, the seasonal nature of the insolation forcing in conjunction with the non-linear dependency of weathering on the deviation from the reference temperature, means that the mean annual fluxes of weathering from both carbonate and silicate weathering may diverge from these value.
                Checking carbonate weathering first -- the model can be run for a few years with \texttt{rg\_par\_weather\_CaSiO3} set to zero. All Ca$^{2+}$ derived from weathering must hence be from carbonate weathering. This flux is reported as a time-series:
\vspace{-5pt}\begin{verbatim}
biogem_series_diag_weather_Ca.res
\end{verbatim}\vspace{-5pt}
Checking the Ca$^{2+}$ flux from carbonate weathering, it can be seen that it is very close to the pre-set reference value of  7.241E+12 mol Ca$^{2+}$ yr$^{-1}$.
Hence, the recommendation is to leave this setting alone.
                
                \item Next -- silicate weathering.
                \\ Similar to before -- now set the carbonate weathering reference value  \texttt{rg\_par\_weather\_CaCO3} to zero, and keep the silicate weathering reference value at 7.241E+12.
                Again, run the model for just a few years and check the Ca$^{2+}$ flux (now only from silicate weathering).
                In this case, the flux is too high by a little over 1\%, and so needs to be scaled back, giving a final setting of: 
\vspace{-5pt}\begin{verbatim}
rg_par_weather_CaSiO3=7.150E+12
                \end{verbatim}\vspace{-5pt}
                
                \item Finally, there is the question of balancing the long-term carbon isotope cycle.
                The mean global $\delta^{13}$C of CaCO3 is reported in the time-series:
\vspace{-5pt}\begin{verbatim}
biogem_series_sed_CaCO3_13C.res
\end{verbatim}\vspace{-5pt}
which in this example is 3.791.
This isotopic signature, multiplied by the total global burial flux (0.144822E+14 mol CaCO3 yr$^{-1}$) must be equal to: the volcanic flux (\texttt{rg\_par\_outgas\_CO2} -- KNOWN) multiplied by its isotopic signature (\texttt{rg\_par\_outgas\_CO2\_d13C} -- KNOWN), plus the carbonate weathering flux (\texttt{rg\_par\_weather\_CaCO3} -- KNOWN) multiplied by its isotopic signature (UNKNOWN). The final weathering parameter is hence solved to be:
\vspace{-5pt}\begin{verbatim}
rg_par_weather_CaCO3_d13C=13.58
\end{verbatim}\vspace{-5pt}

\end{compactenum}

\noindent \textbf{Pre-requisites:} EXAMPLE.p0055c.Gibbsetal2015.SPIN1

\noindent \textbf{Execution}: Command-line launching of the model experiment for a 200000 year run:
\vspace{-10pt}\small\begin{verbatim}
./runmuffin.sh cgenie.eb_go_gs_ac_bg_sg_rg_gl.p0055c.BASES / 
EXAMPLE.p0055c.Gibbsetal2015.SPIN2gl 200000 
EXAMPLE.p0055c.Gibbsetal2015.SPIN1 
\end{verbatim}\normalsize\vspace{-10pt}

\noindent \textbf{Ideas for further development:} 

\noindent \textbf{Relevant HOW-TOs:} '\textit{Set up a (silicate) weathering feedback}',
\\'\textit{Accelerate the weathering-sedimentation mass balance (GEMlite)}'

%---------------------------------------------------------------------------------------------------------------------------------
%---------------------------------------------------------------------------------------------------------------------------------

\subsubsection{1st stage (CLOSED CaCO3 cycle) \textit{spin-up}}\label{EXAMPLE.p0055c.Jennionsetal2015.SPIN1}

\subsection{\textit{Jennions et al.} [2015]}

This is not (yet) a published configuration, but illustrates setting a more complex silicate weathering feedback (including precip and terrestrial biosphere factors). 

The model configuration is based on \textit{Ridgwell and Schmidt} [2010] (16 vertical level ocean circulation with an early Eocene continental configuration, basic (PO$_{4}$-only nutrient) based ocean carbon cycle, plus pelagic carbonate (CaCO$_{3}$) sedimentation and burial in the  deep-sea and weathering input) and differs only in the definition of the sediment core locations and paleo depths, which come from \textit{Panchuk et al.} [2008].

As per described in \textit{Ridgwell and Hargreaves} [2007], there are two main stages required in the \textit{spin-up}, for which example experiments and associated description is provided above. A third experiment is given that serves as an example of accelerating (with GEMlite) the stage 2 \textit{spin-up} and adding a silicate weathering feedback.

        \begin{compactenum}
        
                \item As previously: the first stage of \textit{spin-up} creates an equilibrium climate and ocean circulation consistent with the prescribed value of atmospheric \textit{p}CO$_{2}$ and a given (initial, observed) value of ocean alkalinity and is used to diagnose the equilibrium distribution of surface sediment composition and hence the global rate of (pelagic) CaCO$_{3}$ burial.
                
                \item The second stage of \textit{spin-up} is used to equilibrate the sediments with bioturbational mixing enabled. 
                
                \item A substitute experiment for the normal 2nd stage \textit{spin-up} and including the silicate weathering feedback.
                
        \end{compactenum}

%---------------------------------------------------------------------------------------------------------------------------------

\subsubsection{1st stage (CLOSED CaCO3 cycle) \textit{spin-up}}\label{EXAMPLE.p0055c.Jennionsetal2014.SPIN1}

First stage \textsl{spin-up} of the \textit{Jennions et al.} [2014] configuration of early Eocene marine CaCO$_{3}$ cycling.

\noindent \textbf{Physics configuration}: GOLDSTEIN ocean + sea-ice + EMBM atmosphere modules with with seasonal insolation forcing. Adjusted: continental configuration, planetary albedo, solar constant, ocean salinity, annual averaged wind stress and winds.

\noindent \textbf{Biogeochemistry configuration}: Basic ocean (and atmosphere) carbon cycle with atmospheric restoring of \textit{p}CO$_{2}$ (+ $\delta^{13}$C).

\noindent \textbf{Base-config:} The \textit{base-config} file is:
\vspace{-10pt}\begin{verbatim}cgenie.eb_go_gs_ac_bg_sg_rg.p0055c.BASES\end{verbatim}\vspace{-10pt}

\noindent \textbf{User-config:} The associated \textit{user-config}:
\vspace{-10pt}\begin{verbatim}EXAMPLE.p0055c.Jennionsetal2014.SPIN1\end{verbatim}\vspace{-10pt}
has the following noteworthy differences compared to the 1st stage \textit{spin-up} of 
\\\texttt{EXAMPLE.p0055c.RidgwellSchmidt2010.SPIN1}:

\begin{compactitem}
                \item \texttt{--- INORGANIC MATTER EXPORT RATIOS ---}
                \vspace{-5pt}\begin{verbatim}
# fixed CaCO3:POC
bg_opt_bio_CaCO3toPOCrainratio='FIXED'
# CaCO3 export as a proportion of particulate organic matter (CaCO3/POC)
bg_par_bio_red_POC_CaCO3=0.200
                \end{verbatim}\vspace{-5pt}
which is a slightly different way of accomplishing the same thing (fixed 0.2 rain ratio) as per the \texttt{Ridgwell and Schmidt} [2010] EXAMPLE.\footnote{The only difference is that this ratio is imposed regardless of surface saturation state, whereas the alternative in \texttt{Ridgwell and Schmidt} [2010] has no CaCO3 export and hence a rain ratio of zero for under-saturated surface conditions}
                \item \texttt{--- SEDIMENTS ---}
                \vspace{-5pt}\begin{verbatim}
# sediment water depth grid name
sg_par_sed_topo_D_name='p0055x.depth.36x36x16.Panchuk'
# reef mask
sg_par_sed_reef_mask_name='p0055x.reefmask.36x36x16.Panchuk'
# neritic depth cutoff
sg_par_sed_Dmax_neritic=175.0
# sediment core save mask name
sg_par_sedcore_save_mask_name='p0055x.sedcoremask.36x36x16.Panchuk'
                \end{verbatim}\vspace{-5pt}
                In addition, there is an adjustment to the detrital flux -- becasue site-specific detrital fluxes are used, a globally-uniform value cannor be applied, here:
                \vspace{-5pt}\begin{verbatim}
# NO additional detrital flux (g cm-2 kyr-1)
sg_par_sed_fdet=0.0
                \end{verbatim}\vspace{-5pt}
                and instead a 2D \textit{forcing} field will be specified (below).
                \item \texttt{--- FORCINGS ---}
                In addition to atmospheric pCO2 (and d13C) restoring, a 2D detrital flux field is specified -- uniform everywhere except at the location of the Walvis Ridge sedcore sites:
                \vspace{-5pt}\begin{verbatim}forcings
bg_par_forcing_name="p0055c.RpCO2_Rp13CO2.detJennions"
bg_par_atm_force_scale_val_3=834.0E-06
bg_par_atm_force_scale_val_4=-4.9
                \end{verbatim}\vspace{-5pt}
        \end{compactitem}

\noindent \textbf{Pre-requisites:} [NONE]

\noindent \textbf{Execution}: Command-line launching of the model experiment for a 20000 year run:
\vspace{-10pt}\small\begin{verbatim}
./runmuffin.sh cgenie.eb_go_gs_ac_bg_sg_rg.p0055c.BASES / 
EXAMPLE.p0055c.Jennionsetal2014.SPIN1 20000 
\end{verbatim}\normalsize\vspace{-10pt}
Note here that \texttt{runmuffin.sh} rather than \texttt{runmuffin.t100.sh} (as used in \texttt{Ridgwell and Schmidt} [2010]) is employed. This is primarily to enable a 'cleaner' extraction (averaging) of seasonal or monthly data.

\noindent \textbf{Ideas for further development:} 

\noindent \textbf{Relevant HOW-TOs:} 

%---------------------------------------------------------------------------------------------------------------------------------

\subsubsection{2nd stage (OPEN CaCO3 cycle) \textit{spin-up}}\label{EXAMPLE.p0055c.Jennionsetal2014.SPIN2}

Second stage \textsl{spin-up} of the \textit{Jennions et al.} [2014] configuration of early Eocene marine CaCO$_{3}$ cycling.

\noindent \textbf{Physics configuration}: GOLDSTEIN ocean + sea-ice + EMBM atmosphere modules with with seasonal insolation forcing. Adjusted: continental configuration, planetary albedo, solar constant, ocean salinity, annual averaged wind stress and winds.

\noindent \textbf{Biogeochemistry configuration}: Basic ocean (and atmosphere) carbon cycle with atmospheric restoring of \textit{p}CO$_{2}$ (+ $\delta^{13}$C).

\noindent \textbf{Base-config:} The \textit{base-config} file is as per the 1st stage \textit{spin-up}:
\vspace{-10pt}\begin{verbatim}cgenie.eb_go_gs_ac_bg_sg_rg.p0055c.BASES\end{verbatim}\vspace{-10pt}

\noindent \textbf{User-config:} The associated \textit{user-config} is:
\vspace{-10pt}\begin{verbatim}EXAMPLE.p0055c.Jennionsetal2014.SPIN2\end{verbatim}\vspace{-10pt}
As per the previous examples of closed/open sediment \textit{spin-up} methodology -- \textit{Ridgwell and Hargreaves }[2007] and \textit{Ridgwell and Schmidt} [2010], changes to the \textit{user-config} include turning bioturbation 'on', specifying an 'open' system, and setting the total (CaCO$_{3}$) weathering flux equal to global CaCO$_{3}$ burial recorded at the end of the 1st \textit{spin-up} phase experiment. As such, the changes are no explicitly listed here.

\noindent \textbf{Pre-requisites:} EXAMPLE.p0055c.Jennionsetal2014.SPIN1

\noindent \textbf{Execution}: Command-line launching of the model experiment for a 50000 year run:
\vspace{-10pt}\small\begin{verbatim}
./runmuffin.sh cgenie.eb_go_gs_ac_bg_sg_rg.p0055c.BASES / 
EXAMPLE.p0055c.Jennionsetal2014.SPIN2 50000 
EXAMPLE.p0055c.Jennionsetal2014.SPIN1 
\end{verbatim}\normalsize\vspace{-10pt}

\noindent \textbf{Ideas for further development:} 

\noindent \textbf{Relevant HOW-TOs:} 

%---------------------------------------------------------------------------------------------------------------------------------

\subsubsection{Alternative 2nd stage (accelerated, OPEN CaCO3 cycle) \textit{spin-up}}\label{EXAMPLE.p0055c.Jennionsetal2014.SPIN2gl}

Alternative second stage \textsl{spin-up} of the \textit{Jennions et al.} [2014] configuration of early Eocene marine CaCO$_{3}$ cycling -- accelerated and including the silicate weathering feedback.

\noindent \textbf{Physics configuration}: GOLDSTEIN ocean + sea-ice + EMBM atmosphere modules with with seasonal insolation forcing. Adjusted: continental configuration, planetary albedo, solar constant, ocean salinity, annual averaged wind stress and winds.

\noindent \textbf{Biogeochemistry configuration}: Basic ocean (and atmosphere) carbon cycle with atmospheric restoring of \textit{p}CO$_{2}$ (+ $\delta^{13}$C).

\noindent \textbf{Base-config:} The \textit{base-config} file now differs from the 1st stage \textit{spin-up} in that is specifies the GEMlite module (\texttt{gl}):
\vspace{-10pt}\begin{verbatim}cgenie.eb_go_gs_ac_bg_sg_rg_gl.p0055c.BASES\end{verbatim}\vspace{-10pt}

\noindent \textbf{User-config:} The associated \textit{user-config} is:
\vspace{-10pt}\begin{verbatim}EXAMPLE.p0055c.Jennionsetal2014.SPIN2gl\end{verbatim}\vspace{-10pt}
and differs from the standard 2nd stage \textit{spin-up} as follows:

\begin{compactitem}
                \item \texttt{--- WEATHERING ---}
                \\ As before, an \textit{open  system} is specified:
\vspace{-5pt}\begin{verbatim}
bg_ctrl_force_sed_closedsystem=.false.
                \end{verbatim}\vspace{-5pt}
                Now however, a temperature dependence on silicate and carbonate weathering is specified:
\vspace{-5pt}\begin{verbatim}
rg_opt_weather_T_Ca=.true.
rg_opt_weather_T_Si=.true.
                \end{verbatim}\vspace{-5pt}
A reference temperature parameter, \texttt{rg\_par\_ref\_T0} representing controlling relative changes in weathering fluxes needs to be set. The value for the mean global land surface temperature is given in the \textbf{BIOGEM} \textit{time-series} file\footnote{Note that in the BIOGEM time-series output, this is reported as an annual average. In the ROGEM output, the instantaneous value is reported. Use the annual average.}:
\\\texttt{biogem\_series\_misc\_SLT.res}
\\which in the \texttt{EXAMPLE.p0055c.Jennionsetal2014.SPIN1} \textit{spin-up} is 19.00$^{\circ}$C, resulting in the addition to the \textit{user-config}:
\vspace{-5pt}\begin{verbatim}
rg_par_ref_T0=19.00
                \end{verbatim}\vspace{-5pt}
                It is important to note here that the reference value set is the mean global annual average temperature for the Eocene, not pre-industrial. A strict interpretation of the GEOCARB formulations would lead to a modern reference temperature being used, and base fluxes scaled accordingly (see below). However, given that the continental configuration, lithologies, amount of land ice, etc etc are very different, it is arguably acceptable to adjust the reference values and assume a similar sensitivity of weathering to a change in climate.
                                \\One option is to leave weathering as having a T-only feedback (and which is the single most important). However, for a complete Berner (GEOCARB) like description of the terrestrial weathering feedback on CO$_{2}$, run-off and terrestrial productivity modifiers have to be applied, which are set as follows.
                \\For the run-off modifier of weathering:
\vspace{-5pt}\begin{verbatim}
rg_opt_weather_R_Ca=.true.
rg_opt_weather_R_Si=.true.
                \end{verbatim}\vspace{-5pt}
                This particular parameterization is formulated as a function of SLT and so no additional reference value needs to be set. There is an alternative that can be selected by setting \texttt{rg\_opt\_weather\_R\_explicit=.true.} that instead takes run-off from the mean global precipitation calculated by the EMBM\footnote{In this case a reference (run-off) value is required: \texttt{rg\_par\_ref\_R0}}.
                \\For the terrestrial biosphere productivity modifier of weathering:
\vspace{-5pt}\begin{verbatim}
rg_opt_weather_P_Ca=.true.
rg_opt_weather_P_Si=.true.
                \end{verbatim}\vspace{-5pt}
                Now a reference value is required
\vspace{-5pt}\begin{verbatim}
rg_par_ref_CO20=834.0
                \end{verbatim}\vspace{-5pt}
                                Here the parametrization is formulated as a function of CO$_{2}$ and in units of ppm.
                
                
                
The total weathering flux (before temperature modification) is taken from the total CaCO$_{3}$ burial flux\footnote{Outputted to \textbf{seddiag\_misc\_DATA\_GLOBAL.res} by SEDGEM} calculated by \texttt{EXAMPLE.p0055c.Jennionsetal2014.SPIN1} (the \textit{spin-up} for this experiment) -- 14.77 Tmol Ca yr$^{-1}$.
Initially, this is split between silicate and carbonate weathering in a simple proportion (in terms of Ca$^{2+}$ flux to the ocean). Here, this proportion is 3:2 of carbonate vs. silicate weathering consistent with estimates of the modern balance (estimates actually range form an approximately 2:1 ratio, to 4:3), i.e. giving values of 8.862 Tmol Ca yr$^{-1}$ and 5.806 Tmol Ca yr$^{-1}$ from carbonate and silicate weathering, respectively. To balance the silicate weathering, volcanic CO$_{2}$ out-gassing is then also assigned a value of 5.806 Tmol CO$_{2}$ yr$^{-1}$.
However, the seasonal nature of the insolation forcing in conjunction with the non-linear dependency of weathering on the deviation from the reference temperature, means that there is excess silicate weathering compared to volcanic out-gassing as recorded in the BIOGEM output:
\texttt{biogem\_series\_misc\_exweather\_Ca.res}
in terms of the absolute excess, in mol Ca$^{2+}$ yr$^{-1}$ as well as a percentage.
\\ Splitting exactly global burial 60:40 between the baseline weathering fluxes as described above gives rise to a small imbalance of in silicate weathering compared to CO${_2}$ out-gassing as averaged over the first year. If left unchecked, this small excess of silicate weathering compared to volcanic out-gassing would lead to atmospheric \textit{p}CO${_2}$) draw-down and a slight climate cooling. As described in the HOW-TO, the baseline silicate weathering flux is then reduced by an amount equal to the excess, to give 5.806 mol Ca$^{2+}$ yr$^{-1}$. To maintain the total  weathering flux, the baseline carbonate weathering flux is also adjusted by contrasting the total Ca$^{2+}$ with the required burial rate, leading to an adjustment of the baseline carbonate weathering flux to 8.840 mol Ca$^{2+}$ yr$^{-1}$. The adjusted parameters then read:
\vspace{-5pt}\begin{verbatim}
rg_par_weather_CaCO3=8.840E+12
rg_par_weather_CaSiO3=5.806E+12
rg_par_outgas_CO2=5.908E+12
                \end{verbatim}\vspace{-5pt}
Finally, the carbon isotopic signature of volcanic CO$_{2}$ out-gassing is assigned a value of -6.0\permil. The $\delta$$^{13}$C of weathered CaCO$_{3}$ is then simply set in order that inputs equal the mean $\delta$$^{13}$C of carbonate burial, which as given in the \textbf{BIOGEM} \textit{time-series} file:
\\\texttt{biogem\_series\_sed\_CaCO3\_13C.res}
giving:
\vspace{-5pt}\begin{verbatim}
rg_par_outgas_CO2_d13C=-6.0
rg_par_weather_CaCO3_d13C=10.34
\end{verbatim}\vspace{-5pt}
                \item \texttt{--- GEOCHEM ACCELERATION ---}
Geochemical acceleration is specified in a 10 (normal) to 90 (accelerated) years ratio\footnote{This ratio could equally probably be 10 to 990. This lower gearing is chosen to give a total 20 kyr of full model simulation, otherwise the full model runs for only 2 kyr out of the 200 kyr total.}:
\vspace{-5pt}\begin{verbatim}
gl_ctrl_update_pCO2=.true.
ma_gem_notyr=10
ma_gem_yr=90
ma_gem_adapt_auto=.false.
\end{verbatim}\vspace{-5pt}
\item \texttt{--- FORCINGS ---}
No modification of atmospheric composition is needed (as now volcanic CO$_{2}$ out-gassing is ultimately being balance by sediment burial).
\end{compactitem}

The drift with the adjusted silicate weathering parameters is of the order of 0.4 ppm for atmospheric \textit{p}CO${_2}$), and 0.07\permil for DIC $\delta$$^{13}$C. \textbf{*** NEED UPDATED VALUE ***}

\noindent \textbf{Pre-requisites:} EXAMPLE.p0055c.Jennionsetal2014.SPIN1

\noindent \textbf{Execution}: Command-line launching of the model experiment for a 200000 year run:
\vspace{-10pt}\small\begin{verbatim}
./runmuffin.sh cgenie.eb_go_gs_ac_bg_sg_rg_gl.p0055c.BASES / 
EXAMPLE.p0055c.Jennionsetal2014.SPIN2gl 200000 
EXAMPLE.p0055c.Jennionsetal2014.SPIN1 
\end{verbatim}\normalsize\vspace{-10pt}

\noindent \textbf{Ideas for further development:} 

\noindent \textbf{Relevant HOW-TOs:} '\textit{Set up a (silicate) weathering feedback}',
\\'\textit{Accelerate the weathering-sedimentation mass balance (GEMlite)}'

%---------------------------------------------------------------------------------------------------------------------------------
%--- Idealized cGENIE configurations -----------------------------------------------------------------------------
%---------------------------------------------------------------------------------------------------------------------------------

\newpage
\section{Idealized configurations (\textit{spin-ups}) and experiments}\label{idealized}

%---------------------------------------------------------------------------------------------------------------------------------
%---------------------------------------------------------------------------------------------------------------------------------

Highly idealized continental configurations and experiments.

%---------------------------------------------------------------------------------------------------------------------------------
%---------------------------------------------------------------------------------------------------------------------------------

\subsection{Idealized low resolution configurations -- \textit{spin-ups}}

The following Examples comprise a highly idealized continental configuration at low (18x18) resolution (for the greatest practical computational efficiency):

        \begin{compactenum}
        
                \item \texttt{EXAMPLE.\_rwlla.SPIN}: An abiotic \textit{spin-up} for a 18x18 'ridge world' configuration (actually, not quite, as the land area is ca. 30\% of the Earth's surface rather than being vanishingly small) with a continent bisecting the ocean zonally, with 8 vertical levels.
                
                \item \texttt{EXAMPLE.\_rwlma.SPIN}: An abiotic \textit{spin-up} for a 18x18 'ridge world' configuration (actually, not quite, as the land area is ca. 30\% of the Earth's surface rather than being vanishingly small) with a continent bisecting the ocean zonally, with 16 vertical levels.
        
                \item \texttt{EXAMPLE.\_wwlla.SPIN}: An abiotic \textit{spin-up} for a 18x18 'water world' configuration (no land) with 8 vertical levels.
                
                \item \texttt{EXAMPLE.\_wwlma.SPIN}: An abiotic \textit{spin-up} for a 18x18 'water world' configuration (no land) with 16 vertical levels.
                                
        \end{compactenum}

%---------------------------------------------------------------------------------------------------------------------------------

\subsubsection{18x18x8 'ridge world'}\label{EXAMPLE.rwlla.SPIN}

This configuration employs a low resolution grid with an idealized super-continent.

\noindent \textbf{Physics configuration}: GOLDSTEIN ocean + sea-ice + EMBM atmosphere modules with seasonal insolation forcing.

\noindent \textbf{Biogeochemistry configuration}: \textbf{None}.

\noindent \textbf{Base-config:} The \textit{base-config} file is:
\vspace{-10pt}\begin{verbatim}cgenie.eb_go_gs_ac_bg._rwlma.NONE\end{verbatim}\vspace{-10pt}
This differs from the standard modern configuration in having:
        \begin{compactitem}
        \item An idealized single super-continent on an 18x18 grid and with 8 vertical levels in the ocean (topo: '\texttt{\_rwlla}').
                \item No biogeochemical tracers are selected: 
\texttt{GOLDSTEINNTRACSOPTS='\$(DEFINE)GOLDSTEINNTRACS=2'}
\item The only differences compared to the configuration of Ridgwell et al. [2007] are then:
        \vspace{-5pt}\begin{verbatim}
# minus 1 PSU for an ice-free world
go_saln0=33.9
# sea-ice eddy diffusivity
gs_11=1000.000
# fractional sea-ce coverage threshold for preventing advection
gs_par_sica_thresh=0.9
# set seasonal cycle
ea_dosc=.true.
go_dosc=.true.
gs_dosc=.true.
                \end{verbatim}\vspace{-5pt}
                which should be self-explanatory.
        \end{compactitem}

\noindent \textbf{User-config:} The associated \textit{user-config} is:
\vspace{-10pt}\begin{verbatim}EXAMPLE._rwlla.SPIN\end{verbatim}\vspace{-10pt}
and features the following notable parameters:

\begin{compactitem}

        \item \texttt{ --- CLIMATE ---}
        Firstly, climate is set in this example for a Neoproterozoic-like equatable climate, i.e. a little ice at the poles and hence a little like the present-day.
        \vspace{-5pt}\begin{verbatim}
# solar constant
# NOTE: modern S0 is 1368 W m-2
#       for a ca. 720 Ma time-interval (Sturtian onset), S0 will be 1286.9
#       for a ca. 635 Ma time-interval (Marinoan onset), S0 will be 1295.4
ma_genie_solar_constant=1295.4
# scaling for atmospheric CO2 radiative forcing, relative to 278 ppm
ea_radfor_scl_co2=10.0
                \end{verbatim}\vspace{-5pt}

        \item \texttt{ --- DATA SAVING ---}
        \vspace{-5pt}\begin{verbatim}
        # date saving & reporting options
        bg_par_data_save_level=9
        bg_ctrl_debug_lvl0=.true.
        ma_debug_loop=2
                \end{verbatim}\vspace{-5pt}
                ... sets some basic output.
                
                \item \texttt{--- FORCINGS ---}
                \vspace{-5pt}\begin{verbatim}
# use internal wind-speed
bg_ctrl_force_windspeed=.false.
                \end{verbatim}\vspace{-5pt}
                ... which is sort of a bit redundant considering that there are no biogeochemical gases to exchange across the air-sea interface ...
                \vspace{-5pt}\begin{verbatim}
# add a geothermal heat flux (W m-2)
bg_ctrl_force_GOLDSTEInTS=.TRUE.
bg_par_Fgeothermal=100.0E-3
                \end{verbatim}\vspace{-5pt}
                sets a geothermal heat flux (in case fo doing snowball Earth type experiments).

        \end{compactitem}

\noindent \textbf{Execution}: Command-line launching of the model experiment for a 10000 year integration:
\vspace{-10pt}\begin{verbatim}./runmuffin.sh cgenie.eb_go_gs_ac_bg._rwlla.NONE /
EXAMPLE._rwlla.SPIN 10000\end{verbatim}\vspace{-10pt}

\noindent \textbf{Ideas for further development:} 
Run from a warmer climate state to avoid feedback between ice and climate under orbital forcing, either by increasing the value of the solar constant (\texttt{ma\_genie\_solar\_constant}) or increasing the value of the specific radiative forcing (\texttt{ea\_radfor\_scl\_co2}).

\noindent \textbf{Relevant HOW-TO:} 

%---------------------------------------------------------------------------------------------------------------------------------

\subsubsection{18x18x16 'ridge world'}\label{EXAMPLE.rwlma.SPIN}

BLAH

%---------------------------------------------------------------------------------------------------------------------------------

\subsubsection{18x18x8 'water world'}\label{EXAMPLE.wwlla.SPIN}

BLAH

%---------------------------------------------------------------------------------------------------------------------------------

\subsubsection{18x18x16 'water world'}\label{EXAMPLE.wwlma.SPIN}

BLAH
        
%---------------------------------------------------------------------------------------------------------------------------------
%---------------------------------------------------------------------------------------------------------------------------------

\subsection{Idealized low resolution experiments -- orbital variations}

The following Examples comprise a highly idealized continental configuration at low (18x18) resolution (for the greatest practical computational efficiency):

        \begin{compactenum}
                        
                \item \texttt{EXAMPLE.\_rwlla.abio\_orbits}: An abiotic 18x18 'ridge world' configuration forcing with 1 Myr of orbital (insolation) variations.
                        
                \item \texttt{EXAMPLE.\_rwlla.bio\_orbits}: A biotic (with basic ocean carbon cycle) 18x18 'ridge world' configuration forcing with 1 Myr of orbital (insolation) variations.
                                
        \end{compactenum}

%---------------------------------------------------------------------------------------------------------------------------------

\subsubsection{Orbital variations -- abiotic ocean}\label{EXAMPLE.rwlla.abioorbits}

This example uses a low resolution idealized continental configuration following \texttt{EXAMPLE.\_rwlla.SPIN}, forcing with orbitally-varying seasonal insolation.

\noindent \textbf{Physics configuration}: GOLDSTEIN ocean + sea-ice + EMBM atmosphere modules with seasonal insolation forcing.

\noindent \textbf{Biogeochemistry configuration}: \textbf{None}.

\noindent \textbf{Base-config} The \textit{base-config} file is:
\vspace{-10pt}\begin{verbatim}cgenie.eb_go_gs_ac_bg._rwlla.NONE\end{verbatim}\vspace{-10pt}

\noindent \textbf{User-config:} The associated \textit{user-config} is:
\vspace{-10pt}\begin{verbatim}EXAMPLE._rwlla.abio_orbits\end{verbatim}\vspace{-10pt}
and features the following orbitally-related parameters:

\begin{compactitem}

        \item \texttt{ --- CLIMATE ---}
        Firstly, climate is set in this example for a Neoproterozoic-like equatable climate, i.e. a little ice at the poles and hence a little like the present-day.
        \vspace{-5pt}\begin{verbatim}
# solar constant
# NOTE: modern S0 is 1368 W m-2
#       for a ca. 720 Ma time-interval (Sturtian onset), S0 will be 1286.9
#       for a ca. 635 Ma time-interval (Marinoan onset), S0 will be 1295.4
ma_genie_solar_constant=1295.4
# scaling for atmospheric CO2 radiative forcing, relative to 278 ppm
ea_radfor_scl_co2=10.0
                \end{verbatim}\vspace{-5pt}

        \item \texttt{ --- DATA SAVING ---}
        \vspace{-5pt}\begin{verbatim}
        # date saving & reporting options
        bg_par_data_save_level=9
        bg_ctrl_debug_lvl0=.true.
        ma_debug_loop=0
                \end{verbatim}\vspace{-5pt}
                ... sets some basic output. The last parameter is important in the context of extremely long runs as it prevents year-by-year output.
                \vspace{-5pt}\begin{verbatim}
# save frequency
bg_par_infile_sig_name='save_timeseries_EVERY000100.dat'
                \end{verbatim}\vspace{-5pt}
                ... specifies that (mostly annual averaged) time-series output is saved every 100 years. Time-slices (parameter: \texttt{bg\_par\_infile\_slice\_name}) remain at their default frequency\footnote{See: Section 5.2 of the User-manual.}.
                \\ Then, to extract some illustrative latitude/season insolation output (as time-series)\footnote{See: Section 5.3 of the User-manual.}: 
                \vspace{-5pt}\begin{verbatim}
                # representative NH latitude and season to sample orbital variation of insolation
bg_par_t_sig_count_N=12
bg_par_sig_j_N=17
# SH latitude/season
bg_par_t_sig_count_S=12
bg_par_sig_j_S=2
                \end{verbatim}\vspace{-5pt}
                
                \item \texttt{--- FORCINGS ---}
                \vspace{-5pt}\begin{verbatim}
# ORBITS!!!
# Call orbit_radfor
ea_38="y"
# Specify the type of orbital forcing default (0), time-varying (1),
alternative config (2)
ea_39=1
# Number of data points in orbits file
ea_40=1001
# Interval between data points in goldstein time steps
ea_41=48000
# filename for orbital parameters (must be in genie-embm/data/input)
ea_42="orbits_La2004_1Myr.dat"
                \end{verbatim}\vspace{-5pt}
These are mostly self-explanatory. In the low resolution configuration, GOLDSTEIn takes 48 time-steps per year when using the \texttt{runmuffin.sh} script (this number is reported at experiment start). For an every-1000 years orbital interval in \texttt{orbits\_La2004\_1Myr.dat}, the number of GOLDSTEIn time-steps between orbital time-points is 1000x48 = 48000.

        \end{compactitem}

\noindent \textbf{Execution}: Command-line launching of the model experiment for a 1000000 year integration:
\vspace{-10pt}\begin{verbatim}./runmuffin.sh cgenie.eb_go_gs_ac_bg._rwlla.NONE /
EXAMPLE._rwlla.abio_orbits 1000000\end{verbatim}\vspace{-10pt}

\noindent \textbf{Ideas for further development:} 
Run from a warmer climate state to avoid feedback between ice and climate under orbital forcing, either by increasing the value of the solar constant (\texttt{ma\_genie\_solar\_constant}) or increasing the value of the specific radiative forcing (\texttt{ea\_radfor\_scl\_co2}).

\noindent \textbf{Relevant HOW-TO:} 

%---------------------------------------------------------------------------------------------------------------------------------

\subsubsection{Orbital variations -- basic ocean carbon cycle}\label{EXAMPLE.rwlla.bioorbits}

This example uses a low resolution idealized continental configuration following \texttt{EXAMPLE.\_rwlla.SPIN}, forcing with orbitally-varying seasonal insolation.

\noindent \textbf{Physics configuration}: GOLDSTEIN ocean + sea-ice + EMBM atmosphere modules with seasonal insolation forcing.

\noindent \textbf{Biogeochemistry configuration}: Basic ocean (and atmosphere) carbon cycle as described \textit{Ridgwell et al.} [2007]. Atmospheric restoring of \textit{p}CO2 (+ $\delta^{13}$C).

\noindent \textbf{Base-config} The \textit{base-config} file is:
\vspace{-10pt}\begin{verbatim}cgenie.eb_go_gs_ac_bg._rwlla.BASES\end{verbatim}\vspace{-10pt}

\noindent \textbf{User-config:} The associated \textit{user-config} is:
\vspace{-10pt}\begin{verbatim}EXAMPLE._rwlla.bio_orbits\end{verbatim}\vspace{-10pt}
The only differences between this config and \texttt{EXAMPLE.\_rwlla.abio\_orbits}, is the introduction of a marein carbon cycling following \textit{Ridgwell et al.} [2007]

\noindent \textbf{Execution}: Command-line launching of the model experiment for a 1000000 year integration:
\vspace{-10pt}\begin{verbatim}./runmuffin.sh cgenie.eb_go_gs_ac_bg._rwlla.BASES /
EXAMPLE._rwlla.bio_orbits 1000000\end{verbatim}\vspace{-10pt}

\noindent \textbf{Ideas for further development:} 
Remove the continuous restoring of textit{p}CO2 (+ $\delta^{13}$C).

\noindent \textbf{Relevant HOW-TO:} 

%---------------------------------------------------------------------------------------------------------------------------------
%---------------------------------------------------------------------------------------------------------------------------------

\subsection{Idealized low resolution experiments -- d7Li and d44Ca}

The following Examples illustrate the application of $\delta^{d}$7Li and $\delta^{d}$44Ca tracers:

        \begin{compactenum}
                        
                \item \texttt{EXAMPLE.\_rwlma.snowball.SPIN1}: Stage 1 \textit{spin-up} -- to set ocean chemistry, using an abiotic and idealized 18x18x16 'ridge world' configuration.
                        
                \item \texttt{EXAMPLE.\_rwlma.snowball.SPIN2gl}: Stage 2 \textit{spin-up} -- to spin up the isotope systems, using an abiotic and idealized 18x18x16 'ridge world' configuration.                 
                                
        \end{compactenum}

%---------------------------------------------------------------------------------------------------------------------------------

\subsubsection{ocean bulk geochemical \textit{spin-up}}\label{EXAMPLE.rwlma.snowball.SPIN1}

Stage 1 \textit{spin-up} -- to set ocean chemistry, using an abiotic and idealized 18x18x16 'ridge world' configuration.

\noindent \textbf{Physics configuration}: Idealized 18x18x16 'ridge world' configuration based on \texttt{EXAMPLE.\_rwlma.SPIN}. Seasonal insolation forcing.

\noindent \textbf{Biogeochemistry configuration}: No biological pump in the ocean. Assumption of shallow CaCO3 burial via a non biologically-driven process (i.e. requiring an elevated saturation state). Includes sediment (\texttt{SEDGEM}) and weathering (\texttt{ROKGEM}) modules.

\noindent \textbf{Base-config} The \textit{base-config} is:
\vspace{-10pt}\begin{verbatim}cgenie.eb_go_gs_ac_bg_sg_rg_gl._rwlma.BASESLiCa\end{verbatim}\vspace{-10pt}
and is identical to \texttt{cgenie.eb\_go\_gs\_ac\_bg\_sg\_rg\_gl.\_rwlma.NONE} except with the addition of carbon cycle plus Li and Ca isotope tracers.

\noindent \textbf{User-config} The \textit{user-config} is:
\vspace{-10pt}\begin{verbatim}EXAMPLE.\_rwlma.snowball.SPIN1\end{verbatim}\vspace{-10pt}
and features the following notable differences compared to the equivalent configuration (\texttt{EXAMPLE.\_rwlma.SPIN}):

\begin{compactitem}
        
        \item \texttt{--- CLIMATE ---}
        Firstly, the basic climate boundary conditions are set:
                        \vspace{-5pt}\begin{verbatim}
# NOTE: modern S0 is 1368 W m-2
#       for a ca. 720 Ma time-interval (Sturtian onset), S0 will be 1286.9
#       for a ca. 635 Ma time-interval (Marinoan onset), S0 will be 1295.4
ma_genie_solar_constant=1295.4
# set climate feedback
ea_36=y
                \end{verbatim}\vspace{-5pt}
                (in conjunction with the atmospheric CO2 restoring forcing, below)
                
        \item \texttt{--- SEDIMENTS ---}
        Sediments are set at a matching 18x18 grid resolution with the following grid options:
                        \vspace{-5pt}\begin{verbatim}
SEDGEMNLONSOPTS='$(DEFINE)SEDGEMNLONS=18'
SEDGEMNLATSOPTS='$(DEFINE)SEDGEMNLATS=18'
sg_par_sed_Dmax_neritic=176.0
sg_par_sed_topo_D_name="_rwlma.depth.18x18x16"
sg_par_sed_reef_mask_name="_rwlma.reefmask.18x18x16"
sg_par_sedcore_save_mask_name="_rwlma.sedcoremask.18x18x16"
                \end{verbatim}\vspace{-5pt}
        No diagenetic options are set:
                        \vspace{-5pt}\begin{verbatim}
sg_par_sed_diagen_CaCO3opt="NONE"
                \end{verbatim}\vspace{-5pt}
                To the sediments -- a moderately elevated (compared to modern open ocean) detrital flux is applied:
                        \vspace{-5pt}\begin{verbatim}
# global modern dust: 4483 Tg yr-1, divided by the Earth's surface area (5.1E14 m2), 
and then from: g m-2 yr-1 to g cm-2 kyr-1 = 0.88 g cm-2 kyr-1
sg_par_sed_fdet=0.88
                \end{verbatim}\vspace{-5pt}
                Carbonate depositional 'reef' environments are configured by the following settings
                \\ (see \texttt{EXAMPLE.p0251b.PO4.SPIN0} for more explanation).
                        \vspace{-5pt}\begin{verbatim}
sg_ctrl_sed_neritic_reef_force=.TRUE.
sg_par_sed_Dmax_neritic=0.0
                \end{verbatim}\vspace{-5pt}
                For reefal carbonate deposition, a minimum saturation state of 10.0 is set and a scaling value set (which will be optimized subsequently):
                        \vspace{-5pt}\begin{verbatim}
# set and scale abiotic precip
sg_par_sed_CaCO3_abioticohm_min=10.0
sg_par_sed_CaCO3precip_sf=3.402890846505764e-007
                \end{verbatim}\vspace{-5pt}
                Finally -- a couple of tidy-up options adjusting how the sediment columns are saved:
                        \vspace{-5pt}\begin{verbatim}
# set zero porosity
sg_par_sed_poros_CaCO3_reef=0.0
# increase number of internal array sedcore layers
sg_par_n_sedcore_tot_perky=20
                \end{verbatim}\vspace{-5pt}
                
        \item \texttt{--- WEATHERING ---}
        (Standard phase 1 \textit{spinup} settings.)
        
        \item \texttt{--- GEOCHEM ACCELERATION ---}
        'off', for whcih the key settings are:
                        \vspace{-5pt}\begin{verbatim}
ma_gem_notyr_min=9999999
ma_gem_notyr_max=9999999
ma_gem_yr_min=1
ma_gem_yr_max=1
ma_gem_dyr=0
                \end{verbatim}\vspace{-5pt}
        
        \item \texttt{--- FORCINGS ---}
        The important setting here enables a specified atmospheric CO2 concentration to be maintained, whilst a specified carbonate saturation at the ocean surface is simultaneously achieved. The \textit{forcing} that will do this is:
                        \vspace{-5pt}\begin{verbatim}
bg_par_forcing_name="pyyyyz.RpCO2_Rp13CO2_FRALK_FDIC_F13DIC_FCa"
                \end{verbatim}\vspace{-5pt}
                in which the atmosphere is restored at the same time as fluxes of alkalinity, DIC (and 13C), and Ca are added in order to match a particular saturation target.
                These fluxes are scaled as so:
                        \vspace{-5pt}\begin{verbatim}
bg_par_ocn_force_scale_val_3=1000.0E12
bg_par_ocn_force_scale_val_4=0.0
bg_par_ocn_force_scale_val_12=2000.0E12
bg_par_ocn_force_scale_val_35=1000.0E12
bg_par_ocn_force_scale_val_76=0.0
                \end{verbatim}\vspace{-5pt}
                These tracers are, in order: DIC, d13C of DIC, ALK, Ca, and d44Ca.
                As per normal, the atmospheric restored composition scaled:
                        \vspace{-5pt}\begin{verbatim}
bg_par_atm_force_scale_val_3=2780.0E-6
bg_par_atm_force_scale_val_4=-6.5
                \end{verbatim}\vspace{-5pt}
                Here, taking a value of  x10 PAL CO2 (and modern d13C).
                The all-important saturation value is set:
                        \vspace{-5pt}\begin{verbatim}
bg_par_force_invert_ohmega=10.0
                \end{verbatim}\vspace{-5pt}

        The wind-speed for the purposes of air-sea gas exchange is specified as being derived form the wind-stress field, and the gas transfer coefficient scaled appropriately (requiring either an earlier, initial experiment being run in order to determine the scaling needed):
                        \vspace{-5pt}\begin{verbatim}
bg_ctrl_force_windspeed=.false.
#re-scale gas transfer coeff to give ~0.058 mol m-2 yr-1 uatm-1 
global mean air-sea coefficient (original: 0.310)
bg_par_gastransfer_a=0.715813093980993
                \end{verbatim}\vspace{-5pt}
        Lastly, a geothermal heat flux is specified (so as to limit the sea-ice buildup possible during a snowball state).
                        \vspace{-5pt}\begin{verbatim}
bg_ctrl_force_GOLDSTEInTS=.TRUE.
bg_par_Fgeothermal=100.0E-3
                \end{verbatim}\vspace{-5pt}
        
        \item \texttt{--- MISC ---}
        Firstly, under this category, are some guesstimated initial ocean and atmosphere geochemical settings.
        Secondly, and more importantly, are all the parameters controlling the Li and Ca cycles and associated isotopic fractionation and compositions. For reasons of space here -- refer to the comments in the \textit{user-config} file as to what the settings are and where the values come from.

\end{compactitem}

\noindent \textbf{Pre-requisites:} NONE.

\noindent \textbf{Execution}: Command-line launching:
\vspace{-10pt}\begin{verbatim}./runmuffin.sh cgenie.eb_go_gs_ac_bg_sg_rg_gl._rwlma.BASESLiCa /
EXAMPLE._rwlma.snowball.SPIN1 50000\end{verbatim}\vspace{-10pt}
for a 50,000 year \textit{spin-up}.

\noindent \textbf{Ideas for further development:} 
This config could be adjusted to be more modern-like by changing the atmospheric CO2 restored value together with the solar constant. To reduce the ocean surface saturation target, change the parameter \texttt{bg\_par\_force\_invert\_ohmega}. BUT, the threshold for carbonate precipitation in shallow water environments will also then have to be reduced, i.e. \textit{sg\_par\_sed\_CaCO3\_abioticohm\_min}. (Future Examples will cover a more realistic modern-like low resolution configuration, with for example, deep-sea sediments.)

\noindent \textbf{Relevant HOW-TO}:

%---------------------------------------------------------------------------------------------------------------------------------

\subsubsection{ocean isotopic \textit{spin-up}}\label{EXAMPLE.rwlma.snowball.SPIN2gl}

Stage 1 \textit{spin-up} -- to set ocean isotopic composition, using an abiotic and idealized 18x18x16 'ridge world' configuration.

\noindent \textbf{Physics configuration}: Idealized 18x18x16 'ridge world' configuration based on \texttt{EXAMPLE.\_rwlma.SPIN}. Seasonal insolation forcing.

\noindent \textbf{Biogeochemistry configuration}: No biological pump in the ocean. Assumption of shallow CaCO$\_{3}$ burial via a non biologically-driven process (i.e. requiring an elevated saturation state). Includes sediment (\texttt{SEDGEM}), weathering (\texttt{ROKGEM}), and accelerated global mass budget (\texttt{GEMlite}) modules.

\noindent \textbf{Base-config} The \textit{base-config} is:
\vspace{-10pt}\begin{verbatim}cgenie.eb_go_gs_ac_bg_sg_rg_gl._rwlma.BASESLiCa\end{verbatim}\vspace{-10pt}

\noindent \textbf{User-config} The \textit{user-config} is:
\vspace{-10pt}\begin{verbatim}EXAMPLE.\_rwlma.snowball.SPIN2gl\end{verbatim}\vspace{-10pt}
The main difference compared to the 1st stage \textit{spin-up}, as in all these pairs of with-sediment configurations, is the addition of weathering feedback and an open system.

\begin{compactitem}
        
        \item \texttt{--- WEATHERING ---}
        As per 'normal', an open system configuration is now specified. the weathering short-cut (see: \textit{Colbourn et al.} [2013]) is not strictly necessary to be specified, as it is 'on' by default (but subsequent snowball experiments will turn it 'off', hence it makes it less easy to make mistakes if explicitly defined here):
                        \vspace{-5pt}\begin{verbatim}
# set a 'CLOSED' system
bg_ctrl_force_sed_closedsystem=.false.
# TURN ON WEATHERING SHORTCUT
rg_opt_short_circuit_atm=.true.
                \end{verbatim}\vspace{-5pt}
                Weathering feedbcaks are specified as per 'usual', with a reference temperature set as per 'usual'. the only note-worthy change in this respect is the addition of run-off feedback (see: \textit{Colbourn et al.} [2013]):
                        \vspace{-5pt}\begin{verbatim}
# set CaCO3_weathering-temperature feedback
rg_opt_weather_T_Ca=.TRUE.
# set CaSiO3_weathering-temperature feedback
rg_opt_weather_T_Si=.TRUE.
# set CaCO3_weathering-runoff feedback
rg_opt_weather_R_Ca=.TRUE.
# set CaSiO3_weathering-runoff feedback
rg_opt_weather_R_Si=.TRUE.
# weathering reference mean global land surface temperature (C)
rg_par_ref_T0=13.454777
                \end{verbatim}\vspace{-5pt}
        Here, 2:3 ratio of silicate:carbonate weathering is assumed (compared to the more normal 1:1 in most previous configurations) and a global CaCO$_{3}$ burial rate of 15 Tmol yr$^{-1}$ is assumed. The baseline carbonate weathering rate can then safely be set at 9 Tmol yr$^{-1}$. Actually, becasue in this continental configurtion, land area is zonally-symetrical, the seasonal cycle impacts no bias and total baseline silicate weathering can be set the same as CO2 out-gassing -- 9 Tmol yr$^{-1}$. Except that there is a hydrothermal exchange of Mg for Ca of 2 Tmol yr$^{-1}$. So a 0.333 fraction for Mg in silicate rocks is set. the carbon isotopic composition of weathered carbonates is set to balance global burial as per usual ...
                        \vspace{-5pt}\begin{verbatim}
#CO2 outgassing rate (mol C yr-1)
rg_par_outgas_CO2=6.0E+12
# global silicate weathering rate (mol Ca2+ yr-1)
rg_par_weather_CaSiO3=6.0E+12
rg_par_weather_CaSiO3_fracMg=0.333
# global carbonate weathering rate (mol Ca2+ yr-1)
rg_par_weather_CaCO3=9.0E+12
# d13C -- mean global burial == 2.782 o/oo
rg_par_outgas_CO2_d13C=-6.0
rg_par_weather_CaCO3_d13C=8.64
                \end{verbatim}\vspace{-5pt}
        
        \item \texttt{--- GEOCHEM ACCELERATION ---}
        GEMlite is not set for an aggressive acceleration in order to spin up the isotopic systems:
                        \vspace{-5pt}\begin{verbatim}
gl_ctrl_update_pCO2=.true.
ma_gem_notyr_min=1
ma_gem_notyr_max=1
ma_gem_yr_min=999
ma_gem_yr_max=999
ma_gem_dyr=0
ma_gem_adapt_auto=.false.
                \end{verbatim}\vspace{-5pt}
                Here -- 999 years of accelerated to 1 year of non-accelerated (in a fixed ratio).
        
\end{compactitem}

\noindent \textbf{Pre-requisites:} 
\texttt{EXAMPLE.\_rwlma.snowball.SPIN1}

\noindent \textbf{Execution}: Command-line launching:
\vspace{-10pt}\begin{verbatim}./runmuffin.sh cgenie.eb_go_gs_ac_bg_sg_rg_gl._rwlma.BASESLiCa /
EXAMPLE._rwlma.snowball.SPIN2gl 10000000 EXAMPLE._rwlma.snowball.SPIN1\end{verbatim}\vspace{-10pt}
for a 10,000,000 year \textit{spin-up}.

\noindent \textbf{Ideas for further development:} 

\noindent \textbf{Relevant HOW-TO}:

%---------------------------------------------------------------------------------------------------------------------------------
%--- Example cGENIE configurations and spin-ups -- modern -------------------------------------------
%---------------------------------------------------------------------------------------------------------------------------------

\newpage
\section{Example configurations \textit{spin-ups} -- modern}\label{example_spinups_modern}

%---------------------------------------------------------------------------------------------------------------------------------
%---------------------------------------------------------------------------------------------------------------------------------

In addition to published configurations, a variety of additional different model configurations and associated \textit{spin-ups} design are provided here for reference and for use as helpful starting-points (templates) in creating model experiments.

%---------------------------------------------------------------------------------------------------------------------------------
%---------------------------------------------------------------------------------------------------------------------------------

\subsection{Modern 36x36x16 configuration}\label{EXAMPLE.worjh2.PO4.SPIN}

A basic modern ocean-only configuration. Essentially, the same as Cao et al. [2009] except without the additional anthropogenic tracers and using 96 time-steps per year in the ocean (rather than 100).

\noindent \textbf{Physics configuration:} GOLDSTEIN ocean + sea-ice + EMBM atmosphere modules. Climatology is seasonal and identical to that described in \textit{Cao et al.} [2009] (and references therein).

\noindent \textbf{Biogeochemistry configuration:} Basic ocean (and atmosphere) carbon cycle as described \textit{Cao et al.} [2009]. Atmospheric restoring of CO2 (plus d13C).

\noindent \textbf{Base-config:} The \textit{base-config} file is named:
\vspace{-10pt}\begin{verbatim}cgenie.eb_go_gs_ac_bg.worjh2.BASE\end{verbatim}\vspace{-10pt}

\noindent \textbf{User-config:} The \textit{user-config} file is named\footnote{The model experiment will be assigned the same name as this when using \texttt{runmuffin.sh}.}:
\vspace{-10pt}\begin{verbatim}EXAMPLE.worjh2.PO4.SPIN\end{verbatim}\vspace{-10pt}

\noindent \textbf{Execution:} A command-line launching of the model experiment (10000 years integration) would be:
\vspace{-10pt}\begin{verbatim}./runmuffin.sh cgenie.eb_go_gs_ac_bg.worjh2.BASE /
EXAMPLE.worjh2.PO4.SPIN 10000\end{verbatim}\vspace{-5pt}

\noindent \textbf{Ideas for further development:} 

\noindent \textbf{Relevant HOW-TO}: 

%---------------------------------------------------------------------------------------------------------------------------------
%---------------------------------------------------------------------------------------------------------------------------------

\subsection{Modern 36x36x16 configuration with an iron cycle}\label{EXAMPLE.worjh2.PO4Fe.SPIN}

This \textit{spin-up} is as per \texttt{EXAMPLE.worjh2.PO4.SPIN} except it is configured with an Fe cycle.

\noindent \textbf{Physics configuration}: GOLDSTEIN ocean + sea-ice + EMBM atmosphere modules. Climatology is seasonal and near identical to that described in \textit{Cao et al.} [2009] (and references therein), the main difference being due to the 96-per-year rather than 100-per-year time-stepping in the ocean.

\noindent \textbf{Biogeochemistry configuration}: The ocean carbon cycle includes an iron cycle and co-limitation of biological productivity and is as described in \textit{Ridgwell and De'Ath} [in prep]. During the spin-up, the ocean is forced into equilibrium with Preindustrial atmospheric concentrations of: CO2 and O2, plus the d13C of CO2, via a restoring \textit{forcing} of atmospheric composition.

\noindent \textbf{Base-config} The \textit{base-config} file is named:
\vspace{-10pt}\begin{verbatim}cgenie.eb_go_gs_ac_bg.worjh2.BASEFe\end{verbatim}\vspace{-10pt} which defines the use (and initial values) of the following tracers\footnote{See the \texttt{cGENIE} \textit{Namelist} table for a description of the tracer numbering scheme.}:

\begin{compactenum}
        \item Atmospheric (gaseous) tracers (\texttt{gm\_atm\_select\_xx}):
        \\\texttt{ia\_pCO2} (xx=3), \texttt{ia\_pCO2\_13C} (xx=4), \texttt{ia\_pO2} (xx=6)
                (in addition to atmospheric temperature and humidity)
        \item Ocean (dissolved) tracers (\texttt{gm\_ocn\_select\_xx}):
        \\\texttt{io\_DIC} (xx=3), \texttt{io\_DIC\_13C} (xx=4), \texttt{io\_PO4} (xx=8), \texttt{io\_Fe} (xx=9), \texttt{io\_O2} (xx=10), \texttt{io\_ALK} (xx=12), io\_DOM\_C (xx=15), \texttt{io\_DOM\_C\_13C} (xx=16), \texttt{io\_DOM\_P} (xx=20), \texttt{io\_DOM\_Fe} (xx=22), \texttt{io\_FeL} (xx=23), \texttt{io\_L} (xx=24)
                (in addition to ocean temperature and salinity)
        \item The corresponding sedimentary (solid) tracers (\texttt{gm\_sed\_select\_xx}) are also selected:
        \\\texttt{is\_POC} (xx=3), \texttt{is\_POC\_13C} (xx=4), \texttt{is\_POP} (xx=8), \texttt{is\_POFe} (xx=10), \texttt{is\_POM\_Fe} (xx=13), \texttt{is\_CaCO3} (xx=14), \texttt{is\_CaCO3\_13C} (xx=15), \texttt{is\_CaCO3\_Fe} (xx=21), \texttt{is\_det} (xx=22), \texttt{is\_det\_Fe} (xx=25), \texttt{is\_ash} (xx=32), \texttt{is\_POC\_frac2} (xx=33), \texttt{is\_CaCO3\_frac2} (xx=34), \texttt{is\_CaCO3\_age} (xx=36)
\end{compactenum}

\noindent \textbf{User-config} The \textit{user-config} file is named:
\vspace{-10pt}\begin{verbatim}EXAMPLE.worjh2.PO4Fe.SPIN\end{verbatim}\vspace{-10pt} and contains the following parameter specifications\footnote{Mostly (but not always) these represent changes from the default and thus it would be possible to conduct an identical experiment with slightly fewer namelist specification. Some of the (mainly biological) namelist values are re-defined (identically) for completeness.}:

\begin{compactitem}
        
        \item \texttt{--- BIOLOGICAL NEW PRODUCTION ---}
        \\ \texttt{bg\_par\_bio\_prodopt='bio\_PFe'} == sets the P+Fe nutrient co-limitation 'biological' scheme. See: \textit{Ridgwell and De'Ath} [in prep] for a description of this (plus the other 3 lister parameters and their values).
        
        \item \texttt{--- ORGANIC MATTER EXPORT RATIOS ---}
        \\ Parameters as described in \textit{Ridgwell and De'Ath} [in prep].
        
        \item \texttt{--- INORGANIC MATTER EXPORT RATIOS ---}
        \\ Parameters as defined in \textit{Cao et al.} [2009] and based on the parameterization described in \textit{Ridgwell et al.} [2007a,b].
        
        \item \texttt{--- REMINERALIZATION ---}
        \\ Parameters mostly as defined in \textit{Cao et al.} [2009] and based on the parameterizations described in \textit{Ridgwell et al.} [2007a]., except:
        \\ The lifetime of DOM (\texttt{bg\_par\_bio\_remin\_DOMlifetime}), and 'initial fractional abundance of POC component' (\texttt{bg\_par\_bio\_remin\_POC\_frac2}) adopt parameter values as described in \textit{Ridgwell and De'Ath} [in prep].  
        
        \item \texttt{--- IRON ---}
        \\ Sets the Fe cycle, including:
        \begin{compactitem}
        \item   aeolian Fe solubility (\texttt{bg\_par\_det\_Fe\_sol})
                \item scavenging (\texttt{bg\_par\_scav\_Fe\_sf\_POC})
                \end{compactitem}
                \noindent See: \textit{Ridgwell and De'Ath} [in prep].
        
        \item \texttt{--- FORCINGS ---}
        \\ The \textit{forcing} applied is specified as \texttt{worjh2.FeMahowald2006\_RpCO2\_Rp13CO2} (the files of which live in the equivalently named subdirectory of \texttt{\~{}/cgenie.muffin/genie-forcings}). The \textit{forcing} prescribes fixed boundary conditions of atmospheric pCO2 and d13C, plus a dust flux following \textit{Mahowald et al.} [2006] and consists of:
        \begin{compactitem}
                \item Selection of forcings:
                \begin{compactenum}
                        \item  \texttt{configure\_forcings\_atm.dat} == Selection of restoring forcing\footnote{Time-constant for all \textit{restorings} set to 0.1 years.} of:
                        \\\texttt{ia\_pCO2}, \texttt{ia\_pCO2\_13C}
                        \item  \texttt{configure\_forcings\_ocn.dat} == No ocean tracer forcings.
                        \item  \texttt{configure\_forcings\_sed.dat} == Selection of a flux forcing of:
                        \\\texttt{is\_det}
                \end{compactenum}
                \item Spatial and temporal definition of forcings. All three selected forcings have a file containing time-dependent information associated with them\footnote{See: \textit{User manual}.}: \texttt{biogem\_force\_restore\_yyy\_xxx\_sig.dat}. In addition, the dust flux forcing has a 2D spatial pattern associated with it:
                \\ \texttt{biogem\_force\_flux\_sed\_det\_SUR.dat}.
                The parameter values at the end of this section simply scale atmospheric composition:
                \vspace{-5pt}\begin{verbatim}
                bg_par_atm_force_scale_val_3=278.0E-06
                bg_par_atm_force_scale_val_4=-6.5
                \end{verbatim}\vspace{-5pt}
        \end{compactitem}
        
\end{compactitem}

\noindent \textbf{Execution}: A command-line launching of the model experiment (10000 years integration) would be:
\vspace{-10pt}\begin{verbatim}./runmuffin.sh cgenie.eb_go_gs_ac_bg.worjh2.BASEFe /
EXAMPLE.worjh2.PO4Fe.SPIN 10000\end{verbatim}\vspace{-5pt}

\noindent \textbf{Ideas for further development:} 

\noindent \textbf{Relevant HOW-TO}: 

%---------------------------------------------------------------------------------------------------------------------------------
%---------------------------------------------------------------------------------------------------------------------------------

\subsection{Modern 36x36x16 configuration + Fe \& CH4 cycles}\label{EXAMPLE.worjh2.PO4FeCH4.SPIN}

This \textit{spin-up} is configured as per \texttt{EXAMPLE.worjh2.PO4Fe.SPIN} except it has an added CH4 cycle.

\noindent \textbf{Physics configuration}: GOLDSTEIN ocean + sea-ice + EMBM atmosphere modules.

\noindent \textbf{Biogeochemistry configuration}: Ocean carbon cycle including an iron cycle and co-limitation of biological productivity [\textit{Ridgwell and De'Ath}, in prep]. Atmospheric restoring of CO2 and CH4 (plus d13C of both).

\noindent \textbf{Base-config} The \textit{base-config} file is named:
\vspace{-10pt}\begin{verbatim}cgenie.eb_go_gs_ac_bg.worjh2.BASEFeCH4\end{verbatim}\vspace{-10pt}
and defines the use (and initial values) of the following tracers
\\ (in addition to those described for \texttt{cgenie.eb\_go\_gs\_ac\_bg.worjh2.BASEFe}):
\begin{compactenum}
        
        \item Atmospheric (gaseous) tracers (\texttt{gm\_atm\_select\_xx}):
        \\\texttt{ia\_pCH4} (xx=10), \texttt{ia\_pCH4\_13C} (xx=11)
        \item Ocean (dissolved) tracers (\texttt{gm\_ocn\_select\_xx}):
        \\\texttt{io\_CH4} (xx=25), \texttt{io\_CH4\_13C} (xx=26)
                
\end{compactenum}
By default, a zero concentration of CH4 (in ocean and atmosphere) are set, while the isotopic composition of both pCH4 (atmosphere) and CH4 (ocean, dissolved) is set to -60 per mil.

\noindent \textbf{User-config} The \textit{user-config} file is named:
\vspace{-10pt}\begin{verbatim}EXAMPLE.worjh2.PO4FeCH4.SPIN\end{verbatim}\vspace{-10pt}
and differs from the equivalent standard modern configuration in:
\begin{compactitem}
                
        \item \texttt{--- REMINERALIZATION ---}
        \\ An oxidation rate constant for CH4 in the ocean is prescribed:
        \vspace{-5pt}\begin{verbatim}bg_par_bio_remin_CH4rate=0.00004\end{verbatim}\vspace{-5pt}
        and has units of d-1.\footnote{Note that this particular value does not necessarily reflect any ocean reality ...}
        
        \item \texttt{--- FORCINGS ---}
        \\ The \textit{forcing} prescribes fixed boundary conditions of atmospheric pCO2 and d13C, PLUS fixed boundary conditions of pCH4 and d13C (of CH4), in addition to a surface ocean dust flux.
        The parameter values that follow simply scale atmospheric composition:
        \vspace{-5pt}\begin{verbatim}
        bg_par_forcing_name="worjh2.FeMahowald2006_RpCO2_Rp13CO2_RpCH4_Rp13CH4"
        bg_par_atm_force_scale_val_3=278.0E-06
        bg_par_atm_force_scale_val_4=-6.5
        bg_par_atm_force_scale_val_10=700.0E-9
        bg_par_atm_force_scale_val_11=-60.0
        \end{verbatim}\vspace{-5pt}
        Note that the atmospheric CH4 restoring concentration is specified here as preindustrial (ca. 700 ppb == 700.0E-9 atm).

\end{compactitem}

\noindent \textbf{Execution}: A command-line launching of the model experiment (10000 years integration) would be:
\vspace{-10pt}\begin{verbatim}./runmuffin.sh cgenie.eb_go_gs_ac_bg.worjh2.BASEFeCH4 /
EXAMPLE.worjh2.PO4FeCH4.SPIN 10000\end{verbatim}\vspace{-5pt}

\noindent \textbf{Ideas for further development:} 

\noindent \textbf{Relevant HOW-TO:}

%---------------------------------------------------------------------------------------------------------------------------------
%---------------------------------------------------------------------------------------------------------------------------------

\subsection{Modern 36x36x16 configuration + full CaCO3 cycle}\label{EXAMPLE.worjh2.Archeretal2009.SPIN}

The following examples use a modern configuration, with a basic (PO$_{4}$-only nutrient) based ocean carbon cycle together with carbonate (CaCO$_{3}$) sedimentation and burial and weathering input.
\\ There are 2 parts to the \textit{spin-up}, the purpose of which is:

        \begin{compactenum}
        
                \item The first stage of \textit{spin-up} creates an equilibrium climate and ocean circulation plus ocean carbonate chemistry (specifically; DIC), consistent with the prescribed value of atmospheric pCO2. It is also used to rapidly generate an equilibrium distribution of surface sediment CaCO3 composition and burial in the deep-sea from which a global weathering flux (to balance) can be diagnosed.
                
                \item The second stage of \textit{spin-up} is used to create a depth of well equilibrated sediment composition sufficient for interaction with ocean chemistry and CaO3 burn-down in massive carbon release experiments.
                                
        \end{compactenum}

%---------------------------------------------------------------------------------------------------------------------------------

\subsubsection{CLOSED CaCO3 cycle: 1st stage \textit{spin-up}}\label{EXAMPLE.worjh2.Archeretal2009.SPIN1}

This example uses a modern configuration continental configuration, with a basic (P-only) based ocean carbon cycle, but with deep-sea (CaCO3) sedimentation and burial and weathering input in a '\textit{closed system}'.

\noindent \textbf{Physics configuration}: GOLDSTEIN ocean + sea-ice + EMBM atmosphere modules with with seasonal insolation forcing.

\noindent \textbf{Biogeochemistry configuration}: The basic ocean (and atmosphere) carbon cycle as described \textit{Cao et al.} [2009] and \textit{Archer et al.} [2009]. Atmospheric restoring of pCO2 (+ d13C).

\noindent \textbf{Base-config} The \textit{base-config} file is:
\vspace{-10pt}\begin{verbatim}cgenie.eb_go_gs_ac_bg_sg_rg.worjh2.BASES\end{verbatim}\vspace{-10pt}

\noindent \textbf{User-config} 
\\ This \textit{user-config}:
\vspace{-10pt}\begin{verbatim}EXAMPLE.worjh2.Archeretal2009.SPIN1\end{verbatim}\vspace{-10pt}
contains the follow main features compared to ocean (and atmosphere) only carbon cycle configurations:
\begin{compactitem}
                \item \texttt{--- SEDIMENTS ---}
                \\ STUFF ...
                \item \texttt{--- FORCINGS ---}
                \\ MOAR STUFF ...
        \end{compactitem}

\noindent \textbf{Execution}: Command-line launching of the model experiment for a 20000 year integration:
\vspace{-10pt}\begin{verbatim}./runmuffin.sh cgenie.eb_go_gs_ac_bg_sg_rg.worjh2.BASES /
EXAMPLE.worjh2.Archeretal2009.SPIN1 20000\end{verbatim}\vspace{-10pt}

\noindent \textbf{Ideas for further development:} 

\noindent \textbf{Relevant HOW-TO:} 'Spin-up the full marine carbon cycle including sediments'

%---------------------------------------------------------------------------------------------------------------------------------

\subsubsection{CLOSED CaCO3 cycle: 1st stage \textit{spin-up}}\label{EXAMPLE.worjh2.Archeretal2009.SPIN2}

\noindent \textbf{Relevant HOW-TO}: 

%---------------------------------------------------------------------------------------------------------------------------------
%--- Example cGENIE configurations and spin-ups -- paleo ----------------------------------------------
%---------------------------------------------------------------------------------------------------------------------------------

\newpage
\section{Example configurations and \textit{spin-ups} -- paleo}\label{example_spinups_paleo}

%---------------------------------------------------------------------------------------------------------------------------------
%---------------------------------------------------------------------------------------------------------------------------------

\subsection{Eocene 36x36x16 configuration + CH4 cycle}\label{EXAMPLE.p0055c.PO4CH4.SPIN}

This example uses an early Eocene continental configuration, with a basic (P-only) based ocean carbon cycle but with global biogeochemical cycling of CH4 included.

\noindent \textbf{Physics configuration}: GOLDSTEIN ocean + sea-ice + EMBM atmosphere modules. Adjusted planetary albedo and solar constant. Adjusted continental configuration. Forcing with seasonal insolation (but annual averaged wind stress and winds). See: \textit{Ridgwell and Schmidt} [2010].

\noindent \textbf{Biogeochemistry configuration}: Basic ocean (and atmosphere) carbon cycle as described \textit{Cao et al.} [2009] but with modifications following \textit{Ridgwell and Schmidt} [2010] (and described below). Atmospheric restoring of CO2 and CH4 (plus d13C of both).

\noindent \textbf{Base-config:} \texttt{cgenie.eb\_go\_gs\_ac\_bg.p0055c.BASESCH4}

\noindent \textbf{User-config:} \texttt{EXAMPLE.p0055c.PO4CH4.SPIN} which differs from the equivalent standard Eocene configuration ([\textit{Ridgwell and Schmidt}, 2010]) as follows:
\begin{compactitem}
        \item \texttt{--- REMINERALIZATION ---}
        \\ An oxidation rate constant for CH4 in the ocean is prescribed:
\vspace{-5pt}\begin{verbatim}bg_par_bio_remin_CH4rate=0.00004\end{verbatim}\vspace{-5pt}
and has units of d-1.\footnote{Note that this particular value does not necessarily reflect any ocean reality ...}
                \item \texttt{--- FORCINGS ---}
        \\ The selected \textit{forcing} prescribes fixed boundary conditions of atmospheric pCO2 and d13C, PLUS pCH4 and d13C (of CH4):
\vspace{-5pt}\begin{verbatim}bg_par_forcing_name="pyyyyz.RpCO2_Rp13CO2_RpCH4_Rp13CH4"\end{verbatim}\vspace{-5pt}
        The normalized (unit) values contained in the forcing are then scaled:
        \vspace{-5pt}\begin{verbatim}
bg_par_atm_force_scale_val_3=834.0E-06
bg_par_atm_force_scale_val_4=-4.9
bg_par_atm_force_scale_val_10=3500.0E-9
bg_par_atm_force_scale_val_11=-60.0
                \end{verbatim}\vspace{-5pt}
to give x3 CO2 and approximately x5 CH4.
        \end{compactitem}

\noindent \textbf{Execution}: A command-line launching of the model experiment (10000 years integration) would be:
\vspace{-5pt}\begin{verbatim}./runmuffin.sh cgenie.eb_go_gs_ac_bg.p0055c.BASESCH4 /
EXAMPLE.p0055c.PO4CH4.SPIN 10000\end{verbatim}\vspace{-5pt}

\noindent \textbf{Ideas for further development:} Once the \textit{spin-up} is complete, the restoring of atmospheric pCH4 can be replaced by a prescribed flux that exactly balances oxidation in the ocean and atmosphere for an atmospheric partial pressure of 3500 uatm CH4. The flux implicit in the restoring forcing is reported in the BIOGEM time-series output: 
\vspace{-10pt}\begin{verbatim}biogem_series_diag_misc_specified_forcing_pCH4.res\end{verbatim}\vspace{-10pt}
The final (steady state) value can be set equal to a prescribed 'wetland' flux to the atmosphere of CH4 (+ d13C) to balance oxidation loss (and the restoring removed). The \textit{forcing} section of the \textit{user-config} would then look like:
        \vspace{-10pt}\begin{verbatim}
bg_par_atm_force_scale_val_3=834.0E-06
bg_par_atm_force_scale_val_4=-4.9
                \end{verbatim}\vspace{-10pt}
assuming that atmospheric pCO2 restoring is still required, plus:
        \vspace{-10pt}\begin{verbatim}
ac_par_atm_wetlands_FCH4=0.6206165E+14
ac_par_atm_wetlands_FCH4_d13C=-60.0
                \end{verbatim}\vspace{-10pt}
which prescribes  a steady flux of CH4 (+ 13C) to the atmosphere (as if from wetlands etc.).

\noindent \textbf{Relevant HOW-TO:} 'Determine the CH4 flux required to achieve a particular atmospheric pCH4 value'.

%---------------------------------------------------------------------------------------------------------------------------------
%---------------------------------------------------------------------------------------------------------------------------------

\subsection{Eocene 36x36x16 configuration + CH4 cycle + full CaCO3 cycle}\label{EXAMPLE.p0055c.PO4CH4_S36x36.SPIN}

The following examples are exactly analogous \textit{Panchuk et al.} [2013] except with the inclusion of a CH4 cycle. As such, only the first part of the \textit{spin-up} is detailed here.

%---------------------------------------------------------------------------------------------------------------------------------

\subsubsection{Eocene 36x36x16 configuration + CLOSED CaCO3 cycle}\label{EXAMPLE.p0055c.PO4CH4_S36x36.SPIN1}

First stage \textsl{spin-up} as per the \textit{Panchuk et al.} [2013] configuration of early Eocene marine CaCO$_{3}$ cycling but with a CH$_{4}$ cycle.

\noindent \textbf{Physics configuration}: GOLDSTEIN ocean + sea-ice + EMBM atmosphere modules with with seasonal insolation forcing. Adjusted: continental configuration, planetary albedo, solar constant, ocean salinity, annual averaged wind stress and winds.

\noindent \textbf{Biogeochemistry configuration}: Basic ocean (and atmosphere) carbon cycle with atmospheric restoring of \textit{p}CO$_{2}$ (+ $\delta^{13}$C) and of \textit{p}CH$_{4}$ (+ $\delta^{13}$C).

\noindent \textbf{Base-config:}
\vspace{-10pt}\begin{verbatim}cgenie.eb_go_gs_ac_bg_sg_rg.p0055c.BASESCH4\end{verbatim}\vspace{-10pt}

\noindent \textbf{User-config:} The associated \textit{user-config} is:
\vspace{-10pt}\begin{verbatim}EXAMPLE.p0055c.PO4CH4_S36x36.SPIN1\end{verbatim}\vspace{-10pt}
and has the following noteworthy differences compared to the \textit{Panchuk et al.} [2013] 1st stage \textit{spin-up}:
\begin{compactitem}
        \item \texttt{--- REMINERALIZATION ---}
        \\ An oxidation rate constant for CH4 in the ocean is prescribed:
\vspace{-5pt}\begin{verbatim}bg_par_bio_remin_CH4rate=0.00004\end{verbatim}\vspace{-5pt}
and has units of d-1.\footnote{Note that this particular value does not necessarily reflect any ocean reality ...}
                \item \texttt{--- FORCINGS ---}
        \\ The selected \textit{forcing} prescribes fixed boundary conditions of atmospheric pCO2 and d13C, PLUS pCH4 and d13C (of CH4):
\vspace{-5pt}\begin{verbatim}bg_par_forcing_name="pyyyyz.RpCO2_Rp13CO2_RpCH4_Rp13CH4"\end{verbatim}\vspace{-5pt}
        The normalized (unit) values contained in the forcing are then scaled:
        \vspace{-5pt}\begin{verbatim}
bg_par_atm_force_scale_val_3=834.0E-06
bg_par_atm_force_scale_val_4=-4.9
bg_par_atm_force_scale_val_10=3500.0E-9
bg_par_atm_force_scale_val_11=-60.0
                \end{verbatim}\vspace{-5pt}
to give x3 \textit{p}CO$_{2}$ and approximately x5 \textit{p}CH$_{4}$.
        \end{compactitem}

\noindent \textbf{Execution}: Command-line launching of the model experiment for a 20000 year integration:
\vspace{-5pt}\begin{verbatim}./runmuffin.sh cgenie.eb_go_gs_ac_bg_sg_rg.p0055c.BASESCH4 /
EXAMPLE.p0055c.PO4CH4_S36x36.SPIN1 20000\end{verbatim}

\noindent \textbf{Ideas for further development:} Opening the CH$_{4}$ cycle in the 2nd stage \textit{spin-up}. 
\\ See example \texttt{EXAMPLE.p0055c.PO4CH4.SPIN} above.

\noindent \textbf{Relevant HOW-TO:} 'Spin-up the full marine carbon cycle including sediments'; 'Determine the CH4 flux required to achieve a particular atmospheric pCH4 value'.

%---------------------------------------------------------------------------------------------------------------------------------
%---------------------------------------------------------------------------------------------------------------------------------

\subsection{Late Cenomanian 36x36x16 configuration; PO4-only}\label{EXAMPLE.p0093k.PO4.SPIN}

This example uses a Late Cenomanian continental configuration, with a basic (P-only) based ocean-only carbon cycle.

\noindent \textbf{Physics configuration}: GOLDSTEIN ocean + sea-ice + EMBM atmosphere modules. Climatology is seasonal and identical to that described in \textit{Monteiro et al.} [2012] (and references therein).

\noindent \textbf{Biogeochemistry configuration}: Basic ocean (and atmosphere) carbon cycle based on the PO4-only ocean carbon cycle of \textit{Cao et al.} [2009] rather than the P+N scheme of \textit{Monteiro et al.} [2012]. Atmospheric restoring of CO2 (plus d13C). SO4 and H2S tracers are included and hence questions of (extent, dsitribution, and intensity) water column euxinia can be explored.

\noindent \textbf{Base-config} The \textit{base-config} file is named:
\vspace{-10pt}\begin{verbatim}cgenie.eb_go_gs_ac_bg.p0093k.BASES\end{verbatim}\vspace{-10pt}

\noindent \textbf{User-config} The \textit{user-config} file is named:
\vspace{-10pt}\begin{verbatim}EXAMPLE.p0093k.PO4.SPIN\end{verbatim}\vspace{-10pt}

\noindent \textbf{Execution}: A command-line launching of the model experiment (10000 years integration) would be:
\vspace{-10pt}\begin{verbatim}./runmuffin.sh cgenie.eb_go_gs_ac_bg.p0093k.BASES /
EXAMPLE.p0093k.PO4.SPIN 10000\end{verbatim}\vspace{-5pt}

\noindent \textbf{Ideas for further development:} The global PO4 inventory can be increased in order to promote productivity and hence water column euxinia, either by starting the spin-up with a different initial PO4 inventory, e.g.
\vspace{-10pt}\begin{verbatim}bg_ocn_init_8=4.3E-06\end{verbatim}\vspace{-10pt}
will give approximately x2 modern PO4, or for a continuing experiment, an increase in the global PO4 inventory can be specified by e.g.
\vspace{-10pt}\begin{verbatim}bg_ocn_dinit_8=2.159E-06\end{verbatim}\vspace{-10pt}
which will add a uniform PO4 concent5ration to the ocean equivalent to the modern inventory (i.e. resulting in a doubling compared to modern of the PO4 inventory).

\noindent Water column euxinia can also be promoted by decreasing the atmospheric O2 concentration, which is set at spin-up by the parameter:
\vspace{-10pt}\begin{verbatim}ac_atm_init_6=0.2095\end{verbatim}\vspace{-10pt}

\noindent \textbf{Relevant HOW-TO:} --

%---------------------------------------------------------------------------------------------------------------------------------
%---------------------------------------------------------------------------------------------------------------------------------

\subsection{Late Permian 36x36x16 ocean + full CaCO3 cycle}

The following examples use a late Permian continental configuration, with a basic (PO$_{4}$-only nutrient) based ocean carbon cycle together with carbonate (CaCO$_{3}$) sedimentation and burial and weathering input.
Note that this particular example does not include a methane cycle (nor CH$_{4}$ radiative forcing).
\\ The purpose of these 3 stages of \textit{spin-up} are as follows:

        \begin{compactenum}
        
                \item The first stage of \textit{spin-up} creates an equilibrium climate and ocean circulation consistent with the prescribed value of atmospheric pCO2. It is also used to force ocean chemistry to give a set mean surface ocean saturation state.
                
                \item The second stage of \textit{spin-up} is used to determine the value of the scaling factor that relates surface ocean saturation to rate of shallow water CaCo3 deposition. Typically, more than 1 iteration may be required to obtain a really good estimate.
                
                \item The third and last stage of \textit{spin-up} is sort of optional :o) The situation after stages \#1 and \#2 should be a spin-up ocean and climate, with the required surface ocean saturation in equilibrium with required atmospheric pCO2. However, the carbon isotope system will be out of balance because it is unlikely that the initially restored isotopic composition of atmospheric pCO2 will be consistent with the set values of d13C for volcanic out-gassing and weathered CaCO3. A further spanner in the isotopic works is the DIC flux used to help restore the ocean saturation state, which by default has an isotopic composition of zero per mil. A run rather longer than the residence time of carbon in the system (which can be multiple 100kyr for a highly over-saturated and also high pCO2 system) is required to achieve a balance between input and output.
                
        \end{compactenum}

%---------------------------------------------------------------------------------------------------------------------------------

\subsubsection{CLOSED CaCO3 cycle: 1st stage \textit{spin-up}}\label{EXAMPLE.p0251b.PO4.SPIN0}

This is the first stage of a 3-part \textsl{spin-up} of late Permian weathering-sedimentary CaCO$_{3}$ cycling.

\noindent \textbf{Physics configuration:} GOLDSTEIN ocean + sea-ice + EMBM atmosphere modules with with seasonal insolation forcing. Adjusted: continental configuration, planetary albedo, solar constant, ocean salinity, annual averaged wind stress and winds.

\noindent \textbf{Biogeochemistry configuration:} Basic ocean (and atmosphere) carbon cycle as described \textit{Cao et al.} [2009]. Atmospheric restoring of \textit{p}CO2 (+ $\delta^{13}$C). Forcing of ocean surface saturation state. Shallow water ('reefal') sedimentary CaCO$_{3}$ depositional environments plus climate-dependent silicate and carbonate weathering in a forced closed balance.

\noindent \textbf{Base-config:} The \textit{base-config} file\footnote{Remembering to omit the '\texttt{.config}' when specifying its name.} is:
\vspace{-10pt}\begin{verbatim}cgenie.eb_go_gs_ac_bg_sg_rg_gl.p0251b.BASES.config\end{verbatim}\vspace{-10pt}
and differs from the equivalent standard modern configuration in:
        \begin{compactitem}
        \item A late Permian continental configuration is prescribed, and the grid started at -180E.
                \item Ocean temperatures are initialized at $5\,^{\circ}{\rm C}$:
                \\ \texttt{go\_10=5.0}, \texttt{go\_11=5.0}.
                \item Solar constant reduced by 2.1\% for the late Permian      :
                \\ \texttt{ma\_genie\_solar\_constant=1339.3}.
                \item Planetary albedo adjusted:
                \vspace{-5pt}\begin{verbatim}
ea_albedop_offs=0.230
ea_albedop_amp=0.240
ea_albedop_skew=0.0
ea_albedop_skewp=0
ea_albedop_mod2=-0.000
ea_albedop_mod4=0.000
ea_albedop_mod6=0.250
\end{verbatim}\vspace{-5pt}
\item Ocean salinity reduced by 1 per mil to take into account absence of large land-based ice sheets:
\\ \texttt{go\_saln0=33.9.}
\item By default, CO$_{2}$-climate feedback is set on (\texttt{ea\_36=y}).
        \end{compactitem}

\noindent \textbf{User-config:} 
\\ The associated \textit{user-config}:
\vspace{-10pt}\begin{verbatim}EXAMPLE.p0251b.PO4.SPIN0\end{verbatim}\vspace{-10pt}
has the following noteworthy features (and differences compared to e.g. modern configurations):
\begin{compactitem}
                \item \texttt{--- INORGANIC MATTER EXPORT RATIOS ---}
                \\ Because the late Permian is going to be assumed to be a time prior to any significant pelagic carbonate production and hence a deep-sea CaCO$_{3}$ sink, pelagic CaCO$_{3}$ production is set to zero\footnote{In principal, only the scaling parameter (\texttt{bg\_par\_bio\_red\_POC\_CaCO3}) needs to be set to zero.}:
\vspace{-5pt}\begin{verbatim}
bg_par_bio_red_POC_CaCO3=0.0
bg_par_bio_red_POC_CaCO3_pP=0.0
                \end{verbatim}\vspace{-5pt}
                \item \texttt{--- SEDIMENTS ---}
                \\ A 72x72 resolution sediment topography is assumed and a mask defining reefal grid points specified.
                \\ A series of sediment cores are requested to be created (file: \texttt{sedcore.nc}), defined by the sediment 'save' mask:
\vspace{-5pt}\begin{verbatim}
sg_par_sedcore_save_mask_name='p0251x_save_mask.72x72.reefALL'
                \end{verbatim}\vspace{-5pt}
                which in this example is all reefal points. Note that 2D, coean-sediment data will also still be saved, as will a restart (file: \texttt{\_restart.nc}), which by default is in netCDF format and includes the first 50 sediment layers at all sediment grid points (so partly duplicating both the 2D and \textit{sedcore} output).
                \\ Here, no sediment diagenesis options are selected, meaning that no dissolution of precipitated CaCO3 is possible:
\vspace{-5pt}\begin{verbatim}
sg_par_sed_diagen_CaCO3opt="none"
                \end{verbatim}\vspace{-5pt}
                In addition, no bioturbational mixing is applied to the sediment layers and so the setting of this option is irrelevant. A uniform detrital (non carbonate) flux is specified, although it need not be\footnote{If is isn't, then the only possible compositions of the reefal sediments are zero or 100 wt\% CaCO$_{3}$.}.
\\ One way of defining reef grid points is to specify a depth interval (e.g. 175.0 m) by parameter:
\\\texttt{sg\_par\_sed\_Dmax\_neritic}
\\defining what the maximum water depth of a neritic sedimentary environment is. Any sediment grid points shallower than this and not defined by the reef mask are then assigned to a 2nd category of neritic sediments -- 'mud'. TO avoid any muddy points, the reef mask must encompasses all shallow water grid points.
                Instead, a zero depth for neritic environments can be set and the model 'forced' to accept the locations specified in the reef location mask, as done here:
\vspace{-5pt}\begin{verbatim}
sg_par_sed_reef_mask_name='p0251x_reef_mask.72x72'
sg_ctrl_sed_neritic_reef_force=.TRUE.
sg_par_sed_Dmax_neritic=0.0
                \end{verbatim}\vspace{-5pt}
Having set the locations for reefal CaCO$_{3}$ production and burial, the dominant mineralogy is set, which by default is calcite:
\vspace{-5pt}\begin{verbatim}
sg_par_sed_reef_calcite=.true.
                \end{verbatim}\vspace{-5pt}
                aragonite being selected by setting this to \texttt{.false.}.
                \\ The parameters specifying how the production of reefal CaCO$_{3}$ scale with ambient ocean chemistry are now set:
\vspace{-5pt}\begin{verbatim}
sg_par_sed_reef_CaCO3precip_sf=0.00005
sg_par_sed_reef_CaCO3precip_exp=1.0
                \end{verbatim}\vspace{-5pt}
                with determination of the value of the former being the subject of the the follow-on Examples.
                \\ There is also the option to have abiotic precipitation of CaCO$_{3}$ occurring. Here we avoid it by setting the precipitation rate scale factor to zero\footnote{The other two parameter settings are not actually necessary in this context.}:
\vspace{-5pt}\begin{verbatim}
sg_par_sed_CaCO3precip_sf=0.0
                \end{verbatim}\vspace{-5pt}
Finally, because distributing a large weathering flux across only a relatively restricted number of shallow water grid points can result in quite high sedimentation rates, the default number of array sediment levels used internally to create the \textit{sedcore} outputs is increased:
\vspace{-5pt}\begin{verbatim}
sg_par_n_sedcore_tot_perky=20
                \end{verbatim}\vspace{-5pt}
                which sets the additional number of layers per kyr of run-time that will be allowed. For modern and deep-sea sedimentary configurations, 10 is more than sufficient\footnote{If \textsl{sedcore} results look to contain a pronounced hiatus, it may reflect too layers in the internal array.}.
                \item \texttt{--- WEATHERING ---}
                \\ A \textit{closed system} is specified for this initial \textsl{spin-up}:
\vspace{-5pt}\begin{verbatim}
bg_ctrl_force_sed_closedsystem=.true.
                \end{verbatim}\vspace{-5pt}
                Temperature-only dependence plus baseline fluxes of silicate and carbonate weathering is specified in the bulk of the lines in this section (plus volcanic CO2 out-gassing and isotopic compositions):
\vspace{-5pt}\begin{verbatim}
rg_opt_weather_T_Ca=.true.
rg_opt_weather_T_Si=.true.
rg_par_weather_CaCO3=17.0E+12
rg_par_weather_CaCO3_d13C=3.0
rg_par_outgas_CO2_d13C=3.0
rg_par_ref_T0=17.8
rg_par_weather_CaSiO3=10.5E+12 
rg_par_outgas_CO2=10.5E+12
                \end{verbatim}\vspace{-5pt}
giving a total base (before temperature modification) weathering rate of 27.5 Tmol Ca yr-1.
\\ However, because the system is being run as a \textit{closed system} this is not important as in a closed system, weathering fluxes are automatically and continuously re-scaled in order to exactly balance the burial loss in marine sediments. A single, fixed carbonate weathering flux could have instead been specified and the entire section could have been replaced with something like:
\vspace{-5pt}\begin{verbatim}
rg_par_weather_CaCO3=27.50E+12
rg_par_weather_CaCO3_13C=0.0
                \end{verbatim}\vspace{-5pt}
                \item \texttt{--- GEOCHEM ACCELERATION ---}
                \\ Set not to be used in this particular experiment (which is in any case the default).
                \item \texttt{--- DATA SAVING ---}
                \\ A fairly standard level of output is set:
\vspace{-5pt}\begin{verbatim}
bg_par_data_save_level=4
                \end{verbatim}\vspace{-5pt}
                \item \texttt{--- FORCINGS ---}
                \\ This part is important as it enables the forcing of ocean surface saturation state towards a target:
\vspace{-5pt}\begin{verbatim}
bg_par_forcing_name="pyyyyz.RpCO2_Rp13CO2_FRALK_FDIC_F13DIC_FCa"
                \end{verbatim}\vspace{-5pt}
 Refer to the \texttt{cGENIE.muffin.HOWTO} document for more details.
                \\ Otherwise, consistent with other paleo-geographies for cGENIE, a spatial (annual average) windspeed file plus associated (re)scaling of the air-sea gas transfer coefficient is set:
\vspace{-5pt}\begin{verbatim}
bg_par_windspeed_file="p0251b_windspeed.dat"
bg_par_gastransfer_a=0.7096
                \end{verbatim}\vspace{-5pt}
        \end{compactitem}
        
\noindent \textbf{Pre-requisites:} None, although use of a climate (and ocean biogeochemistry) spin-up would be useful.

\noindent \textbf{Execution}: Command-line launching of the model experiment for a 10000\footnote{10 kyr is about the minimum to spin up a warm past climate. If a N cycle was included, 20 kyr would probably be needed. The fluxes set in the \textsl{forcing} to create the effect of restoring to a given ocean saturation, are fairly aggressive (representing ca. 100x modern weathering) and are capable of achieving any reasonable surface saturation target within 10 kyr.} year integration:
\vspace{-10pt}\small\begin{verbatim}./runmuffin.sh cgenie.eb_go_gs_ac_bg_sg_rg_gl.p0251b.BASES / EXAMPLE.p0251b.PO4.SPIN0 10000\end{verbatim}\normalsize\vspace{-10pt}

\noindent \textbf{Ideas for further development:} Include a CH$_{4}$ cycle and associated radiative forcing.

\noindent \textbf{Relevant HOW-TOs}: 'Include shallow water depositional systems',
\\'Set a specific ocean chemistry or saturation state'.

%---------------------------------------------------------------------------------------------------------------------------------

\subsubsection{OPEN CaCO3 cycle: 2nd stage \textit{spin-up}}\label{EXAMPLE.p0251b.PO4.SPIN1}

This is the second stage of a 3-part \textsl{spin-up} of late Permian weathering-sedimentary CaCO$_{3}$ cycling.

\noindent \textbf{Physics configuration:} GOLDSTEIN ocean + sea-ice + EMBM atmosphere modules with with seasonal insolation forcing. Adjusted: continental configuration, planetary albedo, solar constant, ocean salinity, annual averaged wind stress and winds.

\noindent \textbf{Biogeochemistry configuration:} Basic ocean (and atmosphere) carbon cycle as described \textit{Cao et al.} [2009]. Atmospheric restoring of \textit{p}CO2 (+ $\delta^{13}$C). Shallow water ('reefal') sedimentary CaCO$_{3}$ depositional environments plus climate-dependent silicate and carbonate weathering; free to determine their own balance.

\noindent \textbf{\textit{Base-config}:} \texttt{cgenie.eb\_go\_gs\_ac\_bg\_sg\_rg\_gl.p0251b.BASES}

\noindent \textbf{\textit{User-config}:} The only adjustments to this configuration compared to \texttt{EXAMPLE.p0251b.PO4.SPIN1} are:

\begin{compactitem}

        \item A new estimate of the value required for the rate of reefal CaCO$_{3}$ production is set:
\vspace{-5pt}\begin{verbatim}
sg_par_sed_reef_CaCO3precip_sf=2.102e-005
                \end{verbatim}\vspace{-5pt}
        \item The weathering-sedimentation system is set to 'open:
\vspace{-5pt}\begin{verbatim}
bg_ctrl_force_sed_closedsystem=.false.
                \end{verbatim}\vspace{-5pt}
        \item The specified forcing is now just a simple restoring of atmospheric CO2 (and isotopic composition):
\vspace{-5pt}\begin{verbatim}
bg_par_forcing_name="pyyyyz_RpCO2_Rp13CO2"
bg_par_atm_force_scale_val_3=2800.0E-06
bg_par_atm_force_scale_val_4=-6.5
                \end{verbatim}\vspace{-5pt}
        
\end{compactitem}
                
The value of \texttt{sg\_par\_sed\_reef\_CaCO3precip\_sf} is adjusted by comparing the specified weathering flux to the global burial rate of CaCO$_{3}$ -- the latter can be found in the SEDGEM output in a file called \texttt{seddiag\_misc\_DATA\_GLOBAL.res} (and: \texttt{Total CaCO3 pres} under \texttt{--- REEF SEDIMENT GRID ---}). Simply scale the scaling value for reefal CaCO$_{3}$  production by the ratio of expected burial (i.e. weathering) to achieved burial. This becomes your revised estimate of the parameter value for this experiment. In this example, rather than:
\vspace{-5pt}\begin{verbatim}
sg_par_sed_reef_CaCO3precip_sf=1.772e-005
                \end{verbatim}\vspace{-5pt}
after 1 single iteration, after 2 iterations, we get:
\vspace{-5pt}\begin{verbatim}
sg_par_sed_reef_CaCO3precip_sf=1.704e-005
                \end{verbatim}\vspace{-5pt}
for informing the 23rd experiment (\texttt{EXAMPLE.p0251b.PO4.SPIN2}).
\\ Actually, depending on how far away the global total CaCO3 burial flux in the 1st spin-up was from the total weathering flux, a 2nd iteration of this experiment may be required. For instance, with the parameters given with a value for the reefal CaCO3 burial scaling factor of 2.102e-005, there will be some noticeable drift in the last experiment in the series (\texttt{EXAMPLE.p0251b.PO4.SPIN2}). A 2nd iteration of this same experiment (\texttt{EXAMPLE.p0251b.PO4.SPIN1}) leads to a refined estimate of the scaling factor of 1.803e-005 and which a final refinement to be made and set in \texttt{EXAMPLE.p0251b.PO4.SPIN2}.
\\ Note that even after 2 iterations the balance is not completely perfect, but is as close to the initially prescribed atmospheric pCO2 and to the required mean surface saturation state as makes as difference. 

\noindent \textbf{Pre-requisites:} \texttt{EXAMPLE.p0251b.PO4.SPIN0}

\noindent \textbf{Execution:} 
\vspace{-10pt}\small\begin{verbatim}./runmuffin.sh cgenie.eb_go_gs_ac_bg_sg_rg_gl.p0251b.BASES / EXAMPLE.p0251b.PO4.SPIN1 1000 
EXAMPLE.p0251b.PO4.SPIN0\end{verbatim}\normalsize\vspace{-10pt}
1000 years is suggested as sufficient to allow the rate of CaCO$_{3}$ at the shallow water sites to adjust to any effect of 'opening' the system, but not so long that ocean chemistry will significant drift should the adjusted value of the neritic CaCO$_{3}$ burial parameter still be significantly far away from achieving balance between global weathering and sedimentation rates.

\noindent \textbf{Ideas for further development:} --

\noindent \textbf{Relevant HOW-TOs:} 'Set a specific ocean chemistry or saturation state'.

%---------------------------------------------------------------------------------------------------------------------------------

\subsubsection{OPEN CaCO3 cycle: 3rd stage \textit{spin-up}}\label{EXAMPLE.p0251b.PO4.SPIN2}

This is the third stage of a 3-part \textsl{spin-up} of late Permian weathering-sedimentary CaCO$_{3}$ cycling.

\noindent \textbf{Physics configuration:} GOLDSTEIN ocean + sea-ice + EMBM atmosphere modules with with seasonal insolation forcing. Adjusted: continental configuration, planetary albedo, solar constant, ocean salinity, annual averaged wind stress and winds.

\noindent \textbf{Biogeochemistry configuration:} Basic ocean (and atmosphere) carbon cycle as described \textit{Cao et al.} [2009]. Shallow water ('reefal') sedimentary CaCO$_{3}$ depositional environments plus climate-dependent silicate and carbonate weathering; free to determine their own balance. Accelerated weathering-sedimentation mass balance.

\noindent \textbf{\textit{Base-config}:} \texttt{cgenie.eb\_go\_gs\_ac\_bg\_sg\_rg\_gl.p0251b.BASES}

\noindent \textbf{\textit{User-config}:} The only adjustments to this configuration compared to \texttt{EXAMPLE.p0251b.PO4.SPIN2} are:

\begin{compactitem}

        \item The only really significant change is in the prescription of geochemical acceleration:
\vspace{-5pt}\begin{verbatim}
ma_gem_notyr=10
ma_gem_yr=990
ma_gem_adapt_auto=.false.
\end{verbatim}\vspace{-5pt}
which species a fixed (\texttt{ma\_gem\_adapt\_auto} is \texttt{.false.}) interleaving of non-accelerated and accelerated blocks of time-stepping.
\\ \texttt{ma\_gem\_notyr} sets the duration (years) of each block of non-accelerated tie-stepping.
\\ \texttt{ma\_gem\_yr} sets the duration (years) of each block of-accelerated tie-stepping.
\\ The net results is a 100-fold increase in speed, less the overheads in running the GEMlite code. A 100-fold increase is also not realized in practice because SEDGEM is still called once per year.
\\ Note that by default -- \textit{time-series} results are only save at the start and end of the non-accelerated phases in addition to whatever specified save points fall within an interval of non accelerated running. This is simply because otherwise artifacts appear in the output as BIOGEM switches from calculating an annual mean and saving results only at the end of a year.
\\ Also note that no equivalent adjustment is currently made for \textit{time-slice} saving, hence by default, requested save points that do not fall within an interval of non accelerated running are not saved. It is possible then (but unlucky) to have *no* BIOGEM netCDF \textit{time-slice} output except at the very end of the experiment when a \textit{time-slice} is automatically save anyway. (SEDGEM output saving is unaffected.) A bespoke series of save points can of course be designed and specified to mesh the GEMlite time-step ratioing.

        \item Note that \textit{restoring} of atmospheric composition is no long prescribed. (In fact, no \textit{restoring} is used at all.)
        
        \end{compactitem}
                
However, in addition: the value of \texttt{sg\_par\_sed\_reef\_CaCO3precip\_sf} can be further adjusted / finely tuned if necessary here (see \ref{EXAMPLE.p0251b.PO4.SPIN2}), for example, here:
\vspace{-5pt}\begin{verbatim}
sg_par_sed_reef_CaCO3precip_sf=1.704e-005
                \end{verbatim}\vspace{-5pt}
is the further converged value.

The reference temperature for climate-dependent weathering should also be checked and adjusted if necessary. Although, when using the mean global weathering parameterization, the mean global land surface temperature, which is recorded in the time-series file:
\texttt{biogem\_series\_misc\_SLT.res}
\\should be equal to the reference temperature, parameter:
\texttt{rg\_par\_ref\_T0}
\\because weathering is non-linear in climate, this is true only for a non-seasonal configuration of the model. \textbf{For a seasonal configuration, the reference temperature will be very slightly different from the actual recorded mean annual temperature.}

Actually, it turns out to be a little more complicated than this, and in following these different stages of \textit{spin-up} will still result in a model that drifts slightly in the final stage 500kyr run. The reason is that long-term climate and hence pCo2 is controlled by the silicate weathering component of total weathering, not total weathering. Again, as a result of a seasonal forcing rather than annual average, the non-linearity in the weathering response to temperature deviations complicates things -- this time because the nature of the non-linearity of silicate and carbonate weathering differ. In other words -- matching weathering of 27.5 Tmol ca yr-1 with an equivalent burial flux does not guarantee a fully balance system should the silicate component of the Ca flux not equal volcanic out-gassing. In this example, is does not.

For instance: The baseline total weathering flux of 27.5 Tmol Ca$^{2+}$ yr$^{-1}$ comprises weathering from silicate and carbonate sources. The total DIC weathering flux comprises weathering from carbonate plus volcanic out-gassing\footnote{This is in the 'short-circuit' approximation in which CO$_{2}$ is not explicitly removed from the atmosphere during weathering but rather the net flux of carbon (as DIC)}. As volcanic out-gassing is fixed at 10.5 Tmol C yr$^{-1}$, the initial (at a \textit{p}CO${_2}$ of 2800 uatm) DIC weathering flux is actually 27.11 Tmol C yr$^{-1}$. If the carbonate weathering component of this is 27.11-10.5 = 16.61 Tmol C yr$^{-1}$, in terms of Ca flux, silicate weathering must be 27.5-16.61 = 10.89 Tmol Ca$^{2+}$ yr$^{-1}$. The reason for the small drift in this last experiment is then clear -- at a \textit{p}CO${_2}$ of 2800 uatm, silicate weathering exceeds volcanic CO$_{2}$ out-gassing, meaning that global temperatures (and hence \textit{p}CO${_2}$) must be reduced in order that net (carbonate) carbon removal matches the rate of new (volcanic) carbon input. In this particular experiment, the decline in \textit{p}CO${_2}$ is to 2660 uatm, with ocean surface saturation declining from 10.0 to 9.9. More careful calibration of the reference temperature parameter could resolve this.

A series of brief e.g. 10 (or even just 2) year experiments (not provided explicitly here) but using the same parameters as per:
\vspace{-5pt}\begin{verbatim}
EXAMPLE.p0251b.PO4.SPIN2
\end{verbatim}\vspace{-5pt}
can resolve all of this as follows:

        \begin{compactenum}
        
                \item Running for \texttt{EXAMPLE.p0251b.PO4.SPIN2} for 2 years with \texttt{EXAMPLE.p0251b.PO4.SPIN1} as a \textit{restart} results in the following weathering fluxes averaged over the 2nd year:
                \\Ca$^{2+}$ = 0.2740854E+14 mol yr$^{-1}$
                \\DIC = 0.2706012E+14 mol yr$^{-1}$
                \\Subtracting the volcanic CO$_{2}$ flux from the total DIC flux gives the carbonate weathering flux, which subtracted from the total Ca$^{2+}$ give fives the silicate weathering component, i.e.:
                \\0.2740854E+14-(0.2706012E+14-10.5e12) = 1.084842000000000e+013 mol yr$^{-1}$ and greater than that required to balance the prescribed volcanic CO$_{2}$ flux.
                
                \item Some trial-and-error with the reference temperature (T$_{0}$) leads to a revised value of 18.41 and resulting total weathering fluxes of:
                \\Ca$^{2+}$ = 0.2680189E+14 mol yr$^{-1}$
                \\DIC = 0.2680124E+14 mol yr$^{-1}$
                \\implying a silicate weathering component of: 1.049935000000000e+013 mol yr$^{-1}$ and now matching the prescribed volcanic CO$_{2}$ flux.
                
                \item A final adjustment of \texttt{sg\_par\_sed\_reef\_CaCO3precip\_sf} can now be made, as the global burial flux (tuned previously) is 0.275656E+14 mol Ca$^{2+}$ yr$^{-1}$ compared to a revised total weathering flux of 0.2679969E+14 Ca$^{2+}$ yr$^{-1}$, meaning that the final carbonate deposition scaling is:
                \\ (0.2679969E+14/0.275656E+14)*1.704e-005 = 1.656654372115971e-005
                
        \end{compactenum}

Of course -- now the total global weathering flux is not quite the 27.5 Tmol yr$^{-1}$ envisioned at the outset ... Further iterations adjusting the baseline carbonate weathering flux could now be made, but is it really worth it? Use of a non-seasonal climate is the only *simple* way of achieving specified carbonate and silicate weathering fluxes in balance with volcanic CO$_{2}$ out-gassing.

Now (in this long open system \textit{spin-up}) is the time to ensure a full equilibrium of the d13C system. Either ensure that the d13C of volcanic CO2 out-gassing and of weathering CaCO3 are consistent with data-based estimates, *or*, set both to achieve a mean d13C of deposited CaCO3 that is consistent with observations.

\noindent \textbf{Pre-requisites:} \texttt{EXAMPLE.p0251b.PO4.SPIN1}

\noindent \textbf{Execution:} 
\vspace{-10pt}\small\begin{verbatim}./runmuffin.sh cgenie.eb_go_gs_ac_bg_sg_rg_gl.p0251b.BASES / EXAMPLE.p0251b.PO4.SPIN2 500000 
EXAMPLE.p0251b.PO4.SPIN1\end{verbatim}\normalsize\vspace{-10pt}

\noindent \textbf{Ideas for further development:} --

\noindent \textbf{Relevant HOW-TOs:} --

%---------------------------------------------------------------------------------------------------------------------------------
%--- cGENIE EXPERIMENTS: FUTURE -------------------------------------------------------------------------
%---------------------------------------------------------------------------------------------------------------------------------

\newpage
\section{Example experiments -- \textbf{future climate change}}\label{example_experiments_future}

Model experiments involving future climate change and particularly fossil fuel CO$_{2}$ emissions.

%---------------------------------------------------------------------------------------------------------------------------------
%---------------------------------------------------------------------------------------------------------------------------------

\subsection{CO2 emissions -- RCP6.0}\label{EXAMPLE.worjh2.Caoetal2009.RCP6p0}

This experiment will drive the atmospheric CO$_{2}$ concentration following RCP6.0.
\\Note that the RCP family of scenarios are based on a specific time-history of radiative forcing, itself driven by changes in greenhouse gases. The greenhouse gases include CH$_{4}$ and N$_{2}$O etc in addition to CO2, hence by following the prescribed CO2 curve, the radiative warming and hence climate warming is slightly less than defined by the RCP. An alternative would be to assume all radiative forcing is driven by CO2, and the RCPs give this equivalent concentration time-history, but it is not a methodology that is 'CMIP5 recommended' (hence here a CO$_{2}$ concentration pathway and hence e.g. ocean acidification is assumed consistent with CMIP5 whilst recognizing that the warming may be slightly under-estimated).
\\Also note that the RCP concentration pathway definitions start at year 1765 (1765.5 mid-point). Hence, they become a substitute for a separate historical spin-up such as defined by \texttt{EXAMPLE.worjh2.Caoetal2009.historical}. One could instead the RCP scenario at e.g. year 2010 (following a separate historical spin-up) by changing the start year:
\texttt{bg\_par\_misc\_t\_start=2010.0} \\There would then be a slightly mis-match in atmospheric CO$_{2}$ concentrations at year 2010, likely due to that the RCP is a smothered fit to the data compared to year-by-year observations.

\noindent \textbf{Physics configuration:} As per \textit{Cao et al.} [2009] -- GOLDSTEIN ocean + sea-ice + EMBM atmosphere modules with seasonal insolation forcing.

\noindent \textbf{Biogeochemistry configuration:} Basic ocean (and atmosphere) carbon cycle with a single (PO$_{4}$) limiting nutrient in the ocean, is as described by \textit{Cao et al.} [2009].

\noindent \textbf{\textit{Base-config}:} \texttt{cgenie.eb\_go\_gs\_ac\_bg.worjh2.BASE}

\noindent \textbf{\textit{User-config}:} The \textit{user-config} is \texttt{EXAMPLE.worjh2.Caoetal2009.RCP6p0},
\\ which is similar to \texttt{EXAMPLE.worjh2.Caoetal2009.historical} except with the substituted forcing:

\vspace{-10pt}\begin{verbatim}
bg_par_forcing_name="pyyyyz.RpCO2_Rp13CO2.RCP6p0"
\end{verbatim}\vspace{-10pt}

\noindent \textbf{Pre-requisites:} \texttt{EXAMPLE.worjh2.Caoetal2009.SPIN}

\noindent \textbf{Execution:} 
\vspace{-10pt}\small\begin{verbatim}./runmuffin.t100.sh cgenie.eb_go_gs_ac_bg.worjh2.BASE / EXAMPLE.worjh2.Caoetal2009.RCP6p0  
300 EXAMPLE.worjh2.Caoetal2009.SPIN\end{verbatim}\normalsize\vspace{-10pt}

\noindent \textbf{Ideas for further development:}
Instead of driving the model with a prescribed CO$_{2}$ concentration pathway in an RCP-compatible experiment, \textit{c}GENIE can instead be driven with a time-history of CO$_{2}$ emissions that achieves the same specific CO$_{2}$ concentration pathway. In fact, internally, the model has already had to do this, because it follows a specific concentration pathway by continually adding (or subtracting) CO$_{2}$ to (/from) the atmosphere. \textit{c}GENIE saves this information in the time-series file:
\texttt{biogem\_series\_diag\_misc\_specified\_forcing\_pCO2.res}
which lists year against annual flux of CO$_{2}$, for each of the years for which a time-series save point is requested.
\\To turn this experiment around so that the model is forced by CO$_{2}$ emissions (rather than having \textit{p}CO$_{2}$ restored) -- simply copy-and-paste the diagnosed flux time-series into a flux forcing file, such as the example given in:
\vspace{-10pt}\begin{verbatim}
pyyyyz.FpCO2_Fp13CO2.RCP6p0
\end{verbatim}\vspace{-10pt}
(which is in fact the diagnosed RCP6.0 emissions obtain using the configuration of \textit{Cao et al.} [2009] under (restoring-forced) RCP6.0).
An example of this (converting a concentration pathway into the equivalent flux forcing) is given in Section~\ref{EXAMPLE.worjh2.Caoetal2009.RCP6p0flux}.
\\Additional RCP scenarios can be created using the official concentration pathway scenario definitions that can be obtained (ASCII or Excel format) from e.g. \url{http://www.pik-potsdam.de/~mmalte/rcps/}\footnote{In the Excel sheets -- column \textbf{3} (labeled 'CO2') is the CMIP5 recommended CO$_{2}$ concentration pathway.}.

\noindent \textbf{Relevant information:} Examples:
\\ \texttt{EXAMPLE.worjh2.Caoetal2009.RCP6p0flux}, 
\\ \texttt{EXAMPLE.worjh2.Caoetal2009.historical}, 
\\ \texttt{EXAMPLE.worbe2.Ridgwelletal1997.RCP6p0}

%---------------------------------------------------------------------------------------------------------------------------------

\subsubsection{CO2 emissions -- RCP6.0 (flux forced)}\label{EXAMPLE.worjh2.Caoetal2009.RCP6p0flux}

This experiment drives the atmospheric CO$_{2}$ concentration following RCP6.0 via a diagnosed flux time-history -- see example:
\texttt{EXAMPLE.worjh2.Caoetal2009.RCP6p0} (Section~\ref{EXAMPLE.worjh2.Caoetal2009.RCP6p0}.)

%---------------------------------------------------------------------------------------------------------------------------------
%--- cGENIE EXPERIMENTS: INVERSIONS -----------------------------------------------------------------
%---------------------------------------------------------------------------------------------------------------------------------

\newpage
\section{Example experiments -- \textbf{inversions}}\label{example_experiments_inversions}

Model experiments involving 'inverting' a record.

%---------------------------------------------------------------------------------------------------------------------------------
%---------------------------------------------------------------------------------------------------------------------------------

\subsection{Inversion of the historical atmospheric d13C record}\label{EXAMPLE.worjh2.PO4.INVERSION}

This experiment will 'invert' the atmospheric pCO$_{2}$ $\delta^{13}$C record, i.e. diagnose the time-history of input and removal of isotopically depleted carbon required for atmospheric pCO$_{2}$ $\delta^{13}$C in the model to track historical observations (used as a restoring forcing).

\noindent \textbf{Physics configuration:} As per \textit{Cao et al.} [2009] -- GOLDSTEIN ocean + sea-ice + EMBM atmosphere modules with with seasonal insolation forcing.

\noindent \textbf{Biogeochemistry configuration:} Basic ocean (and atmosphere) carbon cycle with a single (PO$_{4}$) limiting nutrient in the ocean, is as described by \textit{Cao et al.} [2009].

\noindent \textbf{\textit{Base-config}:} \texttt{cgenie.eb\_go\_gs\_ac\_bg.worjh2.BASE}

\noindent \textbf{\textit{User-config}:} The \textit{user-config} is \texttt{EXAMPLE.worjh2.PO4.INVERSION},
\\ which is similar to \texttt{EXAMPLE.worjh2.Caoetal2009.SPIN} except with the substituted forcing:

\vspace{-10pt}\begin{verbatim}
bg_par_forcing_name="pyyyyz.FRpCO2_FRp13CO2.historical2000"
\end{verbatim}\vspace{-10pt}

\noindent which specifies a flux *and* restoring of atmospheric pCO$_{2}$ and $\delta^{13}$C, which \textit{c}GENIE interprets as requiring a flux of CO$_{2}$ (and associated isotopic composition) to be applied such that atmospheric pCO$_{2}$ $\delta^{13}$C follows specified $\delta^{13}$C restoring forcing.

The diagnosed (integrated) flux of CO$_{2}$ is given in a time-series output:
\vspace{-10pt}\begin{verbatim}
biogem_series_diag_misc_inversion_forcing_FpCO2.res
\end{verbatim}\vspace{-10pt}
for the bulk CO2 flux, and
\vspace{-10pt}\begin{verbatim}
biogem_series_diag_misc_inversion_forcing_FpCO2_13C.res
\end{verbatim}\vspace{-10pt}
for the carbon isotopic signature of this CO2 flux (which in this example is fixed).

Note that this time-series represents the integrated flux of CO$_{2}$ between \textit{time-series} save points. The reason for not saving a flux as mol yr$^{-1}$ is that if this is not saved every single year, adjustments of atmospheric composition (and isotopic composition) will be 'missed' in the output. This output is hence treated slightly different from the other \textit{time-series} and the flux adjustment is integrated over every year and reset (to zero) only when exported to file. Hence, the output represents the integrated flux since the last \textit{time-series} save point. If a \textit{time-series} save point is specified every single year, then the output is simply the annual averaged flux in mol yr$^{-1}$.

Also note that if \texttt{EXAMPLE.worjh2.Caoetal2009.SPIN} is used as a \textit{re-start} -- the configuration in this paper is 'special' in that is used 100 rather than 96 time-steps per year in the ocean circulation model. Hence a variant of the standard run script (\texttt{runmuffin.sh}) is needed: \texttt{runmuffin.t100.sh}.

\noindent \textbf{Pre-requisites:} \texttt{EXAMPLE.worjh2.Caoetal2009.SPIN}

\noindent \textbf{Execution:} 
\vspace{-10pt}\small\begin{verbatim}./runmuffin.t100.sh cgenie.eb_go_gs_ac_bg.worjh2.BASE / EXAMPLE.worjh2.PO4.INVERSION  
300 EXAMPLE.worjh2.Caoetal2009.SPIN\end{verbatim}\normalsize\vspace{-10pt}

\noindent \textbf{Ideas for further development:} -- It is apparent that in the simple inversion described above, the atmospheric CO$_{2}$ value at year 2000 is substantially less than observations. Part of the reason for this is a lack of a terrestrial biosphere component in this configuration. A simple additional carbon reservoir, that represented the terrestrial reservoir, and can provide a buffering of the atmospheric isotopic composition as CO$_{2}$ is exchanged between and surface and atmosphere, is provided. In the base-config files, the (commented out) lines:
\vspace{-10pt}\begin{verbatim}
# Initial terrestrial biosphere inventory (mol) [2300 PgC == 1.92E17 mol]
ac_par_atm_slabbiosphere_C=1.92E17
# Initial terrestrial biosphere inventory d13C (o/oo)
ac_par_atm_slabbiosphere_C_d13C=-6.5
# Flux exchange (mol yr-1) with terrestrial carbon store
ac_par_atm_FterrCO2exchange=1.0e16
\end{verbatim}\vspace{-10pt}accomplish the following:

\begin{compactitem}
        \item \texttt{ac\_par\_atm\_slabbiosphere\_C=1.92E17} -- sets the reservoir size, in this case equal to the modern C inventory of 2300 PgC
        \item \texttt{ac\_par\_atm\_slabbiosphere\_C\_d13C=-6.5} -- sets the carbon isotopic composition of the reservoir. Because there is currently no air-vegetation C fractionation, its value should be chosen equal to the atmospheric signature.
        \item \texttt{ac\_par\_atm\_FterrCO2exchange=1.0e16} -- sets the annual exchange flux of CO$_{2}$. By default this is zero and 'turns off' this feature.
\end{compactitem}

Alternative assumptions regarding the isotopic composition of the carbon input could be made. Examples are given in the \textit{forcings}:
\vspace{-5pt}\begin{verbatim}
pyyyyz.FRpCO2_FRp13CO2.historical2000_hydrateC
pyyyyz.FRpCO2_FRp13CO2.historical2000_maizeC
\end{verbatim}\vspace{-5pt}

\noindent \textbf{Relevant information:} --

%---------------------------------------------------------------------------------------------------------------------------------
%---------------------------------------------------------------------------------------------------------------------------------

\subsection{Inversion of a proxy record for surface ocean DIC d13C -- PETM}\label{EXAMPLE.p0055c.Zeebeetel2014.INVERSION}

This experiment will 'invert' a DIC $\delta^{13}$C record, i.e. diagnose the time-history of input and removal of isotopically depleted carbon required for model mean surface ocean DIC $\delta^{13}$C to match the specified restoring forcing.

\noindent \textbf{Physics configuration:} GOLDSTEIN ocean + sea-ice + EMBM atmosphere modules with with seasonal insolation forcing. Adjusted: continental configuration, planetary albedo, solar constant, ocean salinity, annual averaged wind stress and winds.

\noindent \textbf{Biogeochemistry configuration:} Basic ocean + sediments + weathering (and atmosphere) carbon cycle as described \textit{Ridgwell and Schmidt} [2010], plus weathering feedback.

\noindent \textbf{\textit{Base-config}:} \texttt{cgenie.eb\_go\_gs\_ac\_bg\_sg\_rg.p0055c.BASES}

\noindent \textbf{\textit{User-config}:} \texttt{EXAMPLE.p0055c.Zeebeetel2014.INVERSION} 
\\which is similar to \texttt{EXAMPLE.p0055c.RidgwellSchmidt2010.SPIN2}, with the following exceptions:

\begin{compactitem}
        \item \texttt{--- WEATHERING ---}
        \\ Weathering feedback is prescribed, with the total weathering flux taken from \textit{Ridgwell and Schmidt} [2010].
        \vspace{-5pt}\begin{verbatim}
# set a OPEN system
bg_ctrl_force_sed_closedsystem=.false.
# set CaCO3_weathering-temperature feedback
rg_opt_weather_T_Ca=.TRUE.
# set CaSiO3_weathering-temperature feedback
rg_opt_weather_T_Si=.TRUE.
# weathering reference mean global land surface temperature (C)
rg_par_ref_T0=19.0
#CO2 outgassing rate (mol C yr-1)
rg_par_outgas_CO2=7.5106e+012
# global silicate weathering rate (mol Ca2+ yr-1)
rg_par_weather_CaSiO3=7.5106e+012
# global carbonate weathering rate (mol Ca2+ yr-1)
rg_par_weather_CaCO3=7.5106e+012
# d13C
rg_par_outgas_CO2_d13C=-6.0
rg_par_weather_CaCO3_d13C=13.5
        \end{verbatim}\vspace{-5pt}
        \item \texttt{--- FORCINGS ---}
        \\ The forcing:
        \vspace{-5pt}\begin{verbatim}
bg_par_forcing_name="pyyyyz.FpCO2_Fp13CO2_RDIC_R13DIC.Wright_m22"
        \end{verbatim}\vspace{-5pt}
         specifies a restoring of surface ocean DIC $\delta^{13}$C (and DIC) together with a flux of CO$_{2}$ to the atmosphere, which \textit{c}GENIE interprets as requiring a flux of CO2 (and associated isotopic composition) to be applied such that mean surface ocean DIC $\delta^{13}$C follows specified $\delta^{13}$C restoring forcing.
\\ Here, the time-series of atmospheric CO$_{2}$ is deliberately simplified -- a linear decline in mean surface DIC $\delta^{13}$C of -3.5 per mil over 13 years, and held constant thereafter.
        \end{compactitem}

The diagnosed (integrated) flux of CO$_{2}$ is given in a time-series output:
\vspace{-10pt}\begin{verbatim}
biogem_series_diag_misc_inversion_forcing_FpCO2.res
\end{verbatim}\vspace{-10pt}
for the bulk CO2 flux, and
\vspace{-10pt}\begin{verbatim}
biogem_series_diag_misc_inversion_forcing_FpCO2_13C.res
\end{verbatim}\vspace{-10pt}
for the carbon isotopic signature of this CO$_{2}$ flux (which in this Example is fixed).

Some care must be taken in specifying appropriate maximum fluxes that the model will apply in attempting to adjust the model to follow the target. In this Example, with CO$_{2}$ added to the atmosphere and the target being the surface ocean, there is a lag and degree of disequilibrium between $\delta^{13}$C of ocean and atmosphere, which makes the control of surface ocean DIC $\delta^{13}$C more difficult. Ideally, the inversion would be run twice -- once with a large value of the maximum CO$_{2}$ flux prescribed in the file:
\vspace{-5pt}\begin{verbatim}
biogem_force_flux_ocn_DIC_sig.dat
\end{verbatim}\vspace{-5pt}
in the \textit{forcing},  and a second time with the maximum flux adjusted so that it is slightly (e.g. up to twice) larger than the maximum diagnosed flux\footnote{The flux used in the inversion, is given in the same output as in the previous Example. }. 
For instance -- in this Example, the maximum required flux in the inversion was found in initial experiments to be a little over 100 PgC yr-1 (8.3333e+015 mol yr-1) when the isotopic composition of the source is -22 per mil. The allowed maximum flux was therefore set to 200 PgC yr-1 (1.6667e+016 mol yr-1). A much larger value of 8.3333e+016 mol yr-1 (1000 PgC yr-1) resulted in a varying sign of the input over the first few decades, although it appeared to track the DIC d13C target. Choosing less than the maximum needed, e.g. 8.3333e+015 mol yr-1, while generating an apparently 'better behaved' diagnosed flux history, will create a slight lag between model and target as the full rate of change becomes impossible.

It is important to recognize that this (PEM onset in 13 years) is a rather special and extreme case. A typical d13C inversion might require a maximum CO2 flux of around 1 PgC yr-1 (8.3333e+013 mol yr-1).

Note that this time-series represents the integrated flux of DIC between \textit{time-series} save points. The reason for not saving a flux as mol yr$^{-1}$ is that if this is not saved every single year, adjustments of ocean chemistry (and isotopic composition) will be 'missed' in the output. This output is hence treated slightly different from the other \textit{time-series} and the flux adjustment is integrated over every year and reset (to zero) only when exported to file. Hence, the output represents the integrated flux since the last \textit{time-series} save point\footnote{Technically: the integrated flux s measured between the end of the integration interval, which for an annual average, means that the integration interval is really 0.5 years later than compared to the reported time mid-points.}. If a \textit{time-series} save point is specified every single year, then the output is simply the annual averaged flux in mol yr$^{-1}$.

\noindent \textbf{Pre-requisites:} \texttt{EXAMPLE.p0055c.RidgwellSchmidt2010.SPIN2}.

\noindent \textbf{Execution:} 
\vspace{-10pt}\small\begin{verbatim}./runmuffin.t100.sh cgenie.eb_go_gs_ac_bg_sg_rg.p0055c.BASES / EXAMPLE.p0055c.Zeebeete
l2014.INVERSION 10000 EEXAMPLE.p0055c.RidgwellSchmidt2010.SPIN2\end{verbatim}\normalsize\vspace{-10pt}

\noindent \textbf{Ideas for further development:}

\begin{compactitem}
\item Alternative assumptions regarding the isotopic composition of the carbon input could be made. Examples are given in the \textit{forcings}:
\vspace{-5pt}\begin{verbatim}
pyyyyz.FpCO2_Fp13CO2_RDIC_R13DIC.Wright_m60
\end{verbatim}\vspace{-5pt}
which would add (or remove) carbon to the atmosphere at -60 per mil.
\item Carbon could potentially be added (and removed) to the ocean surface instead of the atmosphere. Selecting one of the following \textit{forcings} accomplishes this:
\vspace{-5pt}\begin{verbatim}
pyyyyz.FRDIC_FR13DIC.Wright_m22
pyyyyz.FRDIC_FR13DIC.Wright_m60
\end{verbatim}\vspace{-5pt}
\item Also, as per the earlier Example, a terrestrial carbon isotopic buffer could potentially be included.
\item Finally, alternative sediment core locations could be used
\\ (ca. texttt{EXAMPLE.p0055c.Panchuketal2013.SPIN2}).
        \end{compactitem}

\noindent \textbf{Relevant information:} --

%---------------------------------------------------------------------------------------------------------------------------------
%---------------------------------------------------------------------------------------------------------------------------------

\subsection{Inversion of a proxy record for surface ocean DIC d13C -- end Permian}\label{EXAMPLE.p0251b.PO4.DIC13CINVERSION}

This experiment will 'invert' a DIC $\delta^{13}$C record, i.e. diagnose the time-history of input and removal of isotopically depleted carbon required for model mean surface ocean DIC $\delta^{13}$C to match the specified restoring forcing.

\noindent \textbf{Physics configuration:} GOLDSTEIN ocean + sea-ice + EMBM atmosphere modules with with seasonal insolation forcing. Adjusted: continental configuration, planetary albedo, solar constant, ocean salinity, annual averaged wind stress and winds.

\noindent \textbf{Biogeochemistry configuration:} Basic ocean (and atmosphere) carbon cycle as described \textit{Cao et al.} [2009]. Shallow water ('reefal') sedimentary CaCO$_{3}$ depositional environments plus climate-dependent silicate and carbonate weathering; free to determine their own balance.

\noindent \textbf{\textit{Base-config}:} \texttt{cgenie.eb\_go\_gs\_ac\_bg\_sg\_rg\_gl.p0251b.BASES}

\noindent \textbf{\textit{User-config}:} The \textit{user-config} is similar to \texttt{EXAMPLE.p0251b.PO4.SPIN2} except for no using any geochemical acceleration, and with the following addition as forcing:

\vspace{-5pt}\begin{verbatim}
bg_par_forcing_name="pyyyyz.FpCO2_Fp13CO2_RDIC_R13DIC.Cuietal2013"
\end{verbatim}\vspace{-5pt}

\noindent which specifies a restoring of surface ocean DIC $\delta^{13}$C (and DIC) together with a flux of CO$_{2}$ to the atmosphere, which \textit{c}GENIE interprets as requiring a flux of CO2 (and associated isotopic composition) to be applied such that mean surface ocean DIC $\delta^{13}$C follows specified $\delta^{13}$C restoring forcing.

In the $\delta^{13}$C restoring forcing example given, which is a carbonate $\delta^{13}$C from the end Permian -- ocean $\delta^{13}$C has not been pre-adjusted (e.g. during the \textit{spin-up}: \texttt{EXAMPLE.p0251b.PO4.SPIN2}), and instead, the initial difference between record and ocean is subtracted from the record. Obviously, it is the ocean $\delta^{13}$C composition that should have been adjusted not the observed record if this was being done properly ...

The diagnosed (integrated) flux of CO$_{2}$ is given in a time-series output:
\vspace{-10pt}\begin{verbatim}
biogem_series_diag_misc_inversion_forcing_FpCO2.res
\end{verbatim}\vspace{-10pt}
for the bulk CO2 flux, and
\vspace{-10pt}\begin{verbatim}
biogem_series_diag_misc_inversion_forcing_FpCO2_13C.res
\end{verbatim}\vspace{-10pt}
for the carbon isotopic signature of this CO2 flux (which in this example is fixed).

\noindent \textbf{Pre-requisites:} \texttt{EXAMPLE.p0251b.PO4.SPIN2}

\noindent \textbf{Execution:} 
\vspace{-10pt}\small\begin{verbatim}./runmuffin.sh cgenie.eb_go_gs_ac_bg_sg_rg_gl.p0251b.BASES / EXAMPLE.p0251b.PO4.DIC13CINVERSION  
100000 EXAMPLE.p0251b.PO4.SPIN2\end{verbatim}\normalsize\vspace{-10pt}

\noindent \textbf{Ideas for further development:} -- The ocean can be instead forced with a flux of DIC to the ocean (rather than CO$_{2}$ to the atmosphere) with an alternative forcing:
\vspace{-5pt}\begin{verbatim}
pyyyyz.FRDIC_FR13DIC.Cuietal2013
\end{verbatim}\vspace{-5pt}
In case, the forcing applied will be saved in a different pair time-series files:
\vspace{-10pt}\begin{verbatim}
biogem_series_diag_misc_inversion_forcing_FDIC.res
\end{verbatim}\vspace{-10pt}
for the bulk DIC flux, and
\vspace{-10pt}\begin{verbatim}
biogem_series_diag_misc_inversion_forcing_FDIC_13C.res
\end{verbatim}\vspace{-10pt}
for the carbon isotopic signature of this DIC flux (which in this example is fixed).

\noindent \textbf{Relevant HOW-TOs:} --

%---------------------------------------------------------------------------------------------------------------------------------
%---------------------------------------------------------------------------------------------------------------------------------

\subsection{Inversion of a proxy record for surface ocean pH -- PETM}\label{EXAMPLE.p0055c.PO4.pH_INVERSION}

This experiment will 'invert' a surface ocean pH record, i.e. diagnose the time-history of input and removal of isotopically depleted carbon required for model mean surface ocean pH to match the specified restoring forcing.

\noindent \textbf{Physics configuration:} GOLDSTEIN ocean + sea-ice + EMBM atmosphere modules with with seasonal insolation forcing. Adjusted: continental configuration, planetary albedo, solar constant, ocean salinity, annual averaged wind stress and winds.

\noindent \textbf{Biogeochemistry configuration:} Basic ocean + sediments + weathering (and atmosphere) carbon cycle as described \textit{Ridgwell and Schmidt} [2010], plus weathering feedback.

\noindent \textbf{\textit{Base-config}:} \texttt{cgenie.eb\_go\_gs\_ac\_bg\_sg\_rg.p0055c.BASESr} -- now including a 'color' tracer that is used to implement the surface ocean pH target.

\noindent \textbf{\textit{User-config}:} \texttt{EXAMPLE.p0055c.PO4.pH\_INVERSION} 
\\ which is similar to \texttt{EXAMPLE.p0055c.RidgwellSchmidt2010.SPIN2}, with the following exceptions:

\begin{compactitem}
        \item \texttt{--- WEATHERING ---}
        \\ Weathering feedback is prescribed, with the total weathering flux taken from \textit{Ridgwell and Schmidt} [2010].
        \vspace{-5pt}\begin{verbatim}
# set a OPEN system
bg_ctrl_force_sed_closedsystem=.false.
# set CaCO3_weathering-temperature feedback
rg_opt_weather_T_Ca=.TRUE.
# set CaSiO3_weathering-temperature feedback
rg_opt_weather_T_Si=.TRUE.
# weathering reference mean global land surface temperature (C)
rg_par_ref_T0=19.0
#CO2 outgassing rate (mol C yr-1)
rg_par_outgas_CO2=7.5106e+012
# global silicate weathering rate (mol Ca2+ yr-1)
rg_par_weather_CaSiO3=7.5106e+012
# global carbonate weathering rate (mol Ca2+ yr-1)
rg_par_weather_CaCO3=7.5106e+012
# d13C
rg_par_outgas_CO2_d13C=-6.0
rg_par_weather_CaCO3_d13C=13.5
        \end{verbatim}\vspace{-5pt}
        \item \texttt{--- FORCINGS ---}
        \\ A specialized forcing is constructed with a color tracer ('red') used to define a restoring forcing of surface ocean pH:
        \vspace{-5pt}\begin{verbatim}
        bg_par_forcing_name="pyyyyz.FpCO2_Fp13CO2_Rcolr.PenmanpHinversion"
        \end{verbatim}\vspace{-5pt}
        \end{compactitem}

\noindent \textbf{Pre-requisites:} \texttt{EXAMPLE.p0055c.RidgwellSchmidt2010.SPIN2}.

\noindent \textbf{Execution:} 
\vspace{-10pt}\small\begin{verbatim}./runmuffin.t100.sh cgenie.eb_go_gs_ac_bg_sg_rg.p0055c.BASESr / EXAMPLE.p0055c.PO4.pH_INVERSION 200000 
EXAMPLE.p0055c.RidgwellSchmidt2010.SPIN2\end{verbatim}\normalsize\vspace{-10pt}

\noindent \textbf{Ideas for further development:} Alternative assumptions regarding the isotopic composition of the carbon input could be made. 

\noindent \textbf{Relevant HOW-TOs:} --

%---------------------------------------------------------------------------------------------------------------------------------
%--- cGENIE EXPERIMENTS: GEOENGINEERING --------------------------------------------------------
%---------------------------------------------------------------------------------------------------------------------------------

\newpage
\section{Example experiments -- \textbf{geoengineering}}\label{example_experiments_geoengineering}

Model experiments involving geoengineeering.

%---------------------------------------------------------------------------------------------------------------------------------
%---------------------------------------------------------------------------------------------------------------------------------

\subsection{Inversion by means of addition/subtraction of CaO (or CaCO3)}

The following examples use a modern configuration to illustrate how mitigation target of time-varying pCO2, mean ocean surface pH, or mean ocean surface saturation state, can be met via the addition (and if requested, subtraction) of alkalinity, with or without Ca2+ and with or without the CO2 attendant.
All experiments can be performed against a background of prescribed CO2 flux emissions and assume starting at year 2010. All either require or are recommended to use a \textit{base-config} incorporating the 'red' and'blue' passive color tracers: \texttt{cgenie.eb\_go\_gs\_ac\_bg.worjh2.BASErb}

The three experiments are:

        \begin{compactenum}
        
                \item \texttt{EXAMPLE.worjh2.LIMNING\_pCO2target} -- ocean limning mitigation towards an atmospheric pCO2 target.
                
                \item \texttt{EXAMPLE.worjh2.LIMNING\_pHtarget} -- ocean limning mitigation towards an ocean surface pH target. This is slightly 'odd' in that it makes use of a color tracer (red) restoring forcing in order to specific a time-history of ocean surface pH as the mitigation 'target'.
                
                \item \texttt{EXAMPLE.worjh2.LIMNING\_sattarget} -- ocean limning mitigation towards an ocean surface saturation state target. This is also slightly 'odd' in that it makes use of a color tracer (blue) restoring forcing in order to specific a time-history of ocean surface pH as the mitigation 'target'.
                
        \end{compactenum}

All \textit{user-config}s adopt standard future emissions setting, including:
\begin{compactitem}
                \item The 'Ocean acidification' data save 'level':
\vspace{-5pt}\begin{verbatim}
bg_par_data_save_level=10
                \end{verbatim}\vspace{-5pt}
                \item A default frequency of time-slice and time-series save time-points applicable to both historical and future experiments:
\vspace{-5pt}\begin{verbatim}
bg_par_infile_slice_name='save_timeslice_historicalfuture.dat'
bg_par_infile_sig_name='save_timeseries_historicalfuture.dat'
                \end{verbatim}\vspace{-5pt}
                \item A year 2010 start time:
\vspace{-5pt}\begin{verbatim}
bg_par_misc_t_start=2010.0
                \end{verbatim}\vspace{-5pt}
        \end{compactitem}

Solute input to the ocean, given as mol yr-1, is the \textbf{maximum} that can be added in any one year in order to attempt to match the target.

Note that other applications of the same methodology are possible, e.g. adjusting ocean chemistry towards a specific value simultaneously with a specific concentration of CO2 in the atmosphere.

%---------------------------------------------------------------------------------------------------------------------------------

\subsubsection{Ocean limning mitigation: atmospheric pCO2 target}\label{EXAMPLE.worjh2.LIMNING.pCO2target}

This is the first variant of ocean limning mitigation towards a specific mitigation target.

\noindent \textbf{Physics configuration:} GOLDSTEIN ocean + sea-ice + EMBM atmosphere modules with with seasonal insolation forcing.

\noindent \textbf{Biogeochemistry configuration:} Basic ocean (and atmosphere) carbon cycle as described \textit{Cao et al.} [2009]. Atmospheric flux forcing of \textit{p}CO2 (+ $\delta^{13}$C). Ocean flux forcing of ALK, Ca2+, and  DIC (+ $\delta^{13}$C).

\noindent \textbf{Base-config:} The \textit{base-config} file\footnote{Remembering to omit the '\texttt{.config}' when specifying its name.} is:
\vspace{-10pt}\begin{verbatim}cgenie.eb_go_gs_ac_bg.worjh2.BASErb.config\end{verbatim}\vspace{-10pt}
and differs from the equivalent standard modern configuration by including the 'red' and 'blue' inert numerical color tracers.

\noindent \textbf{User-config:} The associated \textit{user-config}:
\vspace{-10pt}\begin{verbatim}EXAMPLE.worjh2.LIMNING_pCO2target\end{verbatim}\vspace{-10pt}
has the following noteworthy features (/differences compared to a standard future emissions experiment):
\begin{compactitem}
                \item \texttt{--- FORCINGS ---}
                \\ A specialized forcing is constructed to combined flux forcings of the ocean and atmosphere with an atmospheric concentration target pathway:
\vspace{-5pt}\begin{verbatim}
bg_par_forcing_name="pyyyyz.FRpCO2_Fp13CO2_FRALK_FDIC_F13DIC_FCa"
                \end{verbatim}\vspace{-5pt}
The default CO2 emissions as specified in \texttt{biogem\_force\_flux\_atm\_pCO2\_sig.dat} is a simple 'A2' scenario termination in 2100. Emissions are scaled from PgC yr-1 to molC yr-1 via:
\vspace{-5pt}\begin{verbatim}
bg_par_atm_force_scale_val_3=8.3333e+013
                \end{verbatim}\vspace{-5pt}
                The input of solutes to the ocean surface are then scaled (mol yr-1), with zeros leading to the omission of specific components (e.g. the difference between CaO and CaCO3 is the absence of CO2 associated with the former):
\vspace{-5pt}\begin{verbatim}
bg_par_ocn_force_scale_val_3=0.0
bg_par_ocn_force_scale_val_4=0.0
bg_par_ocn_force_scale_val_12=4000.0E12
bg_par_ocn_force_scale_val_35=2000.0E12
                \end{verbatim}\vspace{-5pt}
Negative fluxes of lime (i.e. preventing removal of ALK from the ocean surface) is set via:
\vspace{-5pt}\begin{verbatim}
bg_ctrl_force_invert_noneg=.true.
                \end{verbatim}\vspace{-5pt}
                The example target provided (\texttt{biogem\_force\_restore\_atm\_pCO2\_sig.dat}) restores mean atmospheric pCO2 to pre-industrial (278 ppm).
        \end{compactitem}
        
\noindent \textbf{Pre-requisites:} \texttt{EXAMPLE.worjh2.Caoetal2009.historical}.
\\ Note that it is perfectly legitimate to carry out less future-relevant experiments starting from a pre-industrial spin-up (e.g. \texttt{EXAMPLE.worjh2.Caoetal2009.SPIN}) but then remembering to remove or change the start year parameter.

\noindent \textbf{Execution}: Following the protocol for the Cao et al. [2009] configuration with its inherent 100 ocean time-steps per year:
\vspace{-10pt}\small\begin{verbatim}./runmuffin.t100.sh cgenie.eb_go_gs_ac_bg.worjh2.BASErb / EXAMPLE.worjh2.LIMNING_pCO2target 
990 EXAMPLE.worjh2.Caoetal2009.historical\end{verbatim}\normalsize\vspace{-10pt}
which creates an experiment out until year 3000 (although with emissions only up until the year 2100).

\noindent \textbf{Ideas for further development:} --

\noindent \textbf{Relevant HOW-TOs}: --

%---------------------------------------------------------------------------------------------------------------------------------

\subsubsection{Ocean limning mitigation: ocean surface pH target}\label{EXAMPLE.worjh2.LIMNING.pHtarget}

This is the second variant of ocean limning mitigation towards a specific mitigation target.

\noindent \textbf{Physics configuration:} GOLDSTEIN ocean + sea-ice + EMBM atmosphere modules with with seasonal insolation forcing.

\noindent \textbf{Biogeochemistry configuration:} Basic ocean (and atmosphere) carbon cycle as described \textit{Cao et al.} [2009]. Atmospheric flux forcing of \textit{p}CO2 (+ $\delta^{13}$C). Ocean flux forcing of ALK, Ca2+, and  DIC (+ $\delta^{13}$C).

\noindent \textbf{Base-config:} The \textit{base-config} file\footnote{Remembering to omit the '\texttt{.config}' when specifying its name.} is:
\vspace{-10pt}\begin{verbatim}cgenie.eb_go_gs_ac_bg.worjh2.BASErb.config\end{verbatim}\vspace{-10pt}
and differs from the equivalent standard modern configuration by including the 'red' and 'blue' inert numerical color tracers.

\noindent \textbf{User-config:} The associated \textit{user-config}:
\vspace{-10pt}\begin{verbatim}EXAMPLE.worjh2.LIMNING_pHtarget\end{verbatim}\vspace{-10pt}
has the following noteworthy features (/differences compared to a standard future emissions experiment):
\begin{compactitem}
                \item \texttt{--- FORCINGS ---}
                \\ A specialized forcing is constructed to combined flux forcings of the ocean and atmosphere with an atmospheric concentration target pathway:
\vspace{-5pt}\begin{verbatim}
bg_par_forcing_name="pyyyyz.FpCO2_Fp13CO2_Rcolr_FRALK_FDIC_F13DIC_FCa"
                \end{verbatim}\vspace{-5pt}
The default CO2 emissions as specified in \texttt{biogem\_force\_flux\_atm\_pCO2\_sig.dat} is a simple 'A2' scenario termination in 2100. Emissions are scaled from PgC yr-1 to molC yr-1 via:
\vspace{-5pt}\begin{verbatim}
bg_par_atm_force_scale_val_3=8.3333e+013
                \end{verbatim}\vspace{-5pt}
                The input of solutes to the ocean surface are then scaled (mol yr-1), with zeros leading to the omission of specific components (e.g. the difference between CaO and CaCO3 is the absence of CO2 associated with the former):
\vspace{-5pt}\begin{verbatim}
bg_par_ocn_force_scale_val_3=0.0
bg_par_ocn_force_scale_val_4=0.0
bg_par_ocn_force_scale_val_12=4000.0E12
bg_par_ocn_force_scale_val_35=2000.0E12
                \end{verbatim}\vspace{-5pt}
Negative fluxes of lime (i.e. preventing removal of ALK from the ocean surface) is set via:
\vspace{-5pt}\begin{verbatim}
bg_ctrl_force_invert_noneg=.true.
                \end{verbatim}\vspace{-5pt}
                The example 'red' color tracer target provided (\texttt{biogem\_force\_restore\_ocn\_colr\_sig.dat}) restores mean (ice-free) ocean surface pH to preindustrial
                \\(as per \texttt{EXAMPLE.worjh2.Caoetal2009.SPIN}).
        \end{compactitem}
        
\noindent \textbf{Pre-requisites:} \texttt{EXAMPLE.worjh2.Caoetal2009.historical}.

\noindent \textbf{Execution}: Following the protocol for the Cao et al. [2009] configuration with its inherent 100 ocean time-steps per year:
\vspace{-10pt}\small\begin{verbatim}./runmuffin.t100.sh cgenie.eb_go_gs_ac_bg.worjh2.BASErb / EXAMPLE.worjh2.LIMNING_pHtarget 
990 EXAMPLE.worjh2.Caoetal2009.historical\end{verbatim}\normalsize\vspace{-10pt}
which creates an experiment out until year 3000 (although with emissions only up until the year 2100).

\noindent \textbf{Ideas for further development:} --

\noindent \textbf{Relevant HOW-TOs}: --

%---------------------------------------------------------------------------------------------------------------------------------

\subsubsection{Ocean limning mitigation: ocean surface saturation target}\label{EXAMPLE.worjh2.LIMNING.sattarget}

This is the third variant of ocean limning mitigation towards a specific mitigation target.

\noindent \textbf{Physics configuration:} GOLDSTEIN ocean + sea-ice + EMBM atmosphere modules with with seasonal insolation forcing.

\noindent \textbf{Biogeochemistry configuration:} Basic ocean (and atmosphere) carbon cycle as described \textit{Cao et al.} [2009]. Atmospheric flux forcing of \textit{p}CO2 (+ $\delta^{13}$C). Ocean flux forcing of ALK, Ca2+, and  DIC (+ $\delta^{13}$C).

\noindent \textbf{Base-config:} The \textit{base-config} file\footnote{Remembering to omit the '\texttt{.config}' when specifying its name.} is:
\vspace{-10pt}\begin{verbatim}cgenie.eb_go_gs_ac_bg.worjh2.BASErb.config\end{verbatim}\vspace{-10pt}
and differs from the equivalent standard modern configuration by including the 'red' and 'blue' inert numerical color tracers.

\noindent \textbf{User-config:} The associated \textit{user-config}:
\vspace{-10pt}\begin{verbatim}EXAMPLE.worjh2.LIMNING_sattarget\end{verbatim}\vspace{-10pt}
has the following noteworthy features (/differences compared to a standard future emissions experiment):
\begin{compactitem}
                \item \texttt{--- FORCINGS ---}
                \\ A specialized forcing is constructed to combined flux forcings of the ocean and atmosphere with an atmospheric concentration target pathway:
\vspace{-5pt}\begin{verbatim}
bg_par_forcing_name="pyyyyz.FRpCO2_Fp13CO2_FRALK_FDIC_F13DIC_FCa"
                \end{verbatim}\vspace{-5pt}
The default CO2 emissions as specified in \texttt{biogem\_force\_flux\_atm\_pCO2\_sig.dat} is a simple 'A2' scenario termination in 2100. Emissions are scaled from PgC yr-1 to molC yr-1 via:
\vspace{-5pt}\begin{verbatim}
bg_par_atm_force_scale_val_3=8.3333e+013
                \end{verbatim}\vspace{-5pt}
                The input of solutes to the ocean surface are then scaled (mol yr-1), with zeros leading to the omission of specific components (e.g. the difference between CaO and CaCO3 is the absence of CO2 associated with the former):
\vspace{-5pt}\begin{verbatim}
bg_par_ocn_force_scale_val_3=0.0
bg_par_ocn_force_scale_val_4=0.0
bg_par_ocn_force_scale_val_12=4000.0E12
bg_par_ocn_force_scale_val_35=2000.0E12
                \end{verbatim}\vspace{-5pt}
Negative fluxes of lime (i.e. preventing removal of ALK from the ocean surface) is set via:
\vspace{-5pt}\begin{verbatim}
bg_ctrl_force_invert_noneg=.true.
                \end{verbatim}\vspace{-5pt}
                The example 'blue' color tracer target provided (\texttt{biogem\_force\_restore\_ocn\_colb\_sig.dat}) restores mean (ice-free) ocean surface calcite saturation to preindustrial
                \\(as per \texttt{EXAMPLE.worjh2.Caoetal2009.SPIN}).
        \end{compactitem}
        
\noindent \textbf{Pre-requisites:} \texttt{EXAMPLE.worjh2.Caoetal2009.historical}.

\noindent \textbf{Execution}: Following the protocol for the Cao et al. [2009] configuration with its inherent 100 ocean time-steps per year:
\vspace{-10pt}\small\begin{verbatim}./runmuffin.t100.sh cgenie.eb_go_gs_ac_bg.worjh2.BASErb / EXAMPLE.worjh2.LIMNING_sattarget 
990 EXAMPLE.worjh2.Caoetal2009.historical\end{verbatim}\normalsize\vspace{-10pt}
which creates an experiment out until year 3000 (although with emissions only up until the year 2100).

\noindent \textbf{Ideas for further development:} --

\noindent \textbf{Relevant HOW-TOs}: --

%---------------------------------------------------------------------------------------------------------------------------------
%--- cGENIE EXPERIMENTS: MISC ------------------------------------------------------------------------------
%---------------------------------------------------------------------------------------------------------------------------------

\newpage
\section{Example experiments -- \textbf{miscellaneous}}\label{example_experiments_misc}

A variety of different model experiment for reference and for use as a helpful starting-point (template) in creating model experiments.\footnote{Remember: when trying different examples -- the first time that a different \textit{base-config} is used, a \texttt{make cleanall} must be done.}

%---------------------------------------------------------------------------------------------------------------------------------
%---------------------------------------------------------------------------------------------------------------------------------

\subsection{Prescribed emission of CO2 into the atmosphere}\label{EXAMPLE_worjh2_PO4Fe_CO2EMISSIONS}

This experiment contains an example emission of CO2 (uniformly) to the atmosphere.

\noindent \textbf{Base-config:} \texttt{genie\_eb\_go\_gs\_ac\_bg\_itfclsd\_16l\_JH\_BASEFe}

\noindent \textbf{User-config:} \texttt{EXAMPLE\_worjh2\_PO4Fe\_CO2EMISSION}
\\ This \textit{user-config} contains:

\begin{compactitem}
        \item A prescribed \texttt{forcing:} \texttt{worjh2\_FpCO2\_Fp13CO2\_FeMahowald2006}, which is configured in the \textit{user-config} as follows:
        \begin{compactenum}
                \item 
                \begin{verbatim}
                bg_par_atm_force_scale_val_03=0.0833e15
                bg_par_atm_force_scale_val_04=-27.0
                \end{verbatim}
                which scale the (unit) emissions as mol per year (0.0833e15 = 1 PgC) and the isotopic composition of the emissions, respectively, and
                \item 
                \begin{verbatim}
                bg_par_atm_force_scale_time_03=1.0E1
                bg_par_atm_force_scale_time_04=1.0E1
                \end{verbatim}
                which scale\footnote{Equal scaling of both tracers must be done.} the duration of a (unit) pulse of emissions which in this example is 10 (1.0x10\^1) years.
        \end{compactenum}
\end{compactitem}

\noindent \textbf{Pre-requisites:} A spin-up such as \texttt{EXAMPLE\_worjh2\_PO4Fe\_SPIN}

\noindent \textbf{Execution:}
\vspace{-5pt}\begin{verbatim}
./runcgenie.sh cgenie_eb_go_gs_ac_bg_itfclsd_16l_JH_BASEFe / 
EXAMPLE_worjh2_PO4Fe_CO2EMISSION 100 EXAMPLE_worjh2_PO4Fe_SPIN
\end{verbatim}\vspace{-5pt}

\noindent \textbf{Ideas for further development}: --

\noindent \textbf{Relevant HOW-TO}: --

%---------------------------------------------------------------------------------------------------------------------------------
%---------------------------------------------------------------------------------------------------------------------------------

\subsection{Prescribed emission of CH4 into the atmosphere}\label{EXAMPLE_worjh2_PO4Fe_CH4EMISSION}

This experiment contains an example emission of CH4 (uniformly) to the atmosphere.

\noindent \textbf{User-config}: \texttt{EXAMPLE\_worjh2\_PO4Fe\_CH4EMISSION}
\\ This \textit{user-config} contains:
\begin{compactitem}
        \item A prescribed \texttt{forcing}: \texttt{worjh2\_FpCH4\_Fp13CH4\_FeMahowald2006}, which is configured in the \textit{user-config} as follows:
\begin{compactenum}
        \item 
        \begin{verbatim}
bg_par_atm_force_scale_val_10=0.0833e15
bg_par_atm_force_scale_val_11=-27.0
                \end{verbatim}
                which scale the (unit) emissions as mol per year (0.0833e15 = 1 PgC) and the isotopic composition of the emissions, respectively, and
        \item 
        \begin{verbatim}
bg_par_atm_force_scale_time_10=1.0E1
bg_par_atm_force_scale_time_11=1.0E1
                \end{verbatim}
                which scale the duration of a (unit) pulse of emissions.
\end{compactenum}
\end{compactitem}

\noindent \textbf{Base-config}: \texttt{genie\_eb\_go\_gs\_ac\_bg\_itfclsd\_16l\_JH\_BASEFeCH4}

\noindent \textbf{Pre-requisites}: A spin-up including a CH4 cycle, such as \texttt{EXAMPLE\_worjh2\_PO4Fe\_CH4\_SPIN}

\noindent \textbf{Execution}:
\vspace{-5pt}\begin{verbatim}
./runcgenie.sh cgenie_eb_go_gs_ac_bg_itfclsd_16l_JH_BASEFeCH4 / 
EXAMPLE_worjh2_PO4Fe_CH4EMISSION 100 EXAMPLE_worjh2_PO4Fe_CH4_SPIN
                \end{verbatim}\vspace{-5pt}

\noindent \textbf{Ideas for further development}: --

\noindent \textbf{Relevant HOW-TO}: --

%---------------------------------------------------------------------------------------------------------------------------------
%---------------------------------------------------------------------------------------------------------------------------------

\subsection{Prescribed injection of DIC at a specific location in the ocean.}\label{EXAMPLE_worjh2_PO4Fe_DICINJECTION}

This experiment contains an example injection of dissolved inorganic carbon (DIC) at a specific point location  in the ocean.

\noindent \textbf{User-config}: \texttt{EXAMPLE\_worjh2\_PO4Fe\_DICINJECTION}
\\ This \textit{user-config} contains:
\begin{compactitem}
        \item A prescribed \texttt{forcing}: \texttt{worjh2\_FDIC\_F13DIC\_FeMahowald2006}, which is configured in the \textit{user-config} as follows:
\begin{compactenum}
        \item 
        \begin{verbatim}
bg_par_ocn_force_scale_val_03=0.0833e15
bg_par_ocn_force_scale_val_04=-27.0
                \end{verbatim}
                which scale the (unit) emissions as mol per year (0.0833e15 = 1 PgC) and the isotopic composition of the emissions, respectively,
        \item 
        \begin{verbatim}
bg_par_ocn_force_scale_time_03=1.0E1
bg_par_ocn_force_scale_time_04=1.0E1
                \end{verbatim}
                which scale the duration of a (unit) pulse of emissions\footnote{Equal scaling of both tracers must be done.}, and
        \item 
        \begin{verbatim}
bg_par_force_point_i=18
bg_par_force_point_j=26
bg_par_force_point_k=7
                \end{verbatim}
                which defines the location of a point source for the emissions\footnote{Note that a point location can instead be set in the \textit{forcing} itself}, which in this example is somewhere at the bottom of the Gulf of Mexico.
\end{compactenum}
\item The specification for the saving of additional 2D data fields for ocean bottom waters: 
\\ \texttt{bg\_ctrl\_data\_save\_slice\_ocnsed=.true.}.
\end{compactitem}

\noindent \textbf{Base-config}: \texttt{genie\_eb\_go\_gs\_ac\_bg\_itfclsd\_16l\_JH\_BASEFe}

\noindent \textbf{Pre-requisites}: A spin-up such as \texttt{EXAMPLE\_worjh2\_PO4Fe\_SPIN}

\noindent \textbf{Execution}:
\vspace{-5pt}\begin{verbatim}
./runcgenie.sh cgenie_eb_go_gs_ac_bg_itfclsd_16l_JH_BASEFe / 
EXAMPLE_worjh2_PO4Fe_DICINJECTION 100 EXAMPLE_worjh2_PO4Fe_SPIN
                \end{verbatim}\vspace{-5pt}

\noindent \textbf{Ideas for further development}:
\begin{compactenum}
        \item A trivial change to the experiment would be to set a different injection location (and/or rate and/or duration) ...
                \item Simple changes can also be made so that DIC is injected to the ocean as a whole or to the surface only (and uniformly). This requires modification of the \textit{forcing} but is relatively straight-forward. All this requires is a change to the file: \texttt{configure\_forcings\_ocn.dat}; '\texttt{COLUMN \#06}'.
                                \item A pattern of DIC injection can also be prescribed: e.g., release at all bottom water locations everywhere, or all bottom-waters in a certain depth range and/or basin, or a surface flux with a specific patter (distribution). [\textbf{See \textit{HOW-TO}}]
\end{compactenum}

\noindent \textbf{Relevant HOW-TO}: ---

%---------------------------------------------------------------------------------------------------------------------------------
%---------------------------------------------------------------------------------------------------------------------------------

\subsection{Prescribed injection of (dissolved) CH4 at a specific location in the ocean.}\label{EXAMPLE_worjh2_PO4Fe_CH4INJECTION}

This experiment describes an example injection of (dissolved) CH4 at a point location in the ocean.

\noindent \textbf{User-config}: \texttt{EXAMPLE\_worjh2\_PO4Fe\_CH4INJECTION}
\\ This \textit{user-config} contains:
\begin{compactitem}
        \item A prescribed \texttt{forcing}: \texttt{worjh2\_FCH4\_F13CH4\_FeMahowald2006}, which is configured in the \textit{user-config} as follows:
\begin{compactenum}
        \item 
        \begin{verbatim}
bg_par_ocn_force_scale_val_25=0.0833e15
bg_par_ocn_force_scale_val_26=-60.0
                \end{verbatim}
                which scale the (unit) emissions as mol per year (0.0833e15 = 1 PgC) and the isotopic composition of the emissions, respectively,
        \item 
        \begin{verbatim}
bg_par_ocn_force_scale_time_25=1.0E1
bg_par_ocn_force_scale_time_26=1.0E1
                \end{verbatim}
                which scale the duration of a (unit) pulse of emissions\footnote{Equal scaling of both tracers must be done.}, and
        \item 
        \begin{verbatim}
bg_par_force_point_i=18
bg_par_force_point_j=26
bg_par_force_point_k=7
                \end{verbatim}
                which defines the location of a point source for the emissions\footnote{Note that a point location can instead be set in the \textit{forcing} itself}, which in this example is somewhere at the bottom of the Gulf of Mexico.
\end{compactenum}
\end{compactitem}

\noindent \textbf{Base-config}: \texttt{genie\_eb\_go\_gs\_ac\_bg\_itfclsd\_16l\_JH\_BASEFeCH4}

\noindent \textbf{Pre-requisites}: A spin-up including a CH4 cycle, such as \texttt{EXAMPLE\_worjh2\_PO4Fe\_CH4\_SPIN}

\noindent \textbf{Execution}:
\vspace{-10pt}\begin{verbatim}
./runcgenie.sh cgenie_eb_go_gs_ac_bg_itfclsd_16l_JH_BASEFeCH4 / 
EXAMPLE_worjh2_PO4Fe_CH4INJECTION 100 EXAMPLE_worjh2_PO4Fe_CH4_SPIN
                \end{verbatim}

\noindent \textbf{Further development ideas}:
\begin{compactenum}
        \item A trivial change to the experiment would be to set a different injection location (and/or rate and/or duration) ...
                \item Simple changes can also be made so that dissolved CH4 is injected to the ocean as a whole or to the surface only (and uniformly). This requires modification of the \textit{forcing} but is relatively straight-forward. All this requires is a change to the file:
                \\ \texttt{configure\_forcings\_ocn.dat}; '\texttt{COLUMN \#06}'.
                                \item A pattern of CH4 injection can also be prescribed: e.g., release at all bottom water locations everywhere, or all bottom-waters in a certain depth range and/or basin, or a surface flux with a specific patter (distribution). [\textbf{See \textit{HOW-TO}}]
                                \item Without a restoring CH4 value in the atmosphere as was specified in:
                                \\\texttt{EXAMPLE\_worjh2\_PO4Fe\_CH4\_SPIN}
                                \\means that the atmospheric CH4 concentration will quickly decay to zero (except in the case of massive prescribed CH4 injections, particularly at depths close to the ocean surface). Adding a an additional (restoring) forcing of a fixed CH4 concentration in the atmosphere will obviously prevent the full impact of CH4 injection in the ocean being simulated. Hence, an additional atmospheric CH4 emission source is required that balances (primarily) atmospheric oxidation to achieve an appropriate initial non-zero (e.g., pre-industrial or modern) concentration of CH4 in the atmosphere prior to injection. This requires a parameter defining a baseline flux of CH4 to the atmosphere to be set (and before that: diagnosed consistent with a steady-state CH4 concentration). [\textbf{See \textit{HOW-TO}}]
\end{compactenum}

\noindent \textbf{Relevant HOW-TO}: ---

%---------------------------------------------------------------------------------------------------------------------------------
%---------------------------------------------------------------------------------------------------------------------------------

\subsection{Prescribed emission of CH4 into the atmosphere (Eocene configuration)}\label{EXAMPLE_p0055c_PO4_CH4EMISSION}

This experiment contains an example emission of CH4 (uniformly) to the atmosphere and is designed as a template for adapting to injection of CH4 in the ocean, and/or emission of CO2 to the atmosphere and/or CO2 injection in the ocean.

\noindent \textbf{User-config}: \texttt{EXAMPLE\_p0055c\_PO4\_CH4EMISSION}
\\ This \textit{user-config} has the following notable features:
\begin{compactitem}
        \item The prescribed \textit{forcing}:
\vspace{-5pt}\begin{verbatim}pyyyyz_FpCO2_Fp13CO2_FpCH4_Fp13CH4\end{verbatim}\vspace{-5pt}
        is generic in that CH4 and/or CO2 can equally (and even simultaneously) emitted to the atmosphere. In this example, the setup is for CH4 emission to the atmosphere and no release prescribed for CO2:
        \begin{verbatim}
bg_par_atm_force_scale_val_03=0.0
bg_par_atm_force_scale_val_04=0.0
bg_par_atm_force_scale_time_03=0.0
bg_par_atm_force_scale_time_04=0.0
bg_par_atm_force_scale_val_10=0.0833e15
bg_par_atm_force_scale_val_11=-60.0
bg_par_atm_force_scale_time_10=1.0E1
bg_par_atm_force_scale_time_11=1.0E1
                \end{verbatim}
\end{compactitem}

\noindent \textbf{Base-config}: \texttt{cgenie\_eb\_go\_gs\_ac\_bg\_hadcm3l\_eocene\_36x36x16\_2i\_080928\_BASECH4}

\noindent \textbf{Pre-requisites}: An Eocene configuration spin-up including a CH4 cycle, such as: \\ \texttt{EXAMPLE\_p0055c\_PO4\_CH4\_SPIN}

\noindent \textbf{Execution}:
\vspace{-5pt}\begin{verbatim}
./runcgenie.sh cgenie_eb_go_gs_ac_bg_hadcm3l_eocene_36x36x16_2i_080928_BASECH4 / 
EXAMPLE_p0055c_PO4_CH4EMISSION 100 EXAMPLE_p0055c_PO4_CH4_SPIN
                \end{verbatim}

\noindent \textbf{Ideas for further development}:
\begin{compactenum}
\item Obviously: one modification is to replace CH4 release with CO2 release. (Or combine to create a simultaneous CO2+CH4 releases.)
\item The same \textit{user-config} can be modified for a CH4 (or CO2) injection into the ocean. For this, the generic forcing:
\vspace{-5pt}\begin{verbatim}pyyyyz_FDIC_F13DIC_FCH4_F13CH4\end{verbatim}\vspace{-5pt}
needs to be specified. The scaling factors for the corresponding CH4 injection\footnote{Note that an injection location must also be specified (as per e.g. \texttt{EXAMPLE\_worjh2\_PO4Fe\_CH4INJECTION} above).} would look like:
\begin{verbatim}
bg_par_atm_force_scale_val_03=0.0
bg_par_atm_force_scale_val_04=0.0
bg_par_atm_force_scale_time_03=0.0
bg_par_atm_force_scale_time_04=0.0
bg_par_atm_force_scale_val_10=0.0833e15
bg_par_atm_force_scale_val_11=-60.0
bg_par_atm_force_scale_time_10=1.0E1
bg_par_atm_force_scale_time_11=1.0E1
                \end{verbatim}
\end{compactenum}

\noindent \textbf{Relevant HOW-TO}: ---

%---------------------------------------------------------------------------------------------------------------------------------
%--- Contact Information -----------------------------------------------------------------------------------------------
%---------------------------------------------------------------------------------------------------------------------------------

\newpage
\section{Contact Information}

\begin{compactitem}
        \item Andy Ridgwell: \texttt{bandy@seao2.org}
\end{compactitem}

%=================================================================================================================================
%=== END DOCUMENT ================================================================================================================
%=================================================================================================================================

\end{document}
